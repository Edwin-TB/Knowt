\section{Fluid Mechanics}

\subsection{Vocabulary}
\begin{itemize}
    \item Pressure and its forces are caused by the behavior of molecules 
    \item Static Fluids - fluids that sit still
    \item THe pressure at any point in a fluid is cause by the weight of the column of fluid above it and the pressure acting on that column
    \item Absolute pressure - the difference between the measured pressure and a vacuum
    \item Gauge pressure - The difference between measured pressure and atmospheric pressure
    \item Archimedes' Principle - The bouyant force on a submerged object is based on the weight of displaced fluid
    \item Pascal's Principle - States an increase in pressure on the surface of a fluid creates an equa; increase in pressure in all points throughout the fluid
    \item Dynamic Fluids - Fluids in motion 
    \item Conservation of mass leads to the continuity equation, which helps us determine the speed of a fluid moving through a pipe with changing cross-sectional area. It can give mass of volumetric flow rate
    \item Conservation of energy leads to Bernoulli's equation, which relates the velocity and pressure of a flowing fluid from one point to another
    \item 
\end{itemize}

\subsection{Now the Nano-World Influences the Fluid World We Live In}
\begin{itemize}
    \item The two major causes of pressure are the thermal motion of molecules and gravity
    \item The vibration of molecules due to thermal energy causes them to collide with any surface the fluid is in contact with, creating pressure, which is always perpendicular to the surface the fluid is in contact with
    \item For two points at different heights in a static fluid, the forces acting on them must cancel out for the to remain stationary, so each point must support the weight of the fluid above it
    \item Since a lower point has to support more force to keep the fluid above it stationary, it has more pressure acting on it
    \item Both of these combine to give the overall pressure in a fluid
\end{itemize}

\subsection{Density}
\begin{itemize}
    \item Density gives how much mass is in a volume unit of an object, shown below: \[\rho=\frac{m}{V}\]
    \item Assume the density of water is \(1000kg/m^3\)
    \item The mass of a fluid is equal to \(\rho*V\) and its weight is equal to \(\rho*V*g\)
\end{itemize}

\subsection{Pressure}
\begin{itemize}
    \item Pressure is defined as force per unit are, shown below: \[P=\frac{F}{A}\]
    \item The unit for pressure is the pascal, defined as: \[1Pa = 1 N/m^2\]
    \item Assume air pressure is 100,000 pascals unless otherwise indicated 
    \item The relationship between absolute and guage pressure is given below: \[P_{Absolute}=P_{Gauge}+100,000Pa\]
\end{itemize}

\subsection{Static Fluids}
\begin{itemize}
    \item The pressure at any point in a fluid is gievn by \[P=P_0+\rho*g*h\] Where \(P_0\) is the presure on the surface of the fluid, \(\rho\) is its density, \(g\) is the acceleration due to gravity, and \(h\) is the height of the column of fluid above it
    \item Notice how pressure does not depend on the shape of the contained or column of fluid above it, but force does
    \item The force on a point in a fluid is given by: \[F=P*A\] Where  \(A\) is the surface area of the object in question 
\end{itemize}

\subsection{Applications of Static Fluids}
\begin{itemize}
    \item When two different fluids are in the same U-tube, their different densities will cause different pressures and result in different heights at different ends of the tube
    \item To find the heigh difference, find a two points with the same pressure acting on them and set the two resulting absolute pressure equations equal to each other, shown below: 
    \[P_0+\rho*g*h_{1}=P_0+\rho*g*h_{2}\]
    \item Pressure is also used in hydraulic jacks, which uses two pistons of different surface areas
    \item A force applied on the smaller piston will result in the same pressure on the larger piston, but since the larger piston has more surface area, it will exert more force 
\end{itemize}

\subsection{Barometer}
\begin{itemize}
    \item 
\end{itemize}

\subsection{Bouyancy and Archimedes' Principles}
\begin{itemize}
    \item 
\end{itemize}

\subsection{Dynamic Fluids - Continuity}
\begin{itemize}
    \item 
\end{itemize}

\subsection{Dynamic Fluids - Bernoulli's Equation}
\begin{itemize}
    \item 
\end{itemize}
