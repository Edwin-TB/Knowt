\section{Astrophysics and Cosmology}
Astrophysics is defined as the use of physics techniques to study celestial bodies. Cosmology is the study of the universe as a whole. 

\subsection{Stars and Galaxies}
\begin{itemize}
    \item Objects in space are so far apart their distances are given in light-years, defined as \[1ly=9.46*10^{15}km\]
    \item \textbf{Nebula} - A large cloud of dust or gas in space
    \item Galaxies tend to group themselves into galaxy clusters, held to gether by gravitational attraction
    \item Clusters then group themselves into superclusters
    \item \textbf{Active Galactic Nuclei} - The very bright points in the center of galaxies, thought to come from the matter falling into the center of the black hole
\end{itemize}

\subsection{Stellar Evolution: Birth and Death of Stars, Nucleosynthesis}
\begin{itemize}
    \item \textbf{Luminosity} - Total power radiated in watts
    \item \textbf{Apparent Brightness} - The power crossing unit area at the Earth perpendicular to the path of the light
    \item Luminosity is spread over a sphere's surface area with its radius being the distance, d, from which it is measured, thus apparent brightness is \[b=\frac{L}{4\pi d^2}\]
    \item The more massive a star is, the greater its luminosity
\end{itemize}

\textbf{H-R Diagram}
\begin{itemize}
    \item The color of a star is related to its luminosity and thus its color is also related to its mass
    \item An H-R diagram graphs the temperature and luminosity of stars in an attempt to identify relationships
\end{itemize}

\textbf{Stellar Evolution; Nucleosynthesis}
\begin{itemize}
    \item \textbf{Stellar Evolution} - The life-cycle that stars go through
    \item Stars are born from large clouds of gas and dust, usually nebulae and start nuclear fusion in their cores once they are massive enough
    \item Once the star has fused much of its hydrogen, it shrinks slightly and releases less energy as heavier nuclei are fused
    \item After this, the star moves to the red giant stage where it grows in volume by a factor of 100 or more
    \item Then, if the star has too low of a mass, it expands more then loses its outer layers, revealing its core, becoming a white dwarf
    \item If the star has a high enough mass, the electrons and protons within it can undergo inverse beta decay, becoming neutrons which make the star much more dense 
    \item Eventually, a neutron star's core will collapse and undergo supernova, forming virtually all the elements of the periodic table
\end{itemize}

\subsection{Distance Measurements}
\textbf{Parallax}
\begin{itemize}
    \item \textbf{Parallax} - The apparent motion of a star against the background of more distance stars
\end{itemize}

\textbf{Parsec}
\begin{itemize}
    \item Defined in terms of seconds of arc, one parsec is \[1pc=3.26ly\]
\end{itemize}

\textbf{Distant Stars and Galaxies}
\begin{itemize}
    \item Further than about 100 light-years, parallax angles are too small to measure
    \item One technique used to measure far object is comparing the apparent brightness of two stars or two galaxies and using an inverse square law to solve for the unknown distance
    \item Another technique uses an H-R diagram by determining a star's surface temperature an using the veritcal axis of the diagram to determine distance
\end{itemize}

\textbf{Distance via SNIa, Redshift}
\begin{itemize}
    \item The largest distances are estimated using Type Ia supernovae, which have similar origins and similar luminosities
\end{itemize}

\subsection{General Relativity: Gravity and the Curvature of Space}
\begin{itemize}
    \item Einstein proved that it is impossible for an observer to tell if an inertial reference frame is stationary or moving in a straight line
    \item \textbf{Principle of Equivalence} - No experiment can be performed that could distinguish between a uniform gravitational field and an equivalent uniform acceleration
    \item Although they are equivalent, inertial and gravitational mass are in fact different
    \item Light is affected by gravity
    \item \textbf{Gravitational Lensing} - Nearby galaxies acting as a magnifying glass that bends light from distant objects towards us 
    \item Einstein's general relativity states that gravitational forces bend space itself, which explains the bending of light
    \item The curvature of space can be measured by summing the angles of a triangle, which may not add up to 180 degrees if space is curved
\end{itemize}

\textbf{Curvature of the Universe}
\begin{itemize}
    \item If the universe has a positive curvature, it would be closed and have a finite volume
    \item If the universe has a curvature of 0 or less, it would be open and could be infinite 
    \item Current suggestions say the universe is so flat that we cannot tell if its curvature is slightly positive or slightly negative
\end{itemize}

\textbf{Black Holes}
\begin{itemize}
    \item Einstein's GR states that massive objects curve the space around them
    \item Black Holes curve space the most, to become one, an object mass must be squeezed into a radius threshold called the Schwarzschild radius, defined as \[R=\frac{2GM}{c^2}\] Where G is the gravitational constant, M is its mass, and c is the speed of light
    \item This radius also gives the event horizon of a black hole, which is the surface from which no emitted signals can be detected from 
\end{itemize}

\subsection{The Expanding Universe: Red shift and Hubble's Law}
\begin{itemize}
    \item The doppler effect for light is given by \[\lambda_{obs}=\lambda_{rest}\sqrt{\frac{1+v/c}{1-v/c}}\]
    \item When light moves away from us, it appears longer and thus appears more red, called \emph{redshift}
    \item The distance a galaxy moves away from us is proportional to its distance, given by \[v=H_0d\] where \(H_0\) is the hubble parameter, defined as \[H_0=21km/s/Mly\]
    \item Another way redshift occurs is through \emph{gravitational redshift}
\end{itemize}

\textbf{Expansion and the Cosmological Principle}
\begin{itemize}
    \item It has been established that the universe looks the same from different points
    \item Every galaxy in the universe is racing away from each other at the Hubble Parameter
    \item \textbf{Cosmological Principle} - The assumption that the universe is isotropic (looks the same from all directions) and homogeneous (looks the same from different locations)
    \item \textbf{Steady State Model} - Assumes the universe is infinitely old and, on average, looks the same now as it always has
    \item The steady state model suggests no large scale changes have occurred in the universe and that uniformity requires mass to be created at a very slow rate
\end{itemize}

\subsection{The Big Bang and the Cosmic Microwave Background}
\begin{itemize}
    \item \textbf{Cosmic Microwave Background} - Evidence towards the big bang, this is radiation that comes from the universe as a whole, measured at \(\lambda = 7.35cm\)
    \item \textbf{Anisotrophy} - Having different properties in different directions, this is a property of the universe that has been confirmed
    \item Once the universe cooled to about 3000K, particles could form atoms and free electrons became rare, which allowed photons to travel freely, leading to the CMB
\end{itemize}

\textbf{Looking Back toward the Big Bang}
\begin{itemize}
    \item \textbf{Lookback Time} - The amount of time it takes light from an event to reach an observer
    \item \textbf{Surface of last scattering} - The edge of the observable universe, we cannot see past this
\end{itemize}

\textbf{The Observable Universe}
\begin{itemize}
    \item Many theories suggest that the entire universe is larger than the observable universe
\end{itemize}

\subsection{The Standard Cosmological Model: Early History of the Universe}
\begin{itemize}
    \item \textbf{The Standard Cosmological Model} - The current most popular theory for the history and current state of the universe
    \item It is believed that after an imbalance of matter and antimatter annihilated most of the antimatter, the universe entered the lepton era where electrons and positrons were formed and destroyed each other, releasing photons
    \item Later, the radiation era began, there were mainly photons and neutrinos but the neutrinos only interacted via the weak force and thus the primary interaction was radiation
    \item The universe then became matter dominated with eh formation of the first atoms and molecules
\end{itemize}

\subsection{Inflation: Explaining Flatness, Uniformity, and Structure}
\textbf{Flatness}
\begin{itemize}
    \item One theory that explains the flatness of the universe is that the universe has a curvature of 0 and as the observable universe expands, we see more of the flatness
\end{itemize}

\textbf{CMB Uniformity}
\begin{itemize}
    \item The CMB is almost entirely uniform, which can be attributed to the fact that in the beginning, the universe was in thermal equilibrium
\end{itemize}

\textbf{Galaxy Seeds, Fluctuations}
\begin{itemize}
    \item Forces can undergo quantum fluctuations that are so small they are not detectable unless magnified
    \item It is believed that inflation magnified the tiny fluctuations of the early universe and created the density irregularities we see today
\end{itemize}

\subsection{Dark Matter and Dark Energy}
\begin{itemize}
    \item \textbf{Big Crunch} - The end of the universe if it has a positive curvature, it would be finite and gravity will eventually overtake to bring everything back together
\end{itemize}

\textbf{Critical Density}
\begin{itemize}
    \item \textbf{Critical Density} - The average mass per volume density threshold that, if surpassed, would give the universe positive curvature, defined as roughly \(10^{-26}kg/m^3\)
    \item We currently believe the universe is expanding at an accelerating rate
\end{itemize}

\textbf{Dark Matter}
\begin{itemize}
    \item Current experiments have many convinced that the universe is flat and that the critical density is exactly the true density, but normal matter only accounts for about 5\% of the critical density's requirements
    \item Dark matter is believed to be the other 95\%, and observations of stars or galaxies in motion suggest they have considerably more mass than previously thought
    \item Dark matter, if it exists, is likely made of weakly interacting massive particles, which makes it very difficult to detect
\end{itemize}

\textbf{Dark Energy - Cosmic Acceleration}
\begin{itemize}
    \item The expansion of the universe is accelerating 
    \item \textbf{Dark Energy} - An unknown force that has been suggested to be responsible for this acceleration
    \item Current estimations on the composition of the universe put the planets and the stars to be about 0.5\% of all the mass-energy available
\end{itemize}

\subsection{Large-Scale Structure of the Universe}
\begin{itemize}
    \item Simulations with cold dark energy and dark matter have fit current observations of the universe quite well
    \item Observations of the CMB have been increasing in precision and soon may be able to detect the effects of gravity waves and provide info for both cosmic inflation and particle physics
\end{itemize}

\subsection{Finally...}
\begin{itemize}
    \item \textbf{Anthropic Principle} - States that if the universe were even a little different than it is, we would not be here
\end{itemize}

\newpage