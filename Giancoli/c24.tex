\section{The Wave Nature of Light}
Light can also be observed as a particle carrying energy. Its wave-particle duality has been used to explain a wide variety of phenomena.

\subsection{Waves vs. Particles; Huygen's Principle and Diffraction}
\begin{itemize}
    \item \textbf{Wave Front} - All points along a multi-dimensional wave that form its crest
    \item \textbf{Huygen's Principle} - Every point on a wave front can be considered as a force of tiny wavelets that spread out in the forwards direction at the speed of the wave itself. The new front is the envelope of all the wavelets
    \item \textbf{Diffraction} - The bending of waves behind obstacles into the shadow region (the area where the wave is not present)
\end{itemize}

\subsection{Huygen's Principle and Law of Refraction}
\begin{itemize}
    \item Snell's law is derived from Huygen's principle
    \item When a light wave travels from one medium to another, its frequency remains constant and its wavelength changes, given by \[\lambda_n=\frac{\lambda}{n}\] Where \(\lambda\) is the wavelength in the first medium and \(n\) is the index of refraction of the second medium
    \item Note that the previous equation is consistent with \(c=f\lambda\)
    \item Wave fronts can be used to describe mirages, on hot days the hot air has a lower refraction index and thus light is bent towards it and creates an illusion
\end{itemize}

\subsection{Interference - Young's Double-Slit Experiment}
\begin{itemize}
    \item \textbf{Monochromatic} - Meaning of one color, this describes plane waves with one frequency
    \item When light waves pass through double-slits, they are diffracted ad interfere with each other on the other side
    \item \textbf{Constructive Interference} - Occurs when waves are \emph{in-phase}, the waves and crests combine to become greater in magnitude
    \item \textbf{Destructive Interference} - Occurs when waves are \emph{out-of-phase}, the waves and crests cancel out and decrease in magnitude
    \item \textbf{Fringe} - A dark or bright line on a screen caused by constructive and destructive interference
    \item To determine where fringes are, the following is used \[d\sin\theta=m\lambda\] Where \(\theta\) is the angle the wave makes with the horizontal, d is the distance between the slits, and m is the order of the fringe
    \item The order of a fringe is how many fringes away a fringe is from the central fringe
\end{itemize}

\textbf{Coherence}
\begin{itemize}
    \item \textbf{Coherent Sources} - Sources of light that produce the same wavelength and frequency
    \item Slits in the double-slit experiement are coherent sources
\end{itemize}

\subsection{The Visible Spectrum and Dispersion}
\begin{itemize}
    \item \textbf{Intensity} - How much energy light carries per unit area
    \item \textbf{Color} - Related to frequency or wavelength
    \textbf{Visible Spectrum} - The wavelengths of light that humans can observe
    \item \textbf{Ultraviolet} - The wavelengths of light too long for humans to detect
    \item \textbf{Infrared} - The wavelengths of light too short for humans to detect
    \item \textbf{Dispersion} - The spreading of white light into the full spectrum
\end{itemize}

\subsection{Diffraction by a Single Slit or Disk}
\begin{itemize}
    \item \textbf{Diffraction Pattern} - The light pattern formed by a sharply-edged object illuminated by a point source
    \item The intensity at light on a surface is graphed by \[\sin\theta\frac{\lambda}{D}\] Where \(\lambda\)is wavelength and D is the width of the slit
    \item The graph of the previous function looks like a sine wave whose magnitude decreases dramatically
    \item The intensity is at a maximum at 0 and a minimum at \(\sin\theta\)
\end{itemize}

\subsection{Diffraction Grating}
\begin{itemize}
    \item \textbf{Diffration Grating} - A large number of equally spaced parallel slits
    \item \textbf{Tranmission Grating} - A diffraction gratins with its own slits
    \item \textbf{Reflection Grating} - Made by ruling lines into a reflective surface 
    \item \textbf{Order} - An integer value that tells how many fringes away from the central fringe a fringe is
    \item Maxima and Minima are much sharper and narrower for a grating than for a double-slit
    \item When light strikes a grating and is not monochromatic, each wavelength will produce maxima at different angles
\end{itemize}

\subsection{The Spectrometer and Spectroscopy}
\begin{itemize}
    \item \textbf{Spectrometer} - A deice used to measure wavelength of light, using the following equation \[\lambda=\frac{d}{m}\sin\theta\] Where m is the order of the maxima and d is the distance between grating slits
    \item \textbf{Line Spectrum} - Specific wavelengths emitted when a gas is heated
    \item \textbf{Absorption Lines} - The specific wavelengths of light a material absorbs
\end{itemize}

\subsection{Interference in Thin Films}
\begin{itemize}
    \item \textbf{Thin-Film Interference} - The constructive interference formed between light reflected from the two surfces of a thin film
    \item This occurs because at a certain angle, light from only one wavelength is reflected and the entire surface produces several wavelengths across the spectrum
    \item \textbf{Newton's Rings} - Concentric rings of light formed when a curved glass comes in contact with a flat glass, caused by the widening air gap between the two surfaces leading to interference of light waves coming from different areas 
    \item \textbf{Phase Shift} - A change in the cycle of a wave (ex. is flips and crests become troughs)
    \item A beam of light, reflected by a material with index of refraction greater than that of the material in which it is traveling, changes phase by \(180^\circ\) or 1/2 cycle
    \item When the air gap is an odd multiple of the wavelength, the waves undergo constructive interference and form the rings
\end{itemize}

\textbf{Colors in a Thing Soap Film}
\begin{itemize}
    \item In a soap film, gravity makes the film thicker at the bottom than at the top, thus it shifts phase more and reflects into different wavelengths across the film
    \item Sometimes, the soap film is thinner than the wavelength of light, so it does not change phase, and remains the same
\end{itemize}

\textbf{Lens Coatings}
\begin{itemize}
    \item Optical instruments use thin lenses to reduce the percentage of light reflected from the surface of the glass
    \item A single coating cannot eliminate all wavelengths, so instruments use several to cover the spectrum
\end{itemize}


\subsection{Michelson Interferometer}
\begin{itemize}
    \item \textbf{Michelson Interferometer} - A device that uses a beam splitter in order to split a beam of light in half
    \item The interferometer reflects light to a detector and the intensity of light is dependent on the distance an object is from the source, in multiples of wavelength
    \item The device is used for finding very small distances between objects as the wavelengths of light used are also very small
\end{itemize}

\subsection{Polarization}
\begin{itemize}
    \item \textbf{Polarized} - Describes a wave that oscillates in a single plane
    \item \textbf{Linearly Polarized} - Describes a wave that oscillates in a single plane
    \item If a polarized wave oscillating on the vertical axis passes through a horizontal slit, it will be stopped
    \item \textbf{Unpolarized} - Describes a wave that oscillates in multiple planes
\end{itemize}

\textbf{Polazroids}
\begin{itemize}
    \item \textbf{Polaroid Sheet} - A sheet composed of long molecules parallel to each other, acts as a series of parallel slits, the axis of the molecules is called the transmission axis
    \item A plane-polarized wave passing through a polaroid sheet with transmission axis at an angle will exit plane-polarized to the sheet and its new intensity is \[I=I_0\cos^2{\theta}\]
    \item \textbf{Polarizer} - Light passes through this device and comes out plane-polarized
    \item \textbf{Analyzer} - Determines is light is polarized and the plane of polarization
\end{itemize}

\textbf{Polarization by Reflection}
\begin{itemize}
    \item Reflected light is preferentially in the plane parallel to the surface it reflects off of
    \item This explains why most polaroid sunglasses have their axes vertical, most surfaces in nature are horizontal and thus the sunglasses block more light 
    \item The amount of polarization is dependent on the \emph{polarization angle}
    \item \textbf{Polarization Angle} - Related to the index of refraction of the two materials on either side of a change in medium, given by \[\tan{\theta}=\frac{n_2}{_1}\] Also known as Brewster's Angle
    \item Brewster's angle is used to determine the angle at which a reflection is perfectly plane polarized, by substituting it into Snell's law
\end{itemize}

\subsection{Liquid Crystal Displays}
\begin{itemize}
    \item \textbf{Liquid Crystal Display (LCD)} - Use polarization to display images on cell phones
    \item The pixels in an LCD use rotation of the polarized axis of light to increase the intensity of light and make it seem brighter
    \item Smaller displays use ambient light to display y reflecting it off the inside of the inside of the pixel
    \item Larger displays use subpixels of red, blue, and green light to make colors
\end{itemize}

\subsection{Scattering of Light by the Atmosphere}
\begin{itemize}
    \item As light from the sun hits the atmosphere, some of it is absorbed by air molecules which reemit it
    \item The reemitted light is plane polarized and someone viewing it at a right angle to the direction of sunlight will see polarized light
    \item Scattering of light depends on the wavelength of light and size of molecules, the shorter the wavelength, the more light is scattered
    \item This is why the sky is blue as blue light has shorter wavelengths
\end{itemize}

\newpage
