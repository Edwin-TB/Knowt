\section{Dynamics: Newton's Laws of Motion}
Dynamics Deals with \emph{Why} objects move, what makes them start to move? What causes them to accelerate or decelerate? We will introduce force and the connection between force and motion

\subsection{Force}
\begin{itemize}
    \item \textbf{Force} - A push or pull that causes an object to move
    \item An object at rest requires force to start to move, to change velocity, or to change direction
    \item Forces have magnitude and direction, they are vectors
    \item The same vector math for velocity applies to forces
\end{itemize}

\subsection{Newton's First Law of Motion}
\begin{itemize}
    \item \textbf{Newton's First Law of Motion} - Every object continues in its state of rest, or of uniform velocity in a straight line, as long as no net force acts on it.
    \item Basically, an object with no force acting on its will remain at rest or in motion
    \item \textbf{Inertia} - An object's tendency to maintain its state of rest or uniform velocity 
    \item Thus, Newton's First Law is often called the \textbf{Law of Inertia}
\end{itemize}
\textbf{Inertial Reference Frames}
\begin{itemize}
    \item Newton's first law depends on the frame of reference
    \begin{itemize}
        \item Example - To an outside observer the driver of an accelerating car is accelerating, but from inside the car, you are at rest
    \end{itemize}
    \item \textbf{Inertial Reference Frames} - A frame of reference from which Newton's first law holds
    \item For most purposes a fixed point on Earth serves as an inertial frame of reference
    \item \textbf{Noninertial Reference Frames} - A frame of reference that does not adhere to Newton's First Law, such as a frame of reference that is acceleration
\end{itemize}

\subsection{Mass}
\begin{itemize}
    \item \textbf{Mass} - Measure of inertia of an object, the more mass something has, the more it will resist a force
    \item Also known as the quantity of matter an object has
    \item The SI unit for mass is the kilogram (kg)
    \item Weight is NOT the same as mass, weight measures the force of gravity an object experiences, weight is dependent on mas
\end{itemize}

\subsection{Newton's Second Law of Motion}
\begin{itemize}
    \item \textbf{Newton's Second Law of Motion} - The acceleration of an object is directly proportional to the net force acting on it, and is inversely proportional to the object's mass. The direction of the acceleration is in the direction of the force acting on the object
    \item In equation form: \[\vec{a}=\frac{\Sigma \vec{F}}{m}\]
    Where $\vec{a}$ is acceleration, $\Sigma \vec{F}$ is the sum of the forces acting on an object, and m is the mass of the object. 
    \item Manipulating the equation to solve for the total force acting on an object gives \[\Sigma\vec{F}=m\vec{a}\]
    \item The sum of force on an object is also known as the \emph{net force} 
    \item Net forces cause acceleration
    \item \textbf{Force} - An action capable of accelerating an object
    \item Force can be written in component form as \[\Sigma\vec{F_x}=m\vec{a_x}\] \[\Sigma\vec{F_y}=m\vec{a_y}\]
\end{itemize}

\subsection{Newton's Third Law of Motion}
\begin{itemize}
    \item \textbf{Newton's Third Law of Motion} - Whenever one object exerts a force on a second object, the second object exerts an equal force in the opposite direction of the first
    \item A force always causes an equal and opposite force from the object being acted on
    \begin{itemize}
        \item Example - A hammer on a nail causes the nail to exert an equal force in the opposite direction of the hammer
    \end{itemize}
    \item The equation for the force of object B on object A if A acts on B is \[\vec{F_{AB}}=-\vec{F_{BA}}\]
\end{itemize}

\subsection{Weight - the Force of Gravity; and the Normal Force}
\begin{itemize}
    \item Earth exerts on all objects on it, gravity, which is the same for all objects and acts downwards
    \item The force of gravity of Earth on an object is represented by
    \[\vec{F_{g}}=m\vec{g}\] Where $\vec{g}$ is the acceleration due to gravity of object on Earth, typically accepted to be $9.80\frac{m}{s^2}$
    \item WHen two objects are in contact, they exert a \emph{contact force} on each other
    \item \textbf{Normal Force} - A contact force that is perpendicular to the common surface of two objects
    \item When an object is on a surface, the object exerts a downward gravitational force on the surface and the surface exerts an upward \emph{normal} force on the object
    \item The normal force exerts by a surface on an object has the same magnitude as the gravitational force from the object, given by \[F_N=mg\] where m is the mass of the object and g is the acceleration due to gravity
    \item Normal forces can also be cause when an object pushes on a surface horizontally instead of vertically
\end{itemize}

\subsection{Solving the Problems with Newton's Laws; Free-Body Diagram}
\begin{itemize}
    \item \textbf{Free=body Diagram} - Also called a force diagram, this is a simple way to represent the forces acting on an object in order to determine the net force on it
    \item \textbf{Net Force} -  To review, this is a the vector sum of all the forces on an object
    \item The object in question in a fre body diagram is treated as a \emph{point particle}. That is, its size and dimensions are not taken into account
\end{itemize}
\textbf{Tension in a Flexible Cord}
\begin{itemize}
    \item \textbf{Tension} - The force a flexible cord experiences when it is pulled on
    \item We treat flexible cords as having "negligible mass" or mass so little that it does not have ot be taken into account
\end{itemize}

\subsection{Problems Involving Friction, Inclines}
\textbf{Friction}
\begin{itemize}
    \item The surfaces of objects are not perfectly smooth, they have small imperfections and bumps that cause them to catch on each other when they slide together, the resistance of movement caused by surfaces is called friction and is a force
    \item \textbf{Kinetic Friction} - Friction between two surfaces in motion, represented by \[F_k=\mu_k F_N\] Where $F_k$ is the force of kinetic friction, $\mu_k$ is the coefficient of kinetic friction between the two surfaces,and $F_N$ is the normal force between the two surfaces
    \item \textbf{Static Friction} - Friction between two surfaces that are not in motion, represented by \[F_s\leq\mu_s F_N\] Where $F_s$ is the force of static friction and $\mu_s$ 
    \item Note that the coefficients of kinetic and statics friction for the same surfaces can and most likely will be different, usually $\mu_s$ will be greater than $\mu_k$
    \item The force of static friction scales with your push until the maximum force is reached, represented by \[F_s=\mu_s F_N\], once this limit is surpassed, kinetic friction will take over
\end{itemize}

\textbf{Inclines}
\begin{itemize}
    \item Objects on an incline experience the force of gravity as a pushing force resisted by friction
    \item The normal force experienced by an object on an incline is given by \[F_N=mg\cos\theta\] WHere $\theta$ is the angle of the incline. Since the force of gravity is at an angle relative to the normal force from the incline
    \item The pushing force experienced by an object on an incline is given by \[F_p=mg\sin\theta\] Which is the other component of gravitational force 
\end{itemize}
\newpage
