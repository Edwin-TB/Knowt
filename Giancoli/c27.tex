\section{Early Quantum Theory and Models of the Atom}
Quantum theory took nearly three decades of effort from many scientists and it began with Planck's quantum hypothesis

\subsection{Discovery and Properties of the Electron}
\begin{itemize}
    \item \textbf{Cathode Rays} - Rays that are emitted by cathode tubes
    \item Cathode rays were discovered to be made of electrons using the below equation \[\frac{e}{m}=\frac{E}{B^2r}\] Where \(r\) is the radius of the cathode tube, \(B\) is the magnetic field, and \(E\) is the electric field within the tube
    \item The ratio given by the above equation is \[\frac{e}{m}1.76*10^{13}C/kg\] and describes the electron
\end{itemize}

\textbf{Electron Charge Measurement}
\begin{itemize}
    \item The charge of an electron is measured to be \(e=1.6*10^{-19}V\)
    \item The mass of an electron is accepted as \(m_e=9.11*10^{-31}kg\)
    \item All other charges are a multiple of \(e\), thus electric charge is \emph{quantized}
\end{itemize}

\subsection{Blackbody Radiation; Plank's Quantum Hypothesis}
\textbf{Blackbody Radiation}
\begin{itemize}
    \item \textbf{Blackbody} - A body that, when cool, would absorb all the radiation falling on it
    \item \textbf{Blackbody Radiation} - Radiation emitted by a blackbody when hot and luminous, used to approximate radiation emitted from real objects
    \item \textbf{Wien's Law} - The peak wavelength emitted by an object at a temperature, \(T\) in Kelvin, given by \[\lambda_pT=2.90*10^{-3}m*K)\]
\end{itemize}

\textbf{Planck's Quantum Hypothesis}
\begin{itemize}
    \item \textbf{Planck's Quantum Hypothesis} - Max Planck related the energy in an atom to the product between its frequency and some minimum value, shown below \[E=nhf\] Where \(h\) is Planck's constant, \(n\) is a quantum number and \(f\) is frequency
    \item \textbf{Planck's Constant} - Common in quantum physics, defined as \[h=6.626*10^{-34}J*s\]
    \item \textbf{Quantum Number} - Meaning discrete amount, this is a positive integer
    \item \textbf{Quantum of energy} - The smallest amount of energy that can exist, defined as the quantity \(hf\)
    \item Energy is also quantized, that is, it increases like stairs rather than a ramp
\end{itemize}

\subsection{Photon Theory and Light and the Photoelectric Effect}
\begin{itemize}
    \item Light contains energy in packets or \emph{quanta}, each with energy \[E=hf\]
    \item \textbf{Photons} - Particles of light, these carry quanta
    \item \textbf{Photoelectric Effect} - The emission of electrons when light hits a surface
    \item The first prediction from the Photoelectric Effect is if light intensity increases, the number of electrons ejected and their maximum KE should be increases because higher intensity means greater electric amplitude which should eject electrons farther
    \item The second assumption is the frequency of light should not affect the KE of the ejected electrons, only intensity
    \item Some work, \(W_0\) must be done to eject an electron from the surface of a metal, thus the input energy of a photon is equal to \[hf=KE+W\]
    \item Photon theory makes three predictions, the first is: an increase in intensity of the light beam means more photons are incident so more electrons will be ejected; but since eney of each photon is not changes, the maximum KE of electrons is not changes
    \item The second is: if the frequency of the light is increased, the maximum KE of the electrons increases linearly, that is \[KE_{max}=hf-W_0\]
    \item The third is: If the frequency is less than the cutoff frequency, \(f_0\), where \(hf_0=W_0\), no electrons will be ejected, no matter the intensity of light
\end{itemize}

\textbf{Applications of the Photoelectric Effect}
\begin{itemize}
    \item The effect is used in electronic motion sensors, as an object interrupts a beam of light, electrons in a sensor circuit stop being ejected and trigger the sensor
    \item Any circuit or device that uses light uses the photo electric effect to gain information from light
    \item \textbf{Photodiode} - Device that detects when an electron is ejected by detecting a change in conductivity 
\end{itemize}

\subsection{Energy, Mass, and the Momentum of a Photon}
\begin{itemize}
    \item Photons always travel at the speed of light and are thus relativistic particles
    \item Calculating the relativistic momentum of a photon gives a denominator of 0, thus we assume they have no mass, which is consistent with \(E=hf\)
    \item The momentum of a photon is calculated instead by \[\rho=\frac{h}{\lambda}\]
\end{itemize}

\subsection{Compton Effect}
\begin{itemize}
    \item \textbf{Compton Effect} - The phenomenon that photons tend to lose energy after passing through a material, indicating a loss in wavelength
    \item The new wavelength is given by \[\lambda '=\lambda+\frac{h}{m_ec}(1-\cos\phi)\] Where \(\lambda '\) is the new wavelength, \(\lambda\) is the original wavelength, \(\phi\) is the change in angle of the photon, and \(m_e\) is the mass of the electron the photon collides with to lose energy
    \item \textbf{Compton Wavelength} - The quantity \(\frac{h}{m_ec}\)
    \item The Compton effect has been used to detect the density of electrons in bone material which in turn can be used to detect bone density
\end{itemize}

\subsection{Photon Interactions; Pair Production}
\begin{itemize}
    \item There are four interactions a photon can undergo, the first is: The photoelectric effect. A photon may knock an electrons out of an atoms and in the process the photon disappears
    \item The second is: the photon may knock an atomic electron to a higher energy state in the atom if its energy is not sufficient to knock the electron out altogether. In this process the photon also disappears. and all its energy is given to the atoms. The atoms is said to be in an excited state
    \item The third is: The photon can be scattered from an electron and lose energy; this is the Compton effect
    \item The fourth is: Pair Production: A photon ca actually create matter, such as the creation of an electron and a positron 
    \item In pair production, the photon disappears and the two particles annihilate each other, releasing their energy as more photons
    \item Note that pair production cannot occur in empty space as momentum would not be conserved
\end{itemize}

\subsection{Wave-Particle Duality; the Principle of Complementarity}
\begin{itemize}
    \item \textbf{Wave-Particle Duality} - The fact that light acts as both a wave and a particle
    \item \textbf{Principle of Complementarity} - The notion that in order to understand an experiment, we might have to interpret the results as light behaving as a wave or particle, therefore the two aspects compliment each other
    \item Einstein's equation \(E=hf\) itself refers the two sides of light, the \(E\) refers to the energy of a particle and the \(f\) refers to the frequency of a wave
\end{itemize}

\subsection{Wave Nature of Matter}
\begin{itemize}
    \item \textbf{de Broglie Wavelength} - The wavelength of a particle in linear motion given by \[\lambda=\frac{h}{\rho}\]
\end{itemize}

\textbf{electron Diffraction}
\begin{itemize}
    \item Electrons have wavelengths on the scale of \(10^{-10}m\) which can be the distance between atoms in a crystal that can serve as a diffraction grating
    \item Using crystals to diffract electrons forms a diffraction pattern
    \item Thus, wave-particle duality applies to both light and matter
\end{itemize}

\subsection{Electron Microscopes}
\begin{itemize}
    \item \textbf{Electron Microscope} - Developed using the wave aspects of electrons, this device can produce images with greater magnification than standard light microscopes
    \item Electrons are focused using electric fields from wires rather than lenses
\end{itemize}

\subsection{Early Models of the Atom}
\begin{itemize}
    \item \textbf{Plum-Pudding Model} - Early ideas of atoms imagined them as dots of negative charge floating in a sea of positive charge
    \item \textbf{Planetary Model} - The gold foil experiment led to the conclusion that atoms are made of massive positively charged centers and surrounded by smaller negatively charged electrons
\end{itemize}

\subsection{Atomic Spectra: Key to the Structure of the Atom}
\begin{itemize}
    \item It was discovered that excited gases emit light, but only at certain wavelengths unique to each gas
    \item \textbf{Line Spectrum} - The specific wavelengths of light an element of compound emits when it is excited
    \item \textbf{Emission Spectrum} - Similar to line spectrum, this serves as a "fingerprint" for the material 
    \item \textbf{Absorption Spectrum} - Shows the wavelengths of light that a gas absorbs, this spectrum is the inverse of an emission spectrum
    \item The spacing between lines on a hydrogen emission spectrum decreases regularly and can be modeled by the Balmer series
    \item \textbf{Balmer Series} - The formula used to determine the spacing between hydrogen's emission spectrum in the visible spectrum, given by \[\frac{1}{\lambda}=R(\frac{1}{2^2}-\frac{1}{n^2}),\ \ \ \ n=3,4,...\] Where \(R\) is the Rydberg constant and \(n\) is any integer greater than 2
    \item \textbf{Rydberg Constant} - Defined as \(R=1.0974*10^7m^{-1}\)
    \item \textbf{Lyman Series} - Similar to the Balmer series, this contains lines in the UV spectrum, given by \[\frac{1}{\lambda}=R(\frac{1}{1^2}-\frac{1}{n^2}),\ \ \ \ n=3,4,...\]
    \item \textbf{Paschen Series} - Similar to the Balmer series, this contains lines in the IR spectrum, given by \[\frac{1}{\lambda}=R(\frac{1}{3^2}-\frac{1}{n^2}),\ \ \ \ n=3,4,...\]
\end{itemize}

\subsection{The Bohr Model}
\begin{itemize}
    \item Niels Bohr proposed that electrons orbit the nucleus but only in discrete orbits and an electron would move in each orbit without radiating energy
    \item \textbf{Stationary States} - The discrete orbits electrons can move about in
    \item In the \emph{Bohr Model}, electrons only emit light when they jump from a higher to lower energy level, a single level jump releases a single photon, \[hf=E_u-E_l\] Where \(u\) indicates upper and \(l\) indicates lower
    \item \textbf{Bohr's Quantum Condition} - Gives the angular momentum of an electron, \[L=mvr_n=n\frac{h}{2\pi},\ \ \ \ n=1,2,3,...\] Where n is an integer, called a \emph{quantum number} and \(r_n\) is the radius of the electron's orbit
    \item \(n\) labels both the orbit radii and \emph{energy level} of an electron
    \item The lowest energy level, \(E_1\) is called the ground state and higher levels are called excited states
\end{itemize}

\textbf{Spectra Lines Explained}
\begin{itemize}
    \item The different series shown before model electrons jumping to an energy state from another \(n\) levels higher
    \item A general formula for the wavelengths of light emitted by a material is given by \[\frac{1}{\lambda}=\frac{2\pi^2Z^2e^4k^2}{h^3c}(\frac{1}{n'^2}-\frac{1}{n^2})\] WHere \(Z\) is the charge on the nucleus o the atom, \(e\) is the charge on the proton, \(k\) is Coulomb's Constant, and \(n'\) corresponds to the series in question (n'=2 for the Balmer Series)
    \item \textbf{Correspondence Principle} - The use of classical mechanics in the quantum world, still yields precise results
\end{itemize}

\subsection{de Broglie's Hypothesis Applied to Atoms}
\begin{itemize}
    \item Applying de Broglie's Principle to electrons proposes the notion that electron orbits are standing waves that can only exist in a whole number of orbits
    \item Another quantum condition is the follwing: \[mvr_n=\frac{nh}{2\pi}\] which gives discrete orbits and energy levels of electrons
\end{itemize}

\newpage
