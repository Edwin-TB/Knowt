\section{DC Circuits}
Electrical circuits are fundamental to modern life, visual representations of electrical circuits use symbols for each component called schematics

\subsection{EMF and Terminal Voltage}
\begin{itemize}
    \item \textbf{Source} - Also called \emph{electromotive force}, this is a device that provides electrical energy
    \item \textbf{EMF} - The potential difference between two terminals of a source
    \item \textbf{Internal Resistance} - The resistance within a source
    \item \textbf{Terminal Voltage} - The voltage between two terminals, calculated by \[V_{ab}=V_a-V_b\]
    \item Chemical reactions in a battey produce a voltage, represtned by \(\xi\) if there is no current being drawn, then \[V_{ab}=\xi-Ir\]
    \item Internal resistance is assumed to be negligible
\end{itemize}

\subsection{Resistors in Series and Parallel}
\begin{itemize}
    \item \textbf{Series} - Describe the connection between resistors when they are connected in a single path, end to end
    \item Total voltage is equal to the sum of the voltages across each resistor \[V=V_1+V_2+...=IR_1+IR_2+...\] or \[V=IR_{eq}\] 
    \item \(R_{eq}\) is the sum of the resistances, for resistors in series this is \[R_{eq}=R_1+R_2+...\]
    \item \textbf{Parallel} - Describes connection in a circuit where the current branches off into multiple paths
    \item Splits in current cause it to break into components, so the total current is \[I=I_1+I_2+...\] Where \(I_n\) is \[\frac{V}{R_n}\]
    \item The sum of resistance in resistors in parallel is \[\frac{1}{R_{eq}}=\frac{1}{R_1}+\frac{1}{R_2}+...\]
\end{itemize}

\subsection{Kirchhoff's Rules}
\begin{itemize}
    \item \textbf{Junction Rule} - at any junction point, the sum of all currents entering the junction must equal the sum of all currents leaving the junction
    \item \textbf{Loop Rule} - The sum of the changes in potential around any closed loop of a circuit must be zero
    \item Roller Coaster analogy - As a roller coaster mves along the track, energy is converted between kinetic and potential (junction rule) and once it returns, its potential energy is the same as it started (loop rule)
\end{itemize}

\subsection{EMFs in Series and in Parallel; Charging a Battery}
\begin{itemize}
    \item Sum of EMFs from sources in series is the sum of the individual EMFs
    \item Batteries in parallel with the same EMF will only have to produce a fraction of the current needed for the circuit, so they wear out less
\end{itemize}

\textbf{Safety when Jump Starting}
\begin{itemize}
    \item When jump starting a car, connect the negative terminal of the good battery with a piece of bare metal on the dead car and connect the positive terminals of each battery
    \item  Do not connect the negative terminal of the good battery to that of the bad one as it may spark hydrogen gas leaking from the bad battery
\end{itemize}

\subsection{Circuits Containing Capacitors in Series and in Parallel}
\begin{itemize}
    \item The equation for capacitors in parallel is \[C_{eq}=C_1+C_2+...\]
    \item The equation for capacitors in series is \[\frac{1}{C_{eq}}=\frac{1}{C_1}+\frac{1}{C_2}+...\]
\end{itemize}

\subsection{RC Circuits - Resistor and Capacitor Series}
\textbf{Capacitor Charging}
\begin{itemize}
    \item \textbf{RC Circuit} - A circuit containing both resistors and capacitors
    \item The voltage across a capacitor is given by \[V_C=\xi(1-e^{-t/RC})\] Where \(\xi\) is the voltage of the source, e is the base of the natural log, R is the resistance of the resistor in the RC circuit, and C is the capacitance
    \item The charge within a capacitor is given by \[Q=Q_0(1-e^{-tRC})\]
    \item \textbf{Time COnstant} - The product of resistance and capacitance in an RC circuit, represented by \(\tau\) \[\tau=RC\]
    \item The voltage across a resistor in an RC circuit is calculated by \[V_R=\xi e^{-t/RC}\]
    Thus current is \[I=\frac{\xi e^{-t/RC}}{R}\]
\end{itemize}

\textbf{Capacitor Discharging}
\begin{itemize}
    \item The voltage in a capacitor as it discharges is \[V_C=V_0e^{-t/RC}\]
    \item The charge as it discharges is \[Q=Q_0e^{-t/RC}\]
\end{itemize}

\textbf{Medical and Other Applications of RC Circuits}
\begin{itemize}
    \item \textbf{Sawtooth Voltage} - A voltage whose graph resembles a saw blade
    \item \textbf{Electronic Pacemaker} - A medical device that uses an RC circuit to deliver evenly spaced electrical pulses to the heart 
\end{itemize}

\subsection{Electric Hazards}
\begin{itemize}
    \item Electric current running through the body may cause burns or stimulate nerves annd muscles
    \item \textbf{Ventricular Fibrillation} - Occurs when a high current passes trhough the heart, it will cause irregular rhythms and blood is not properly pumped
\end{itemize}

\textbf{Safe Wiring}
\begin{itemize}
    \item Power outlets contain 3 wires, the hot wire from which current flows, the neutral wire which carries away current, and the ground wire that connects to ground
    \item \textbf{Circuit Breakers} - Devices that dtecte when too much current flows thruogh a wire, if the current passes over the threshold then it stops current from flowing
    \item \textbf{Leakage Current} - Current that flows along an unintended path
\end{itemize}

\subsection{Ammeters and Voltmeters - Measurement Affects the Quantity being Measured}
\begin{itemize}
    \item \textbf{Ammeter} - A device that measures current
    \item \textbf{Voltmeter} - A device that measures voltage
    \item \textbf{Galvanometer} - An analog ammeter or voltmeter that works by using the force between a magnetic field and a current carrying coil
\end{itemize}

\textbf{How to Connect Meters}
\begin{itemize}
    \item Ammeters must be connect directly within a circuit 
    \item Voltmeters are connect in parallel with the points between which voltage is being measured
\end{itemize}

\textbf{Effects of Meter Resistance}
\begin{itemize}
    \item Sometimes, a meter can give a misleading reading
    \item Voltmeters with high resistances compared to the circuit they are measuring have more accurate readings
    \item The more sensitive a galvanometer is, the less its effect on the circuit is
\end{itemize}

\textbf{Other Meters}
\begin{itemize}
    \item \textbf{Multimeter} - A meter that measure multiple electrical units
    \item \textbf{Ohmeter} - A meter that measures resistance
\end{itemize}

\textbf{Digital Meters}
\begin{itemize}
    \item Digital meters do not use a galvanometer but rather use semiconductor devices to measure
    \item Their precision is extremely high, often around 0.01\%
\end{itemize}

\newpage