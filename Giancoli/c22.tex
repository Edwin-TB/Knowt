\section{Electromagnetic Waves}
The greatest discovery of 19th century electromagnetic theory was that waves of EM fields can travel through space. Which led to an increase in communications worldwide with the advent of the telegraph, radio,and cellphones

\subsection{Changing Electric Fields Produce Magnetic Fields; Maxwell's Equations}
\begin{itemize}
    \item \textbf{Maxwell's Equations} - The basic equations for all electromagnetism
    \item The equations summarized in words are 
    \begin{enumerate}
        \item A general form of Coulomb's law that relates electric field to its source
        \item A similar law for the magnetic field, except that magnetic field lines are always continuous, they do not begin or end
        \item An electric field is produced by a changing magnetic field
        \item A magnetic field is produced by an electric current or by a changing electric field
    \end{enumerate}
\end{itemize}

\textbf{Maxwell's Fourth Equation (Ampere's Law Extended)}
\begin{itemize}
    \item Two different surfaces bound by the same enclosed path and the same magnetic field will have the same current passing through them
    \item \textbf{Displacement Current} - Represented by \(I_D\), this is the changing electric field between two plates of a capacitor
    \item \(I_D\) is given by \[I_D=\epsilon_0\frac{\Delta \phi E}{\Delta t}\]
\end{itemize}

\subsection{Production of Electromagnetic Waves}
\begin{itemize}
    \item \textbf{Electromagnetic Waves} - Waves produced y oscillations in electric and magnetic fields
    \item \textbf{Radiation Field} - The electromagnetic field far away from an antenna, where the fields for loops
    \item The energy carried by EM waves follows a reverse square law
    \item The electric and magnetic fields at any point are perpendicular to each other, and to the direction of wave travel
    \item \textbf{Plane Waves}Farfrom an antenna, a wave is very flat and spread over a large area
    \item EM waves are waves of fields, they can move through empty space
    \item Accelerating electric charges gives rise to EM waves
    \item The speed of EM waves is given b \[v=c=\frac{E}{B}\] Where c is defined as \[\frac{1}{\sqrt{\epsilon_0\mu_0}}\]
\end{itemize}

\subsection{Light as an Electromagnetic Wave and the Electromagnetic Spectrum}
\begin{itemize}
    \item Light is a form of EM waves travels at \[c=~3.00*10^8\]m/s
    \item The freqeuncy and wavelength of EM waves are modeled by \[c=\lambda f\]
    \item The ranges of frequencies of EM waves along with their respective wavelengths are shown in the \emph{electromagnetic spectrum}
    \item If EM waves travel through materials with an electric permittivity and magnetic permeability different that that of free space's, the speed of the waves is given b \[v=\frac{1}{\sqrt{\epsilon\mu}}\]
\end{itemize}

\subsection{Measuring the Speed of Light}
\begin{itemize}
    \item One of the most precise methods of determining the speed of light involved hitting a rotating eight-sided mirror with light
    \item The light would reflect and hit another mirror a large distance away and back
    \item The rotating mirror had to move at a specific rate in order for the reflected light to be measured, using this rate of rotation and the distance of the far mirror was used to determine the speed of light
    \item The speed of light in a vacuum  is given by \[c=2.99792458*10^8\]m/s
\end{itemize}

\subsection{Energy in EM Waves}
\begin{itemize}
    \item The energy stored per unit volume, or energy density, where an EM wave is present is given by \[u=u_E+u_B=\frac{1}{2}\epsilon_0E^2+\frac{1}{2}\frac{B^2}{\mu_0}\] or \[u=\sqrt{\frac{\epsilon_0}{\mu_0}}EB\]
    \item Note that the energy due to the electric and magnetic fields are equal
    \item \textbf{Intensity} - The enegry an EM wave transfers per unit time per unit area
    \item The energy passing through an area at a time due to an EM wave is given by \[\Delta U=(\epsilon_0E)(Ac\Delta t)\]
    \item The average intensity of an EM waves is given by \[I=\frac{E_0B_0}{2\mu_0}\]
\end{itemize}

\subsection{Momentum Transfer and Radiation Pressure}
\begin{itemize}
    \item Maxwell predicted that EM waves exerts a change in momentum when they hit a surface
    \item \textbf{Radiation Pressure} - The force exerted by an EM wave when it encounters a surface
    \item The momentum transferred when radiation is fully absorbed is given by \[\Delta\rho=\frac{\Delta U}{c}\]
    \item The momentum transferred when radiation is fully reflected is given by \[\Delta\rho=\frac{2\Delta U}{c}\]
    \item The radiation pressure when radiation is fully absorbed is given by
    \[P\frac{I}{c}\]
    \item The radiation pressure when radiation is fully reflected is given by
    \[P\frac{2I}{c}\]
    \item \textbf{Optical Tweezers} - Devices that use EM waves to manipulate extremely small objects such as individual components within a living cell
\end{itemize}

\subsection{Radio and Television; Wireless Communication}
\begin{itemize}
    \item \textbf{Audiofrequency} - The frequency at which the audio signal being tranmistted oscillates
    \item \textbf{Carrier Frequency} - The frequency at which the EM waves carrying the audio frequency oscillates
    \item \textbf{Amplitude Modulations (AM)} - A method used to mix aurdio and carrier frequencies by having the amplitude  (height) of the EM waves vary with proportion to the audio signal
    \item \textbf{Frequency Modulation (FM)} - Another method of mixing the signals, which is done by varying the frequency of the carrier signal with proportion to the frequency of the audio signal
    \item Devices receive signals of specific frequencies by adjusting the resonant frequency of an LC cicuit to equal that of a station's carrier frequency
    \item The signal then goes through the demodulator where the audio frequency is separated from the carrier frequency
\end{itemize}

\textbf{Other EM Wave Communications}
\begin{itemize}
    \item Government agencies assign different frequency ranges to different functions such as telecommunications
    \item EM waves can be transmitted through the "air" or through coaxial cables that run through the ground
\end{itemize}

\newpage