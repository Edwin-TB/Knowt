\section{Heat}
Heat is said to "flow" from hot objects to cold objects, this flow eventually causes the objects to be in thermal equilibrium, when there is no more heat flow. Note that heat and temperature are NOT the same. 

\subsection{Heat as Energy Transfer}
\begin{itemize}
    \item \textbf{Calorie} - The amount of heat necessary to raise the temperature of 1 gram of water by 1 degree Celsius.
    \item \textbf{BUT} - Stands for british thermal unit, this is the amount of heat required to raise the temperature of 1lb of water by 1 degree Fahrenheit
     \item \textbf{Mechanical Equivalent of Heat} - An amount of work always has an equivalent amount of heat, the conversion between Joules and Calories is given by \[1.186J=1cal\] \[1.186kJ=1kcal\]
     \item \textbf{Heat} - The energy transferred from one object to another because of a difference in temperature
\end{itemize}

\subsection{Internal Energy}
\begin{itemize}
    \item \textbf{Internal Energy} - The sum of the energy of all the moleculaes in an object, also called \emph{thermal energy}
\end{itemize}

\textbf{Distinguishing Temperature, Heat, and Internal Energy}
\begin{itemize}
    \item Temperature is the average kinetic energy of individual molecules
    \item Internal Energy is the total energy of all molecules
    \item Heat is the transfer of energy from one object to another
\end{itemize}

\textbf{Internal Energy of an Ideal Gas}
\begin{itemize}
    \item The internal energy of the ideal non atomic gas is \[U=\frac{3}{2}nRT\] Where n is the number of moles, R is the universal gas constant, and T is temperature
    \item If molecules contain more than one atom, their vibrational and rotational energies must also be accounted for
    \item The internal energy of liquids is not as simple as therre are electrical potential energies between molecules that must be accounted for
\end{itemize}

\subsection{Specific Heat}
\begin{itemize}
    \item The amount of heat, Q, required to change the temperature of an object is calculated by \[Q=mc\Delta T\] Where m is mass, \(\Delta T\) is the intended change in temperature, and c is the \emph{specific heat}
    \item \textbf{Specific Heat} - Unique to each material, this is used to determine how much temperature will change with a given amount of heat, units of \(\frac{kcal}{kg*^\circ C}\) or \(\frac{cal}{g*^\circ C}\)
    \item Water has one of the highest specific heats of all substances, which gives it many useful applications in insulating or maintaining heat
\end{itemize}

\textbf{Specific Heats for Gases}
\begin{itemize}
    \item Since gases change in volume due to changes in temperatures, specific heat is not very applicable to them
    \item Instead, constant pressure, \(c_p\) and constant volume, \(c_v\), are used, with the same units as specific heat
\end{itemize}

\subsection{Calorimetry - Solving Problems}
\begin{itemize}
    \item \textbf{System} - Object(s) considered in a proble
    \item \textbf{Open System} - A system where mass not leave, if energy does not leave either it is said to be \emph{isolated}
    \item \textbf{Closed System} - A system where mass and energy may leave
    \item \textbf{Thermal Equilibrium} - A state that describes when an entire system has reached the same temperature
    \item For an isolated system, 
    \begin{itemize}
        \item heat lost = heat gained
        \item energy \emph{out} one part = energy \emph{into} another part
        \item The sum of all the heat transfers is equal to zero, \[\Sigma Q=0\] where each Q represents heat entering or leaving a part of the system
    \end{itemize}
    \item \textbf{Calorimetry} - A technique that measures heat exchange, 
    \item\textbf{Calorimeter} = A device that measures heat exchange, used to determine the specific heats of substances
\end{itemize}

\textbf{Bomb Calorimeter}
\begin{itemize}
    \item \textbf{Bomb Calorimeter} - Used to measure thermal energy released from substances burning, also used to measure the caloric content of foods
    \item The caloric content determined from this method may be unreliable as our bodies cannot metabolize all the available energy in a food
\end{itemize}

\subsection{Latent Heat}
\begin{itemize}
    \item Changes of phase always involve energy
    \item \textbf{Heat of fusion} - Refers to the heat required to convert 1.0kg of a substance from solid to liquid
    \item \textbf{Heat of Vaporization} - Refers to the heat required to convert 1.0kg of a substance from liquid to vapor
    \item Heats of fusion and vaporization are also called \emph{latent heats}, which refers to heat released by a substance when it changes from gas to liquid or liquid to solid
    \item Latent heats can be modeled by \[Q=mL\] Where m is the mass of the substance, L is its latent heat, and Q is the heat added or released
\end{itemize}

\textbf{Evaporation}
\begin{itemize}
    \item \textbf{Evaporation} - The process of a liquid converting to a gas at room temperature
    \item Evaporation is a cooling process and is used by our bodies to keep cool
\end{itemize}

\textbf{Kinetic Theory of Latent Heats}
\begin{itemize}
    \item At the point of fusion, latent heat overcomes the attractive forces between molecules and allows them to become liquid
    \item The same at the point of vaporization, where the latent heat allows molecules to spread farther apart
\end{itemize}

\subsection{Heat Transfer: Conduction}
\begin{itemize}
    \item \textbf{Conduction} - A method of heat transfer through direct contact with a heat course
    \item Can be visualized as molecular collisions
    \item Heat flow is given by \[\frac{Q}{t}=kA\frac{T_1-T_2}{l}\] Where \(\frac{Q}{t}\) is the rate of heat flow over time, k is the thermal conductivity of the material, A is the cross sectional area, l is the distance between the hot and cold regions, \(T_1\) is the temperature of the hotter region, and \(T_2\) is the temperature of the cooler region
    \item Thermal conductivity is unique to each material and has units of \(\frac{J}{s*m*^\circ C}\)
    \item Materials with high thermal conductivities are said to be good thermal conductors, they can transfer heat easily
    \item Materials with low thermal conductivities are said to be good thermal insulators, they transfer heat slower
\end{itemize}

\textbf{R-Value for Building Materials}
\begin{itemize}
    \item R-values specify the insulating properties of materials, units of \(ft^2*\)
    \item Larger R values mean a material is a better insulator
\end{itemize}

\subsection{Heat Transfer: Convection}
\begin{itemize}
    \item \textbf{Convection} - Process by which heat flows by bulk movement of molecules from one place to another
    \item For example, hot water expands and is less dense than cold water, it rises, but once it reaches the surface it cools down, gets denser and sinks
    \item Convection is important in home heating, some systems use it to transfer heat from loawr levels of the house to higher ones
\end{itemize}

\subsection{Heat Transfer: Radiation}
\begin{itemize}
    \item \textbf{Radiation} - Heat transferred as radiant energy
    \item \textbf{Stefan-Boltzmann Equation} - The heat flow radiated by a source is calculated by \[\frac{Q}{t}=\epsilon\sigma AT^4\] Where \(\epsilon\) is the emissivity, \(\sigma\) is the Stefan-Boltzmann constant, A is the surface area of the object, and T is its temperature
    \item Emissivity is a value that ranges from 0 to 1 and depends on the surface of the radiating material, if it is darker in color it has an emissivity close to 1 and vice versa
    \item The Stefan-Boltzmann constant is defined as \[5.67*10^-8W/m^2K^4\]
    \item Good absorbers are good emitters
    \item The net rate of radiant heat flow is given by \[\frac{Q}{t}=\epsilon\sigma A(T_1^4-T_2^4)\]
    \item Since the sun is a point source, its radiant heat cannot be calculated using the above equations, instead, the following is used \[\frac{Q}{t}=(1000W/m^2)\epsilon A\cos\theta\]
    WHere \(\theta\) is the angle between the direction of the sun's rays and a line perpendicular to the surface of the object being hit
    \item \textbf{Themography} - A technique that measured the intensity of infrared rays from many points on the body to form a picture
\end{itemize}

\newpage