\section{Electric Charge and Electric Field}
Atomic theory tells us liquids and solids are formed by the electric bonds between molecules. Many forces, including elastic and friction, result from the electric forces acting at the atomic level. Note that gravity is a separate force

\subsection{Static Electricity; Electric Charge and its Conservation}
\begin{itemize}
    \item \textbf{Static Electricity} - An electric charge caused by friction
    \item There are two types of electric charge, positive and negative 
    \item Unlike charges attract, like charges repel
    \item \textbf{Law of Conservation of Electric Charge} - The net amount of electric charge produced in any process is zero, no net charge can be created or destroyed
\end{itemize}

\subsection{Electric Charge in the Atom}
\begin{itemize}
    \item Atoms are made of a positively charged nucleus surrounded by negatively charged electron cloud
    \item Nucleus is made of protons (positive charge), neutrons (no charge, "neutral") and the electron cloud is made of electrons (negative charge)
    \item Atoms can lose or gain electrons, in this case their charges are imbalanced and they become an ion
    \item \textbf{Polar} - Describes an object whose net charge is 0, but the charge is not distributed evenly
    \item Water is polar and can attract electrons that have "leaked off" of objects
\end{itemize}

\subsection{Insulators and Conductors}
\begin{itemize}
    \item \textbf{Conductor} - An object that can transfer electricity easily
    \item \textbf{Insulator} - An object that does not transfer electricity well
    \item \textbf{Semiconductor} - A material that falls between being a conductor ans insulator
    \item Good conductors have loosely bound electrons that can move freely within a material, such electrons are called \emph{free electrons} or conduction electrons
    \item Free electrons in a conductor can be attracted or repelled by a charge coming near the conductor
\end{itemize}

\subsection{Induced Charge; the Electroscope}
\begin{itemize}
    \item \textbf{Charging by COnduction} - THe process of causing a charge in an object by touching with a charged object
    \item \textbf{Charge by induction} - Bringing a charged object close to a neutral object will separate the charges inside and make the two sides have opposite charges
    \item \textbf{Grounded} - Describe an object connected to the Earth, which is used as a reservoir of electric charge
    \item \textbf{Electroscope} - A device used to measure electric charge, uses two small metal leaves to show the magnitude of an object's charge, if it has a high charge, the leaves separate further apart
    \item \textbf{Electrometers} - More precise, sensitive electroscopes used in actual measurements
\end{itemize}

\subsection{Coulomb's Law}
\begin{itemize}
    \item \textbf{Coulomb's Law} - Relates the force between two charges with the distance between them \[F=k\frac{Q_1Q_2}{r^2}\] Where \(Q_1\) is the first charge, \(Q_1\) is the second charge, r is the distance between the two charged, and k is a proportionality constant
    \item Only gives the magnitude, the direction of the force is along the line connecting the two charges and follows the rules of attraction between like or opposite charges
    \item \textbf{Coulomb} - The SI unit of charge
    \item k in Coulomb's Law is defined as \[k=\frac{8.988*10^9N*m^2}{C^2}\]
    \item 1C of charge placed on two objects 1.0M apart would produce a force of \(10^9\)N (a lot!) 
    \item \textbf{Elementary Charge} - Smallest observable charge, defined as \[e=1.6022*10^-19C\]
    \item Any electric charge can be given in terms of elementary charges, since the elementary charge the equivalent to the charge given by a single electron or proton
    \item Similar to the law of universal grvitatio, Coulomb's Law is an inverse square law \(F\propto 1/r^2\)
    \item \textbf{Permitivity of free space} - Another constant that can be used for Coulomb's Law, \[\epsilon=\frac{1}{4\pi k}\]
    \item Coulomb's law is best used when determining the force between two objects whose size is much smaller than the distance between them, if this is not the case then r is defined as the distance betwene their centers
    \item \textbf{Point Charges} - Charges whose spacial size is negigible
    \item \textbf{Electrostatics} - The study of electrical charges at rest, Coulomb's Law gives the electrostatic force
    \item \textbf{Principle of Superposition} - The net force on any one charge will be the vector sum of the forces due to each of the others
\end{itemize}

\subsection{Solving Problems Involving Coulomb's Law and Vectors}
\begin{itemize}
    \item The electrostatics force is a vector and follows the same vector math rules as any other
\end{itemize}

\textbf{Vector Addition Review} 
\begin{itemize}
    \item Vectors in multiple dimensions are added via their components, \[F_x=F\cos\theta\] \[F_y=F\sin\theta\]
    \item The directional components of each vector are added to get the component of the resultant vector \[F_x=F_1x+F_2x\] \[F_y=F_1y+F_2y\]
    \item The magnitude of the resultant vector is \[F=\sqrt{F_x^2+F_y^2}\]
    \item The direction of the resultant vector is \[\tan\theta=\frac{F_y}{F_x}\]
\end{itemize}

\textbf{Adding Vector FOrces; Principle of Superposition}
\begin{itemize}
    \item To determine the net force on an object with multiple charges acting upon it, 
    \begin{itemize}
        \item Use vector addition to determine the magnitude of the forces
        \item Determine the direction of each force graphically
        \item Combine the forces with their directions to determine the net force
    \end{itemize}
\end{itemize}

\subsection{The Electric Field}
\begin{itemize}
    \item An electric field, extends outward from every charge and permeates space
    \item \textbf{Electric Field} - E, at any point in space is defined as the force exerted on a tiny positive test charge placed at that point divided by the magnitude of the test charge \[E=\lim_{q\to0}\frac{F}{q}\]
    \item Electric field at a point is also equal to \[E=k\frac{Q}{r^2}\]
    \item The force on a charge is thus \[F=qE\]
    \item \textbf{Superposition Principle} - FOr an electric field, the electric field at a point is the sum of all the electric fields exrted at that poit
\end{itemize}

\subsection{Electric Field Lines}
\begin{itemize}
    \item \textbf{Electric FIeld Lies} - Also called lies of force, these are used to visualize the direction of the electric field at different points in space, start at positive charges and end at negative charges
    \item Lines are drawn so that the number of lines attached to a charge is proportional to the magnitude of the charge
    \item The more lines there are in a region, the stronger the electric filed is at that region
    \item \textbf{Electric Dipole} - A system that contains two charges of opposite charge but equal magnitude
    \item For a charge between two closely spaces, oppositely charged flat parallel plates, the electric field is constant, \[E=constant\]
    \item \textbf{Properties of Field lines}
\end{itemize}

\textbf{Gravitational Field}
\begin{itemize}
    \item \textbf{Gravitational FIeld} - Aplies to every object that has mass, defined as force per unit mass
    \item Fields can be tought of as areas of possibility, a force CAN be applied there but only if something that can be influenced by that force is present
\end{itemize}

\subsection{Electric Field and Conductors}
\begin{itemize}
    \item The electric field inside a conductor is zero in the static situation
    \item Any net charge on a conductor distributes itself on the surface
    \item The electric field is always perpendicular to the surface outside of a conductor
    \item These rules only apply to conductors, not nonconductors 
\end{itemize}

\subsection{Electric and Molecular Biology: DNA Structure and Replication}
\begin{itemize}
    \item Cellular processes are considered to be a result of random molecular motion plus the ordering effect of the electrostatic force
    \item The compounds inside of DNA are held together by electrostatic bonds since the molecules within them are polar
    \item \textbf{Hydrogen Bond} -  Weak bond formed between a hydrogen ion and a negatively charged ion
    \item DNA shapes are so specific that small changes can make the electrostatic bonds between molecules impossible to form
\end{itemize}

\subsection{Photocopy Machines and Computer Printers Use Electrostatics}
\begin{itemize}
    \item \textbf{Photoconductvitiy} - The propoerty of being nonconductive in the dark but conductive when exposed to light
    \item \textbf{Photocpiers} - Use the property of conductivity to replicate an image
    \item \textbf{Laser Printers} - Similar to photocopiers except they use a laser as their light source
    \item \textbf{Inkjet Printer} - Use tiny jets to spray small droplets of ink onto paper, so not use electric propoerties
\end{itemize}

\subsection{Gauss's Law}
\begin{itemize}
    \item \textbf{Electic Flux} - The electric field passing through a given area, defined as \[\phi_E=EA\cos\theta\] Where E is electric field, A is area, and \(\theta\) is the angle between the electric field direction and a line perpendicular to the area
    \item The number of electric field lines in an area is proportional to electric flux \[N\propto\phi\] 
    \item The total flux in a given area is defined as \[\phi_e=\Sigma E_{\bot}\Delta A\] Where \(E_{\bot}\) is equal to \(E\cos\theta\)
    \item \textbf{Coulomb's Law} - The total flux in an enclosed area is proportional to the net charge encolsed by the surface, \[\Sigma E_{\bot}\Delta\frac{Q_{encl}}{\epsilon_0}\] Where \(Q_{encl}\) is the net charge enclosed by a surface and \(\epsilon_0\) is the constant of proportionality
    \item Normally, surfaces with symmetry are chosen
    \item The electric field between two equally spaced, oppositely charged, parallel plates is \[E=\frac{Q}{\epsilon_0 A}\]
\end{itemize}

\newpage