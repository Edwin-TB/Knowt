\section{Nuclear Physics and Radioactivity}
Rutherford's experiments hinted at the center of atoms being a tiny, massive, positively charged nucleus. Later, quantum, theory was developed and more was being learned about the nucleus, this chapter focuses on nuclear physics

\subsection{Structure and Properties of the Nucleus}
\begin{itemize}
    \item \textbf{Proton} - Has a mass of \[m_p=1.67262*10^{-27}kg\]
    \item \textbf{Neutron} - Has a mass of \[m_p=1.67493*10^{-27}kg\]
    \item Protons and neutrons are referred to as \emph{nucleons}
    \item \textbf{Nuclides} - Used to refer to different nuclei
    \item \textbf{Atomic Number} - The number of protons in a nucleus, denoted by Z
    \item \textbf{Atomic Mass Number} - The number of protons plus the number of neutrons in an atom, denoted by A
    \item Nuclides can be defined by \[^A_ZX\] Where X is the chemical symbol for the element
    \item \textbf{Isotope} - Nuclei that share the same atomic number but different mass numbers
    \item \textbf{Natural Abundance} - The percentage of a naturally occurring isotope of a certain element
    \item Nuclei have a roughly spherical shape, the radius of which can be approximated by \[r\approx (1.2*10^{-15}m)(A^{\frac{1}{3}})\]
    \item \textbf{Unified Atomic Mass Units} - A unit of mass used to specify the masses of nuclei, defined as \[1u=1.66054*10^{-27}kg\]
    \item Converting this to energy gives, \[1u=931.5MeV/c^2\]
    \item \textbf{Nuclear Spin} - A quantum number of nuclei that can be either an integer or a half integer
\end{itemize}

\subsection{Binding Energy and Nuclear Forces}
\textbf{Binding Energy}
\begin{itemize}
    \item The mass of a nucleus is always less than the sum of the masses of its individual nucleons
    \item \textbf{Total Binding Energy} - The difference in mass between a nuclei's true mass and the sum of its nucleons, represents the energy that must be put into a nucleus to separate it into its components
    \item Binding energy is something a nucleus lacks
    \item \textbf{Binding Energy per Nucleon} - Defined as the total binding energy divided by A
\end{itemize}

\textbf{Nuclear Forces}
\begin{itemize}
    \item \textbf{Strong Nuclear Force} - The force that holds nucleons together an opposes the electrostatic repulsion between protons
    \item The strong nuclear force is a \emph{short range} force, it acts over a very short distance
    \item \textbf{Weak Nuclear Force} - A weaker type of nuclear force that is only known about because of its presence in certain types of nuclear decay
    \item The four fundamental forces are the strong and weak nuclear forces, the electromagnetic force, and gravity
\end{itemize}

\subsection{Radioactivity}
\begin{itemize}
    \item \textbf{Radioactivity} - The phenomenon of certain materials producing radiation that are not x-rays or weaker
    \item Radioactivity is the result of the \emph{decay} of an unstable nucleus
    \item \textbf{Radioisotope} - A radioactive isotope of a certain element
    \item The tree types of radiation are alpha, beta, and gamma, each have a different charge
\end{itemize}

\subsection{Alpha Decay}
\begin{itemize}
    \item An alpha particle is defined as (\(^4_2He\)) or a helium nucleus
    \item When alpha (\(\alpha\) decay occurs, a new element is formed and its nucleus is referred to as the daughter nucleus while the original is called the parent
    \item \textbf{Transmutation} - The conversion from one element to another
    \item Alpha decay occurs when the strong nuclear force is no longer to hold large nuclei together due to electrostatic repulsion
    \item \textbf{Disintegration Energy} - Total energy released when a alpha decay occurs
    \item Alpha particles are held very strongly together, so their mass is less than the sum of their nucleons, so it is easier for them to separate from a parent nucleus
    \item \textbf{Smoke Detectors} - Devices that use alpha decay from Americium to ionize nitrogen and oxygen in the surrounding air, allowing current to flow, when smoke is present the smoke particles absorb the radiation instead and stop the current flow, setting off the alarm
\end{itemize}

\subsection{Beta Decay}
\textbf{\(\beta^-\) Decay}
\begin{itemize}
    \item A \(\beta^-\) particle is an electron
    \item During beta decay, a neutrino is also released, which is a particle with a very small mass and no charge
    \item Because the electron is emitted from the nucleus itself, its charge is +1e more than it was previously
    \item What really happens is one of the neutrons within the nucleus turns into a proton and releases an electron, thus \[n=p+e+neutrino\]
    \item The kinetic energy of a beta particle can range between 0 and 156keV
    \item The neutrino was discovered in an effort to determine why the kinetic energy was not the same for all beta particles
    \item The weak nuclear force is only present during beta decay, as the neutrino only interacts with matter through the weak nuclear force
\end{itemize}

\textbf{\(\beta^+\) Decay}
\begin{itemize}
    \item Some isotopes have too many protons and too few neutrons, these decay by emitting a positron instead of an electron
    \item A positron has the same mass an electron but has a positive charge, it is also called the antiparticle to the electron
\end{itemize}

\textbf{Electron Capture}
\begin{itemize}
    \item \textbf{Electron Capture} - Occurs when a nucleus absorbs one of its orbiting electrons
    \item A proton in the nucleus then becomes a neutron and the nucleus moves down an atomic number
\end{itemize}

\subsection{Gamma Decay}
\begin{itemize}
    \item \textbf{Gamma Ray} - An extremely high-energy photon, come from decaying nuclei
    \item Nuclei can have an excited state just like an atom, when they jump from a higher to lower energy state, they release a photon called a \(\gamma\) ray
    \item The energy states of nuclei vary on the order of keV to MeV, thus the photons they emit can have energies of a few MeV
    \item There is no change in an element as a result of gamma decay
    \item Nuclei are denoted with an asterisk to represent an excited state
    \item Sometimes, excited nuclei can remain that way for a long time, they are said to be in a \emph{metastable} state and are called isomers
    \item \textbf{Internal Conversion} - The process through which an excited nucleus returns to its ground state not by releasing a gamma ray, but by releasing an orbital with the same kinetic energy a gamma ray would
\end{itemize}

\subsection{Conservation of Nucleon Number and Other Conservative Laws}
\begin{itemize}
    \item In all three types of decay, conversion laws hold
    \item \textbf{The Law of Conversion of Nucleon Number} - States that the total number of nucleons, remains constant in any process
\end{itemize}

\subsection{Half-Life and Rate of Decay}
\begin{itemize}
    \item A sample of a radioactive isotope does not release all of its radiation at the same time, but the nuclei do so one by one randomly
    \item The number of decays that occur over a time interval is proportional to the time interval and the number of radioactive nuclei present, represented by \[\Delta N=-\lambda N\Delta T\] or \[\frac{\Delta N}{\Delta t}=-\lambda N\] Where \(\Delta N\) is the number of decays, \(\Delta t\) is the time interval, the - means N is decreasing, N is the number of radioactive nuclei in the sample, and \(\lambda\) is the \emph{decay constant}
    \item \textbf{Decay Constant} - A value that varies for different isotopes
\end{itemize}

\textbf{Exponential Decay}
\begin{itemize}
    \item \textbf{Radioactive Decay Law} - Solving for N gives \[N=N_0e^{-kt}\] Where \(N_0\) is the original amount of radioactive nuclei, e is the natural exponential, k is a constant, and t is time passed thus the number of radioactive nuclei decreases exponentially with time.
    \item The decay rate of a substance is also called its \emph{activity}
    \item\textbf{Radioactive Decay Law} - Used to calculate decay rate of radioactivity, given by \[R=|\frac{\Delta N}{\Delta t}|=R_0e^{-kt}\]
\end{itemize}

\textbf{Half-Life}
\begin{itemize}
    \item \textbf{Half-Life} - Defined as the amount time it takes half the original amount of parent isotope to decay
\end{itemize}

\subsection{Calculations Involving Decay Rates and Half-Life}
\begin{itemize}
    \item Problems involving half-life can be easily reasoned through using dimensional analysis
    \item Another important prerequisite to approaching such problems is knowledge of logarithms and what they can physically represent
\end{itemize}

\subsection{Decay Series}
\begin{itemize}
    \item \textbf{Decay Series} - A chain of successive radioactive decays where daughter isotopes lead to other daughter isotopes
    \item Decay series cause us to observe isotopes with short half-lives occur naturally as they are replenished by the decay of isotopes higher in the series
\end{itemize}

\subsection{Radioactive Dating}
\begin{itemize}
    \item \textbf{Radioactive Dating} - A technique used to determine the age of a sample by comparing the presence of radioactive isotopes before and after
    \item THe age of living matter can be determined by measuring the presence of carbon-14 since its presence in the atmosphere is relatively constant
    \item A similar technique is used to measure the age of rocks by measuring the presence of uranium-238
\end{itemize}

\subsection{Stability and Tunneling}
\begin{itemize}
    \item A question that arises is why don't all radioactive nuclei immediately decay
    \item This is because decay particles must surpass the \emph{coulomb barrier}, which it does so via \emph{quantum mechanical tunneling}
    \item This occurs when, for a brief period of time, the particle violates the conservation of energy and is able to surpass the barrier, giving it enough time to "tunnel" through the barrier
\end{itemize}

\subsection{Detection of Particles}
\textbf{Counters}
\begin{itemize}
    \item \textbf{Geiger Counter} - A device consisting of a metal tube filled with a gas and a wire kept at a high positive voltage
    \item When a charged particle enters the tube, it ionizes a few gas particles whose electrons move to the wire and ionize other gas particles, repeating the process
    \item The avalanche of electrons then creates a voltage pulse that the counter receives 
    \item \textbf{Scintillation Counter} - Uses a scintillator or phosphor whose particles are easily excited and emit light when they return to ground state, they detect the light released
    \item \textbf{Photomultiplier Tube} - Converts the light emitted by a scintillator to an electric signal
    \item \textbf{Semiconductor Detector} - Uses a reverse-biased pn junction diode to detect a charged particle when it excites electrons into the conduction band
\end{itemize}

\textbf{Visualization}
\begin{itemize}
    \item \textbf{Silicon Wafer Semiconductors} - Used to visualize charged particles, have pixels etched onto their surfaces that each give particle position information
    \item \textbf{Cloud Chamber} - Works bu super cooling a gas, tiny droplets form around charged particles and the reflection of light is used to detect the movement of particles
    \item \textbf{Bubble Chamber} - Similar to a cloud chamber, super heats a liquid and bubble of vapor form around charged particles
    \item \textbf{Multi wire Chamber} - Consists of many wires in a gas, some of which are charged. Charged particles excite molecules in the gas and ionize them, causing them to be attracted by the charged wires and sending a pulse once they come in contact
\end{itemize}

\newpage