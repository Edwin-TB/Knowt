\section{Temperature and Kinetic Energy}
This section focuses on atoms and their continuous random motion

\subsection{Atomic Theory of Matter}
\begin{itemize}
    \item \textbf{Atom} - Means indivisible, this is the smallest piece of matter
    \item \textbf{Atomic and Molecular Mass} - The mass of an atom or molecule in unified atomic mass units (u), defined as \[1u=1.6605*10^-27kg\]
    \item \textbf{Element} - A substance that cannot be broken down into smaller parts
    \item \textbf{Compound} - Substances made of elements
    \item \textbf{Molecule} - The smallest piece of a compound
    \item \textbf{Brownian Motion} - Describes the erratic, random motion of small particles, later used by Einstein to determine the average size of an atom to be around \(10^-10\)m
    \item \textbf{Macroscopic} - Meaning large scale
    \item \textbf{Microscopic} - Meaning small scale
    \item Atoms and molecules naturally repel each other, leading to constant motion as they try to push away from each other
\end{itemize}

\subsection{Temperature and Thermometers}
\begin{itemize}
    \item \textbf{Temperature} - A measure of how hot or cole something is
    \item Objects tend to expand when they are at a higher temperature
    \item Some objects emit light at extreme temperatures
    \item Many thermometers depend on the expansion caused by heat to measure temperature
    \item \textbf{Bimetallic Strip} - A thermometer made by binding two different metals with different rates of expansion, as the temperature increases, the different expansions cause the strip to bend
\end{itemize}

\textbf{temperature Scales}
\begin{itemize}
    \item Most temperature scales use readily reproducible temperatures as points of reference
    \item To convert from Celsius to Fahrenheit
    \[T(^\circ C)=\frac{5}{9}[T(^\circ F)-32]\]
    \item To convert from Fahrenheit to Celsius \[T(^\circ F)=\frac{9}{5}T(^\circ F)+32\]
\end{itemize}

\textbf{Standard Temperature Scale}
\begin{itemize}
    \item Materials usually do not expand at the same rate throughout a wide range of temperaturers
    \item To improve precision, a \emph{constant-volume gas thermometer} is used which expands a gas at the same rate
    \item The thermometer mentioned above is used as the basis for the \emph{standard temperature scale}
\end{itemize}


\subsection{Thermal Equilibrium and the Zeroth Law of Thermodynamics}
\begin{itemize}
    \item Two objects of different temperatures placed in thermal contact will eventually reach the same temperature or \emph{thermal equilibrium}
    \item \textbf{Thermal Contact} - A connection between two objects that allows thermal energy to transfer between them
\end{itemize}

\textbf{The Zeroth Law of Thermodynamics}
\begin{itemize}
    \item \textbf{Zeroth Law of Thermodynamics} - If two systems are in thermal equilibrium with a third system, then they are in thermal equilibium with each other
    \item When systems are in thermal equilibrium with each other there is no net thermal energy exchanged
\end{itemize}

\subsection{Thermal Expansion}
\begin{itemize}
    \item The rate of thermal expansion of an ibject is dependent on its material
\end{itemize}

\textbf{Linear Expansion}
\begin{itemize}
    \item Thermal expansion is proportional to the original length of the object, represented by \[\Delta l=\alpha l_0\Delta T\] Where \(l_0\) is the original length, \(\alpha\) is the coefficient of linear expansion, and \(\Delta T\) is change in temperature
    \item This can be rewritten as \[l=l_0(1+\alpha\Delta T)\]
\end{itemize}

\textbf{Volume Expansion}
\begin{itemize}
    \item Objects also expand in volume when they gain heat energy, represetned by \[\Delta V=\beta V_0\Delta T\] Where \(V_0\) is the original volume, \(\beta is the coefficient of volume expansion\), and \(\Delta T\) is the change in temperature
    \item The equations fof change in length and volume due to thermal expansion are only accurate for small changes and \(\beta\) varies significantly with temperature
\end{itemize}

\textbf{Anomalous Behavior of Water Below
\(4^\circ\)}
\begin{itemize}
    \item Water has its greatest density at \(4^\circ\)C 
    \item Because of this, ice is less dense than water and bodies of water at freezing temperatures freeze on the surface, leaving the rest at around \(4^\circ\)C
\end{itemize}

\textbf{Thermal Stresses}
\begin{itemize}
    \item \textbf{Thermal Stresses} - Tensile or compressive stresses within a material due to thermal expansion
    \item The thermal stresses within an object is calculate by \[\frac{F}{A}=\alpha E\Delta T\] Where A is the cross sectional area of the object, \(\alpha\) is the coefficient of linear expansion, E is the young's modulus, and \(\Delta\)T is the change in temperature
\end{itemize}

\subsection{The Gas Laws and Absolute Temperature}
\begin{itemize}
    \item \textbf{Equation of State} - A relation between the volume pressure, temperature, and quantity of a gas
    \item \textbf{Equilibrium State} - A state of a system where the variables that describe is do not change in time
    \item \textbf{Boyle's Law} - Typically, the volume of a gas is inversely proportional to its pressure at a constant temperature, \[V\propto \frac{1}{P}\] Therefore \[PV=constant\]
    \item \textbf{Absolute Zero} - The lowest temperature possible, the theoretical temperature where a gas would have no volume
    \textbf{Absolute Scale} - ALso called the \emph{Kelvin Scale}, this is a temperature scale with the same distance between temperatures as Celsius but 0 is defined as \(-273.15.15^\circ\)C
    \item TO convert from Celsius to Kelvin, \[T(K)=T(^\circ C)+273.15\]
    \item \textbf{Charles's Law} - The volume of a fixed quantity of a gas is directly proportional to the absolute temperature when the pressure is kept constant, \[V\propto T\]
    \item \textbf{Gay-Lussac's Law} - At constant volume, the absolute pressure of a fixed quantity of a gas is directly proportional to the absolute temperature \[P\propto T\]
\end{itemize}

\subsection{The Ideal Gas Law}
\begin{itemize}
    \item \textbf{Ideal Gas Law} - Combines Boyle's, Charles's, and Gay-Lusac's Laws, note that changing two of the variables will change the other \[PV\propto T\]
    \item Including the mass of a gas gives \[PV\propto mT\]
    \item \textbf{Mole} - SI unit for the amount of a substance, defined as \(6.02*10^23\) objects of something, used to determine how much a substance there is
    \item Number of moles of a substance is calculated by \[n(mole)=\frac{mass(grams)}{molecular\ mass(g/mol)}\]
    \item The proportion can be rewritten as \[PV=nRT\] Where R is the \emph{universal gas constant}, defined as \[R=1.99calories(mol*K)\]
    \item \textbf{Ideal Gas Law} - Real gases do not follow it precisely, but close enough at low pressures \[PV=nRT\]
\end{itemize}

\subsection{Problem Solving with The Ideal Gas Law}
\begin{itemize}
    \item \textbf{Standard Conditions} - Also called \emph{standard temperature and pressure} (STP), defines the temperature and pressure for most problems, \[T=273K\ (0^\circ)C\] and \[P=1.00atm=101.3kPa\]
    \item Problems that involve a change in pressure, volume, or temperature but no change in the volume of gas can be solved with \[\frac{P_1V_1}{V_1}=\frac{P_2V_2}{V_2}\]
\end{itemize}

\subsection{Ideal Gas Law in Terms of Molecules; Avogadro's Number}
\begin{itemize}
    \item \textbf{Avogadro's Hypothesis} - Equal volumes of gas at the same pressure and temperature contain equal numbers of molecules, consistent with R being the same for all gases
    \item \textbf{Avogadro's Number} - \(N_A=6.02*10^{23}\)
    \item The ideal gas law can be rewritten in terms of numbers of molecules as \[PV=NkT\] Where N is the total number of molecules present and k is the \emph{Boltzmann Constant}, defined as \[k=\frac{R}{NN_A}=1.38*10^{-23}J/K\]
\end{itemize}

\subsection{Kinetic Theory and the Molecular Interpretation of Temperature}
\begin{itemize}
    \item \textbf{Kinetic Theory} - The study of matter in terms of atoms in continuous random motion
	\item Kinetic Theory uses statistical approached to predict the behavior of atoms on macroscopic scales
	\item Postulates of the Ideal Gas
	\begin{itemize}
		\item Our assumption is that gases fill the volume of their container, in the atmosphere's case, it is kept from escaping by Earth's gravity
		\item Molecules are on average far apart from each other, the distance between them is greater than the diameter of each molecule
		\item Molecules are assumed to obey classical mechanics and collide with each other accordingly, molecules exert weak attractive force on each other, the potential energy caused by this is small compared to the kinetic energy
		\item Collisions are assumed to be perfectly elastic, also the time of a collision is very short compared to the time between collisions, thus potential energy is ignored
	\end{itemize}
	\item The pressure on the vessel of a gas is caused by the collisions of the molecules on its walls
	\item The average force exerted by one molecule of a gas is \[F=\frac{mv_x^2}{l}\] Where m is the mass, \(v_x\) is the x component of its velocity, and l is the length is the vessel
	\item The pressure of a gas on the wall of its vessel is \[P=\frac{1}{3}\frac{NMv^2}{V}\] 
	\item The kinetic energy of objects in motions in an ideal gas is directly proportional to the absolute temperature of the gas, \[KE=\frac{1}{2}m\overline{v^2}=\frac{3}{2}kT\]
\end{itemize}

\textbf{Kinetic Energy Near Absolute Zero}
\begin{itemize}
    \item As temperatures approach absolute zero, molecular motion does not approach zero, instead it approaches a very small non-zero value
    \item Molecular motion does not cease even at absolute zero
\end{itemize}

\subsection{Distribution of Molecular Speeds}
\begin{itemize}
    \item \textbf{Maxwell Distribution of Speeds} - A graph that shows the relative number of molecules at a certain speed
    \item The Maxwell Distribution for the same gas changes depending on its temperature
    \item At higher temperatures, the proportion of molecules with high speed increases
\end{itemize}

\subsection{Real gases and Changes of Phase}
\begin{itemize}
    \item When temperatures and pressures of gases are high, the ideal gas law begins to break down
    \item \textbf{PV Diagram} - Relates the Pressure and volume of a gas at a constant temperature
    \item At lower temperatures, it is possible for a gas to liquefy at certain volumes or pressures, shown as the liquid-vapor region
    \item \textbf{Critical Temperature} - The temperature at which the PV diagram of a gas has a horizontal curve that touched the liquid-vapor region
    \item \textbf{Critical Point} - The point at which the critical temperature touches the liquid-vapor region
    \item \textbf{PT Diagram} - also called a \emph{phase diagram}, shows the phase of a substance at temperature and pressure combinations
    \item \textbf{Sublimation} - The process by which a material converts from solid directly to gas
    \item \textbf{Triple Point} - The point at which the three curves on a PT Diagram intersect, all three phases can exist in equilibrium
    \item \textbf{Superfluidity} - Describes a liquid that has no viscosity
    \item \textbf{Liquid Crystals} - Can be considered a phase between liquid and solid
\end{itemize}

\subsection{Vapor Pressure and Humidity}

\textbf{Evaporation}
\begin{itemize}
    \item \textbf{Evaporation} - The process by which a liquid converts to gas
    \item Caused when the fastest molecules in a liquid escape its surface
    \item Evaporation is a cooling process
\end{itemize}

\textbf{Vapor Pressure}
\begin{itemize}
    \item \textbf{Condensation} - The process of vapor reentering its liquid
    \item \textbf{Saturated Vapor Pressure} - The pressure of a vapor when it is saturated
    \item \textbf{Saturation} - When the amount of vapor in the surrounding air has reached the maximum the air can hold and the number of molecules exiting and entering is equal
\end{itemize}

\textbf{Boiling}
\begin{itemize}
    \item A liquid boils when its saturated vapor pressure equals the external pressure
    \item Saturated vapor pressure increases with temperature
    \item High elevations cause boiling points to lower as the surrounding air presure is lower and thus the saturated vapor pressure does not have to reach as high
\end{itemize}

\textbf{Partial Pressure and Humidity}
\begin{itemize}
    \item \textbf{Partial Pressure} - The pressure exrted by each individual gas in air if it were alone
    \item \textbf{Relative Humidity} - THe ration between the partial pressure of water and its saturated vapor pressure in an environment, \[Relative\ Humidity=\frac{partial\ pressure\ of\ H_2O}{Saturated\ vapor\ pressure\ of\ H_2O}*100\%\]
    \item \textbf{Supersaturaed} - Describes when the partial pressure of water in air exceeds is saturated vapor pressure, can occur when a temperature decrease occurs, when this happens water can condense back into clouds or dew
    \item \textbf{Dew Point} - The temperature at which the partial pressure of water exceeds its saturated vapor pressure for a given humidity
\end{itemize}

\subsection{Diffusion}
\begin{itemize}
    \item \textbf{Diffusion} - The natural mixing of fluids until they become unifmr
    \item Occurs due to the random motion of molecules explained by kinetic theory
    \item Diffusing substances move from a region where its concentration is high to a region where its concentration is low
    \item \textbf{Fick's Law} - The rate of diffusion for a substance is given by \[J=DA\frac{C_1-C_2}{\Delta x}\] Where D is the diffusion constant of the fluid, A is the area of the cross section of the vessel, \(C_1\) is the concentration in the area of high concentration, \(C_2\) is the concentration in the area of low concentration, and \(\Delta x\) is the distance between the two regions
    \item Diffusion is essential for many biological processes as it allows fluids to be exchanged without the organism itself expending energy
\end{itemize}

\newpage