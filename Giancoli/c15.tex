\section{The Laws of Thermodynamics}
\textbf{Thermodynamics} is the study of processes in which energy is transferred as heat and as work. 

\subsection{The First Law of Thermodynamics}
\begin{itemize}
    \item \textbf{The First Law of Thermodynamics} - The change in internal energy of a closed system, \(\Delta U\), will be equal to the energy added to the system by heating minus the work done by the system on the surroundings, \[\Delta U=Q-W\]
    \item This is a general statement of the law of conservation of energy
    \item \textbf{State Variables} - Variables that describe the state of a system, these are energy U, pressure P, volume V, temperature T, and mass M or number of moles n
\end{itemize}

\textbf{The First Law of Thermodynamics Extended}
\begin{itemize}
    \item In a system with kinetic and potential energy, the first law would be rewritten  as \[\Delta KE+\Delta PU+\Delta U=Q-W\]
\end{itemize}

\subsection{Thermodynamic Processes and the First Law}
\textbf{Isothermic Processes}
\begin{itemize}
    \item \textbf{Isothermic Process} - A process that occurs at a constant temperature
    \item An isothermic ideal gas system would follow \[PV=constant\] The curves describing this state are called isotherms
    \item We assume the gas is in contact with a \emph{heat resevoir} and that changes in volume occur very slowly
    \item Internal energy does not change, thus \[\Delta U=\frac{3}{2}nR\Delta T=0\]
\end{itemize}

\textbf{Adiabatic Processes}
\begin{itemize}
    \item \textbf{Adiabatic Process} - A process where heat does not flow in or out of a system, or it is so well insulated that heat does not get the chance to flow out, Q=0
    \item Adiabatic systems have PV curves similar to those of isotherms, but not the same as heat is still allowed to flow within the system and thus changes in pressure and volume change the temperature
\end{itemize}

\textbf{Osibaric and Isovolumetric Processes} 
\begin{itemize}
    \item \textbf{Isobaric} -  A process where pressure is kept constant
    \item \textbf{Isovolumetric} - A process where volume is kept constant
\end{itemize}

\textbf{Work done in Volume Changes}
\begin{itemize}
    \item The work done in a volume change is given by \[W=P\Delta V\]
    \item Only the volume change in an isotherm is used to determine work done
\end{itemize}


\subsection{Human Metabolism and the First Law}
\begin{itemize}
    \item \textbf{Metabolism} - The energy changing processes in an organism
    \item The human body does work all the time, it gets energy for this by eating food and lets it out by releasing it into the surrounding area
\end{itemize}

\subsection{The Second Law of Thermodynamics - Introduction}
\begin{itemize}
    \item \textbf{Second Law of Thermodynamics} - Heat can flow spontaneously from a hot object to a cold object; heat wil not flow spontaneously from a cold object to a hot object
    \item The above statement is specific to certain processes but implies that many processes are not reversible
\end{itemize}


\subsection{Heat Engines}
\begin{itemize}
    \item \textbf{Heat Engine} - a device that changes thermal energy to mechanical work
    \item Heat engines that cycle have an internal energy change of 0 since they alwasy return to their starting point
    \item \textbf{Operating Temeratures} - The high temperature input and lower temperature output of a heat enginer
\end{itemize}

\textbf{Steam Engine and Internal Combustion Engine}
\begin{itemize}
    \item \item These engines use thermal energy to heat a substance to produce work
    \item \textbf{Working Substance} - The work heated to work, usually either steam or fuels such as gasoline
\end{itemize}

\textbf{Why \(\Delta T\) is Needed to Drive a Heat Engine}
\begin{itemize}
    \item Heat engines need some form of energy loss from intake to exhaust in order to use that energy loss as work
    \item If there is no change in temperature, pressure remains the same on both sides and thus work cannot be done
\end{itemize}

\textbf{Efficiency}
\begin{itemize}
    \item \textbf{Efficiency} - The ratio of the work a heat engine does and the input at the high temperature, \[e=\frac{W}{Q_H}\]
\end{itemize}

\textbf{Carnot Engine}
\begin{itemize}
    \item \textbf{Carnot Engine} - Named after French scientist Sadi Carnot, this is the ideal engineer but it does actually exist
    \item Carnot Engines do work reversibly, that is they can be done in reverse with no change in work done or heat exchanged
    \item Real life processes have many factors that interrupt reversibility and are thus \emph{irreversible}
    \item \textbf{The Third Law of Thermodynamics} - No device is possoble whose sole effects is to transform a given amount of heat completely into work
\end{itemize}

\subsection{Refrigerators, Air Conditioners, and Heat Pumps}
\begin{itemize}
    \item \textbf{Clausius statement of the second law of thermodynamics} - No device is possible whose sole effect is to transfer heat from one system at a temperature \(T_L\) into a system at a higher temperature \(T_H\)
    \item All refrigerators exert some work to move heat outside of themselves, in most cases it is a motor
    \item \textbf{Coefficient of Performance} - The heat removed from the lower temperature area divided by the work done to do so \[COP=\frac{Q_L}{W}\] Where \(Q_L\) is the heat removed from the inside of the refrigerator
    \item The COP for a heat pump is \[COP=\frac{Q_H}{W}\] Where \(Q_L\) is the heat delivered to the inside of the house
\end{itemize}

\textbf{SEER Rating}
\begin{itemize}
    \item SEER stands for seasonal energy efficiency ratio, it is similar to the COP of a device, defined as \[SEER=\frac{heat\ removed\ in\ BTU}{electrical\ input\ in \ watt-hours}\]
\end{itemize}

\subsection{Entropy and the Second Law of Thermodynamics}
\begin{itemize}
    \item \textbf{Entropy} - A function of the state of a system, it goes along with temperature, volume, pressure, and mass
    \item Change in entropy is \[\Delta S=\frac{Q}{T}\] Where Q is the heat added and T is a constant temperature in a system in kelvin 
    \item Since processes are not reversible, Q in the equation cannot be negative and thus change in entropy is always positive
    \item The second law of thermodynamics in terms of entropy is: The entropy of an isolated system never decreases. It can only stay the same or increase
    \item \textbf{General Statement of the Second Law of Thermodynamics} - The total entropy of any system plus that of its environment increases as a result of many natural processes
\end{itemize}

\subsection{Order to Disorder}
\begin{itemize}
    \item Entropy can be considered the measure of disorder of the system, thus the second law of thermodynamics can be expressed as: natural processes tend to move to a state of great disorder
    \item An example of order to disorder is dropping a rock, some of its kinetic falling energy is converted to thermal energy which adds to the random disorderly movement of molecules
\end{itemize}

\textbf{Biological Development}
\begin{itemize}
    \item As organisms develop over time, they seem to become more orderly
    \item This does not violate entropy as waste molecules without order are produced by their metabolism as they grow
\end{itemize}

\textbf{Time's Arrow}
\begin{itemize}
    \item The second law also tells us which direction processes go, you can tell a movie is played backwards when you see it
    \item A decrease in entropy indicates a process occurred backwards in time, thus entropy is called \emph{time's arrow}
\end{itemize}

\subsection{Unavailability of Energy; Heat Death}
\begin{itemize}
    \item Another aspect of the second law: in any natural process, some energy becomes unavailable to do useful work
    \item As time goes on, energy is degraded and eventually converts to less useful forms such as thermal or internal enegry
    \item \textbf{Heat Death} - The prediction that all energy in the universe will, at some point, convert to heat energy
\end{itemize}

\subsection{Statistical Interpretation of Energy and the Second Law}
\begin{itemize}
    \item \textbf{Microstate} - Specifies the position and velocity of every particle in a system
    \item \textbf{Macrostate} - Gives visible scale states of the system, such as temperature and pressure
    \item It is assumed that statistically, every microstate is equally probable
    \item The second law of thermodynamics implies that \emph{those processes occur which are most probable}
    \item The second law does not \emph{forbid} entropy decreasing, but makes it extremely unlikely
    \item On the macroscopic scale, there are so many molecules that deviating from what is expected is extremely unlikely
\end{itemize}

\subsection{Thermal Pollution, Global Warming, and the Energy Resources}
\begin{itemize}
    \item \textbf{Thermal Pollution} - The thermal energy output by every heat engine, must be absorbed by the environment, which alters the Earth's ecology
    \item \textbf{Air Pollution} - Chemicals put into the air by burning fuels, mainly in the form of Carbon Dioxide, CO\(_2\)
    \item Carbon dioxide in the atmosphere absorbs some of the Earth's infrared radiation and keeps it from escaping, causing \emph{global warming}
    \item \textbf{Carbon Footprint} - Refers to the negative impact of an activity by how much carbon dioxide it releases
\end{itemize}

\newpage