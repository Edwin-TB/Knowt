\section{Elementary Particles}
After the discovery of protons and neutrons, more particles began being discovered, each with unique properties. Fundamental particles are defined as particles so simple they have no internal structure, the fundamental constituents of matter are quarks and leptons. The theory that describes out present view is the standard model.

\subsection{High-Energy Particles and Accelerators}
\begin{itemize}
    \item \textbf{High-Energy Accelerator} - Devices used to accelerate particles to high energies for experimental purposes
\end{itemize}

\textbf{Wavelength and Resolution}
\begin{itemize}
    \item De Broglie's wavelength formula, \[\lambda=\frac{h}{p}\] Says that as momentum increases, the wavelength of a particle decreases
    \item Thus, using a higher energy accelerator can lead to more detail as the shorter wavelength allows more to be seen
\end{itemize}

\textbf{Cyclotron}
\begin{itemize}
    \item A device that uses a magnetic field to maintain charged ions in a nearly circular path, still used for treating cancer today
    \item The frequency, f, of the applied voltage must be the same as the circulating ions, the force, F, on each ion is given by \[F=qvB\] thus \[qvB=\frac{mv^2}{r}\]
    \item \textbf{Synchrocyclotron} - A device used in a cyclotron that reduces frequency in time to correspond to the increase in \(\gamma m\)
\end{itemize}

\textbf{Synchrotron}
\begin{itemize}
    \item A device that accelerates particles by increasing the magnetic field, \(B\), particles move in a circle
    \item The largest accelerators in the world are synchrotrons and use rows of magnets in order to carry particles
    \item \textbf{Synchrotron Radiation} - Radiation emitted by the EM fields generated by a synchrotron 
\end{itemize}

\textbf{Linear Accelerator}
\begin{itemize}
    \item \textbf{Linear Accelerator} - Particles are accelerated along a straight path passing through tubular conductors, voltage within the tubes alternates to attract particles between the gaps and allow them to flow more
\end{itemize}

\textbf{Colliding Beams}
\begin{itemize}
    \item \textbf{Colliding Beams} - Two beams of particles are accelerated and made to collide head on, accomplished in a single accelerator through the use of storage rings
\end{itemize}

\subsection{Beginning of Elementary Particle Physics - Particle
Exchange}
\begin{itemize}
    \item A force can be thought of as a field OR an exchange of particles between two objects
    \item For the electromagnetic force, photons are exchanged which gives rise to the force between two objects
    \item The photon is emitted by one charged particle and absorbed by another, so it cannot be seen and is referred to as a \emph{virtual} photon, and is said to carry the EM force
    \item Another particle serves a similar purpose but for the strong nuclear force, called the \emph{meson}
    \item Mesons carry the strong nuclear force between protons and neutrons
    \item The particles that carry the weak force are referre to as \(W^+,\ W^-,\ Z^0\)
    \item The particle that carries the gravitational force is called the graviton, but it has not yet been detected
\end{itemize}

\subsection{Particles and Antiparticles}
\begin{itemize}
    \item Antiparticles are very similar to regular particles but have some properties reversed
    \item \textbf{Antiprotons} - Have the same mass as protons but has a negative charge
    \item All particles have an antiparticle
    \item \textbf{Antimatter} - Refers to matter composed of antiparticles
\end{itemize}

\textbf{Negative Sea of Electrons}
\begin{itemize}
    \item It was discovered that that equation \[E=\pm\sqrt{p^2c^2+m^2c^4}\] Has negative solutions
    \item This implies there are negative energy states, which were thought to be filled with negative of electrons
    \item If a photon hit one of these negative electrons, it would go up an energy level and leave a hole, a positron
    \item The uncertainty principle allows a particle to jump briefly to a normal energy level, thus creating a particle-antiparticle pair
\end{itemize}

\subsection{Particle Interactions and Conservation Laws}
\begin{itemize}
    \item \textbf{Baryon Number} - A generalization of nucleon number, nucleons have a baryon number of 1 and antinucleons have a baryon number of -1
    \item Some reactions do not occur due to the law of conservation of baryon number
    \item Sometimes, beta decay emits a muon instead of an electron, which is more massive and has its own neutrino
    \item There also exists the conservation of electron lepton number, where electrons and their neutrinos are assigned \(L_e=+1\) and their antiparticles \(L_e=-1\)
    \item \textbf{Muon Lepton Number} - Muons and muon neutrinos have a muon lepton number of \(L_{\mu}=+1\) and their antiparticles have a number of \(L_{\mu}=-1\)
    \item \textbf{Tau Lepton Number} - Tau leptons and tau neutrinos have a tau lepton number of \(L_{\tau}=+1\) and their antiparticles have a number of \(L_{\tau}=-1\)
    \item All the lepton numbers must be conserved in a reaction
\end{itemize}

\subsection{Neutrinos}
\begin{itemize}
    \item \textbf{Solar Neutrino Problem} - The sun emits electron neutrinos but when measurements began in the 1960s, rates of neutrinos were much lower than expected
    \item It was proposed and proved that electron neutrinos can turn into mu or tau neutrinos under certain circumstances, thus lepton numbers are not perfectly conserved but their sum is 
    \item It has also been speculated that not all neutrinos are massless
    \item Neutrinos may be \emph{majorana} particles, or be there own antiparticles
\end{itemize}

\subsection{Particle Classification}
\begin{itemize}
    \item The fundamental particles include the gauge bosons (photons, gluons, W and Z particles), and leptons
    \item Another particle category is the hadrons which are composite (made of quarks) particles and interact via the strong nuclear force
\end{itemize}

\subsection{Particle Stability and Resonances}
\begin{itemize}
    \item The lifetime of an unstable particle depends on which force is most active in causing its decay
    \item Particles that decay via the strong nuclear force typically have very short lifetimes
    \item A \emph{resonance} occurs when two particles interact significantly at a certain energy level
\end{itemize}

\subsection{Strangeness? Charm? Towards a New Model}
\begin{itemize}
    \item New strange particles were discovered that did not allow some reactions to occur, thus the strange quantum number was invented and its conservation law was introduced
    \item Strangeness is a partially conserved quantity, it is conserved by the strong force but not the weak
\end{itemize}

\subsection{Quarks}
\begin{itemize}
    \item Hadrons are made of smaller, point-like particles called quarks
    \item The 6 quarks are up, down, top, bottom, charm, and strange
    \item All quarks have a spin of 1/2 and a charge of \(+2/3e\) or \(-1/3e\)
    \item Current models suggest it is impossible for quarls to exist freely and must always be bound together
\end{itemize}

\subsection{The Standard Model: QCD and Electroweak Theory}
\begin{itemize}
    \item Quarks also have a color property, and can be classified as red, green, or blue, note that they are not actually colored particles
    \item Particles that obey the exclusion principle are called \emph{fermions}
    \item Each quark is assumed to carry a color charge and the strong force between them is the color force
    \item There have been efforts to find a unified basis for all four fundamental forces
    \item The standard model is a combination of the electroweak theory (which combines the electromagnetic and weak nuclear forces) and the strong interaction, all that is missing is gravity
\end{itemize}

\subsection{Grand United Theories}
\begin{itemize}
    \item \textbf{Grand Unified Theory} - A theory that incorporates multiple forces
    \item One theory attempts to unite the EM and strong and weak nuclear forces into one by stating their particles can switch at any time but only at an energy level of around \(10^{16}GeV\), at this level, distinguishing between these would become almost impossible
    \item \textbf{Symmetry Breaking} - The phenomenon that occurs when different types of particles are closer than their unification distance, the force between elementary particles at a distance of \(10^{-31} meters is thought to be the same force\)
\end{itemize}

\textbf{Proton Decay}
\begin{itemize}
    \item One testable prediction is that protons decay (and would violate the conservation of baryon number) if two quarks within it came within \(10{-31}m\) of each other
\end{itemize}

\textbf{GUT and Cosmology}
\begin{itemize}
    \item Another theory suggests that at the beginning of the universe, it was so hot that particles reached energis on the unification scale, which would have caused baryon number to not be conserved
    \item This could have led to the imbalance of matter and antimatter we see today 
\end{itemize}

\subsection{Strings and Supersymmetry}
\begin{itemize}
    \item \textbf{String Theory} - An attempt to unify all 4 forces, it suggests fundamental particles are actually strings that vibrate at a specific standing wave pattern 
    \item Applying another idea, supersymmetry, to string theory gives superstring theory
    \item \textbf{Supersymmetry} - A theory that suggests interactions that would change fermions to bosons and vice versa exist
    \item Some think current theories are approximations of a more fundamental M-Theory
\end{itemize}

\newpage