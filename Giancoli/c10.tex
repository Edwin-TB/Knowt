\section{Fluids}
The focus is now on materials with no definite shape that flow, called fluids, which include liquids and gases

\subsection{Phases of Matter}
\begin{itemize}
    \item \textbf{Liquid} - Retains its volume but changes shape to fits its container
    \item \textbf{Gas} - Same as a liquid but changes volume to fill its container
\end{itemize}

\subsection{Density and Specific Gravity}
\begin{itemize}
    \item \textbf{Density} - Defined as mass per units volume, calculated with \[\rho=\frac{m}{V}\] Where \(\rho\) is density, m is mass, and V is volume, units are usually either \(kg/m^3\) or \(g/cm^3\)
    \item Mass can be calculated from volume and density with \[m=\rho V\]
    \item \textbf{Specific Gravity} - The ratio between the density of a substance and the density of water at \(4.0^\circ\)C
\end{itemize}

\subsection{Pressure in Fluids}
\begin{itemize}
    \item \textbf{Pressure} - Force per unit area, where F is the magnitude of force perpendicular to the surface area \[P=\frac{F}{A}\] Pressure has units pascals, Pa, defined as \[1Pa=\frac{1N}{m^2}\]
    \item Fluids exert pressure in every direction
    \item At rest, fluid pressure acts perpendicular to the surface is is resting on
    \item The pressure due to the weight of a liquid is calculated with \[P=\rho gh\] Where h is the depth of the liquid
    \item Pressure varies across the depth of a liquid, the height in the previous equation is the height of the liquid above the point in question
    \item Liquids may also change in height, the change in pressure is given by \[\Delta P=\rho g\Delta h\]
\end{itemize}

\subsection{Atmospheric pressure and Guage Pressure}

\textbf{Atmospheric Pressure}
\begin{itemize}
    \item The Earth's atmosphere exerts pressure, one atmosphere of pressure is defined as \[101.3kPa\]
    \item Another unit of pressure is the bar, defined as \[1 bar=1.000*10^5N/m^2\]
\end{itemize}

\textbf{Gauge Pressure}
\begin{itemize}
    \item \textbf{Gauge Pressure} - The pressure registered y measuring devices
    \item These devices do not account for atomspheric pressure
    \item \textbf{Absolute Pressure} - The sum of gauge pressure and atmospheric pressure, calculated by \[P=P_G+P_0\] Where \(P_G\) is gauge pressure and \(P_0\) is atmospheric pressure
\end{itemize}

\subsection{Pascal’s Principle}
\begin{itemize}
    \item \textbf{Pascal's Principle} - If an external pressure is applied dot a confined fluid, the pressure at every point within the fluid increases by that amount
    \item The force input to a fluid \(F_{in}\) increases pressure equally throughout a fluid, thus in a hydraulic press, \[P_{out}=P_{in}\] Or, \[\frac{F_{out}}{F_{in}}=\frac{A_{out}}{A_{in}}\]
    \item \textbf{Mechanical advantage} - Ratio of the areas of a hydraulic device, calculated by \(F_{in}/F_{out}\)
\end{itemize}

\subsection{Measurement of Pressure: Gauges and Barometer}
\begin{itemize}
    \item \textbf{Manometer} - Simplest device used to measure pressure, where pressure is \[P=P_0+\rho g\Delta h\]
    \item \textbf{Torr} - Unit of measuring pressure, called mm-Hg or "millimeters of Mercury", defined as roughly 133Pa since that is the pressure of Mercury at a depth of one millimeter 
    \item \textbf{Aneroid Gauge} - Another type of pressure gauge, uses deformation to measure pressure and a pointer to indicate measurement
    \item \textbf{Barometer} - Modified mercury manometer, uses mercury's properties to make measurement easier
    \item A common misconception is that vacuums and other suction devices move fluids by actively moving them but this is not the case, what is actually happening is a drop in pressure that causes the atmosphere to push the fluid up into the suction device
\end{itemize}

\subsection{Buoyancy and Archimedes’ Principle}
\begin{itemize}
    \item \textbf{Buoyancy} - An upward force that causes objects to float on/in fluids, calculated by \[F_B=m_Fg\] Which is also equal to the weight of the fluid with the volume an object displaces
    \item \textbf{Archimedes' Principle} - The buoyant force on an object immersed in a fluid is equal to the weight of the fluid displaced by the object
    \item \textbf{Apparent weight} - Weight of an object in a fluid, such as water, which can be used to determine an object's density
    \item Note that the specific gravity of an object multiplied y the density of water gives its density
    \item The following relationship is true, \[\frac{V_{displ}}{V_O}=\frac{\rho_0}{\rho_F}\] Where \(V_{displ}\) - is volume displaced by an object, \(V_O\), is its full volume, \(\rho_O\), is its density, and \(\rho_F\) is the density of the fluid it is in
\end{itemize}

\subsection{Fluids in motion; FlowRate and the Equation of Continuity}
\begin{itemize}
    \item \textbf{Fluid Dynamics} - The study of fluids in motion
    \item \textbf{Hydrodynamics} - Fluid dynamics for water
    \item \textbf{Laminar Flow} - Layers of a flowing fluid slide by each other smoothly
    \item \textbf{Turbulent Flow} - Layers of a flowing fluid collide and form small whirlpools called \emph{eddies}
    \item \textbf{Viscosity} - The internal friction of a fluid, the higher viscosity it has, the slower it flows
    \item\textbf{Mass Flow Rate} - The mass of a fluid that passes a point per unit time, calculated by \[Mas flow rate = \frac{\Delta m}{\Delta t}=\rho_1A_1V_1\]
    \item\textbf{Equation of Continuity} - The flow rate at two points of the same ube with different areas is the same, given by \[\rho_1A_1V_1=\rho_2A_2V_2\]
    \item If a fluid is incompressible, that is its density does not change with pressure, its density remains constant and the equation of continuity becomes \[A_1V_1=A_2V_2\]
\end{itemize}

\subsection{Bernoulli’s Equation}
\begin{itemize}
    \item \textbf{Bernoulli's Principle} - Where the velocity of a fluid is high, the pressure is low, and where the velocity is low, the pressure is high
    \item The work done by a force to displace a fluid is given by \[W=P_1A_1l_1\] The work done on the other end of the fluid is \[W=-P_2A_2l_2\] which is negative because the force exerted is opposite to displacement, the work done by the force of gravity on a fluid is \[W=-mg(y_2-y_1)\]
    \item \textbf{Bernoulli's equation} - Combining the three previous equations gives \[P_2+\frac{1}{2}\rho v^2_2+\rho gy_2=P_1+\frac{1}{2}\rho v^2_1+\rho gy_1\] 
    \item Points 1 and 2 can be along any tube flow, therefore \[P+\frac{1}{2}\rho v^2+\rho gy=constant\]
\end{itemize}

\subsection{Applications if Bernoulli’s Principle Torricelli, Airplanes, Baseballs, Blood Flow}
\begin{itemize}
    \item \textbf{Torricelli's Theorem} - If pressure is equal at both points, Bernoulli's equation can be rearranged to solve for velocity, \[v_1=\sqrt{2g(y_2-y_1)}\]
    \item If a fluid is flowing horizontally, Bernoulli's equation simplifies to \[P_2+\frac{1}{2}\rho v^2_2=P_1+\frac{1}{2}\rho v^2_1\]
\end{itemize}

\textbf{Airplane Wings and Dynamic Lift}
\begin{itemize}
    \item \textbf{Dynamic Lift} - An upward force caused on wings with an upper rounded surface, which causes air to travel faster over the top of the wing and thus have lower pressure than air travelling under the wing
\end{itemize}

\textbf{Sailboats}
\begin{itemize}
    \item Sailboats apply a similar concept to their sails, they cause a lower pressure on their fronts which causes the sail to be pushed by higher pressure air behind it
\end{itemize}

\textbf{Baseball Curve}
\begin{itemize}
    \item Spinning baseballs curve because they cause one side of the air surrounding them to have a higher pressure than the other and thus be pushed by it
\end{itemize}

\textbf{Lack of blood to the Brain}
\begin{itemize}
    \item TIA is a temporary lack of blood to the brain
    \item This is caused when a sudden pressure change in the body combined with a blockage in the arteries, this causes low pressure on one side of the body and the other side tries to divert blood to it, but accidentally ignores the brain in the process
\end{itemize}

\textbf{Other Applications}
\begin{itemize}
    \item \textbf{Venturi Tube} - A pipe with a narrow constriciton in the middle
    \item \textbf{Venturi meter} - Used to measure flow speed across a venturi tube using the differences in area and pressure across the wider and narrower sections of the venturi tube
\end{itemize}

\subsection{Viscosity}
\begin{itemize}
    \item \textbf{Viscosity} - Internal friction between a fluid
    \item Each fluid has its own viscosity, represented by its coefficient of viscosity, \(\eta\), with units Pa*s
    \item The force required to move a plate over a fluid is given by \[F=\eta A\frac{v}{l}\] Where A is the area of the plate in contact with the fluid, v is the velocity of the plate, and l is the distance between the moving plate and stationary surface
\end{itemize}

\subsection{Flow in Tubes: Poiseuille's Equation, Blood Flow}
\begin{itemize}
    \item \textbf{Poiseuille's Equation} - Used to determine volume rate of flow of an incompressible fluid in a tube in laminar flow, mainly blood, which is \[Q=\frac{\pi R^4(P_1-P_2)}{8\eta l}\] Where R is the radius of the tube, l is the length of the tube, and Q is the volume rate of flow
\end{itemize}

\subsection{Surface Tension and Capillarity}
\begin{itemize}
    \item \textbf{Surface tension} - Fores between the molecules on the surface of a liquid that keep it together, calculated by \[\gamma=\frac{F}{l}\] Defined as Force per unit length, l, that acts perpendicular to any line in a liquid surface
    \item Soaps lower the surface tension of water which allows it to penetrate small crevices in objects being cleaned
\end{itemize}

\textbf{Capillarity}
\begin{itemize}
    \item \textbf{Cohesion} - Bonding forces between molecules of the same type
    \item\textbf{Adhesion} - Bonding forces between molecules of different types
    \item\textbf{Capillarity} - The phenomenon where liquids in thin tubes rise and fall relative to the level of the surrounding liquid
\end{itemize}

\subsection{Pumps and the Heart}
\begin{itemize}
    \item \textbf{Vacuum Pump} - Moves fluids by reducing pressure
    \item \textbf{Force Pump} - Moves fluids by increasing pressure
    \item \textbf{Circulating Pump} - Moves fluids by moving them in a closed path
    \item The heart moves blood by acting as a circulating pump, moving blood throughout two main paths in the body
\end{itemize}

\newpage