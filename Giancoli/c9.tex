\section{Static Equilibrium; Elasticity and Fracture}
Static equilibrium is the analysis of objects with no net force or net torque and no linear or rotational motion. Though objects with no net force can still experience internal forces and when these forces get too great they may deform or fracture
\textbf{Statics} - The study of forces acting on structures that are not in motion

\subsection{The Conditions of Equilibrium}
\begin{itemize}
    \item \textbf{Equilibrium} - The state of begin at rest
\end{itemize}

\textbf{The First Condition for Equilibrium}
\begin{itemize}
    \item The forces acting on an object must add up to zero, an object experiences no net force
\end{itemize}

\textbf{The Second Condition for Equilibrium}
\begin{itemize}
    \item The sum of the torques acting on an object must be zero
\end{itemize}

\subsection{Solving Statics Problems}
\begin{itemize}
    \item Statics allows us to calculate forces while others are already known
    \item In most problems a surface is considered where only 3 equations are need, one for each dimension and one for torque
    \item The strategy for statics problems consists of drawing a free body diagram, defining a coordinate system, write down the force and torque equations, and solving
    \item \textbf{Cantilever} - A beam that extends beyond its support, such a beam experiences both force and torque
\end{itemize}

\subsection{Applications to muscles and joints}
\begin{itemize}
    \item Bones in the human body connect at \emph{joints} and attach at \emph{insertions}
    \item The human body contains many joints and rotating parts that can be used as examples for statics problems
\end{itemize}

\subsection{Stability and balance}
\begin{itemize}
    \item When an object is displaced, there are 3 possible outcomes:
    \begin{itemize}
        \item \textbf{Stable Equilibrium} - the object returns to its original position
        \item \textbf{Unstable Equilibrium} - THe object moves even farther from its original position
        \item \textbf{Neutral Equilibrium} - The object remains in its new position
    \end{itemize}
    \item Stable equilibrium is also called \emph{balance}
    \item An object that is difficult to make fall over is said to be more stable than another object
    \item The wider an object's base is the more stable it is
\end{itemize}

\subsection{Elasticity; Stress and Strain}
\begin{itemize}
    \item Elasticity studies the effects of forces on objects 
\end{itemize}

\textbf{Elasticity and Hooke's Law}
\begin{itemize}
    \item \textbf{Hooke's Law} - Force exerted on an object will cause its length to change slightly, represented as \[F=k\Delta l\] Where F is the force applies, \(\Delta l\) is the change in length, and k is a proportionality constant
    \item Hooke's law remains true for most materials up to the \emph{proportional limit}, after which the relationship between F and \(\Delta l\) is not easily predicted
    \item \textbf{Elasticity Limit} - The point up to which an object will return to its original shape, elongating it beyond this point will deform the object permanently
    \item \textbf{Ultimate Strength} - Also called \emph{breaking point}, this is the maximum force that can be applied to a material before it breaks
\end{itemize}

\textbf{Young's Modulus}
\begin{itemize}
    \item The elongation of an object also depends on the material it is made of
    \item The elongation of an object can be calculated with \[\Delta l=\frac{1}{E}\frac{F}{A}l_0\] Where \(l_0\) is the original length, A is the cross-sectional area, F is the force applied, and E is the constant of proportionality, also called \emph{elastic modulus} or \emph{Young's modulus}
    \item The young's modulus of a material has units \(\frac{N}{m^2}\), the greater the young's modulus is, the more it resists elongation
\end{itemize}

\textbf{Stress and Strain}
\begin{itemize}
    \item \textbf{Stress} - The force per unit area in an elongating object, calculated with \[stress=\frac{F}{A}\] Where F is force and A is area, with units \(N/m^2\)
    \item \textbf{Strain} - The ratio between an object's change in length and its original length, calculated with \[strain=\frac{\Delta l}{l_0}\]
    \item Another important equation when dealing with stress and strain is \[\frac{F}{A}=E\frac{\Delta l}{l_0}\]
    \item From this, the following can be derived \[E=\frac{stress}{strain}\]
\end{itemize}

\textbf{Tension, Compression, and Shear Stress}
\begin{itemize}
    \item An object being pulled on is under tension and experiencing \emph{tensile stress}
    \item An object being compressed is under \emph{compressive stress}
    \item An object with equal and opposite forces across its opposite faces is under \emph{shear stress}
    \item Shear strain can be calculated with \[\Delta l=\frac{1}{G}\frac{F}{A}\Delta l_0\] Where G is the shear modulus
\end{itemize}

\textbf{Volume change - Bulk Modulus}
\begin{itemize}
    \item \textbf{Pressure} - Force per unit area on an object, equivalent to stress
    \item An object experiencing inward forces from all sides will decrease in volume
    \item The change in volume has its own constant of proportionality called the bulk modulus, B, which can be calculate with \[B=-\frac{\Delta P}{\Delta V/V_0}\] Where \(\Delta P\) is the change in pressure, \(\Delta V\) is the change in volume, and \(V_0\) is the original volume
    \item For liquids and gases, only the bulk modulus applies
\end{itemize}

\subsection{Fracture}
\begin{itemize}
    \item \textbf{Fracture} - Occurs when a solid object is under too much stress and breaks
    \item \textbf{Safety Factor} - A factor that maintains the safety of materials by not allowing them to be subject to a certain fraction of their ultimate strengths
    \item Sometimes materials are combined to combine their ultimate strengths, such as reinforces concrete which combines the compressive strength of concrete with the tensile strength of steel
\end{itemize}

\subsection{Spanning a Space: Arched and Domes}
\begin{itemize}
    \item \textbf{Arch} - An architectural design made to make its components experience compressive strength in order to avoid excessive deformation
    \item \textbf{Dome} - An arch spanning 3-dimensional space for the same purpose
    \item Each design is not stable until all the stones or other elements are in place
\end{itemize}

\newpage