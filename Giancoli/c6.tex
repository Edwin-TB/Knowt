\section{Work and Energy}
New concepts explored here are energy and momentum, as well as their laws of conservation. Work is also explored.

\subsection{Work Done by a Constant Force}
\begin{itemize}
    \item Work is done on an object when a force is applied wo move it a certain distance, calculated by \[W=Fd\cos\theta\]
    Where W is work, F is force, d is distance, and \(\cos\theta\) is the angle between the force and direction of displacement, this is included because work only take into account the component of force parallel to displacement
    \item Work is measured in Joules (J) \[1J=1N*m\]
    \item A force can be exerted on an object but if it does not move then no work was done on it
    \item Important to specify if work is being done \emph{by} or \emph{on} an object
\end{itemize}

\subsection{Work Done by a Varying Force}
\begin{itemize}
    \item Work done by a varying force can be calculated graphically, plot force as a function of distance, work is the area under the graph
    \item Dividing the graph into smaller and smaller segments makes the calculation more precise
\end{itemize}

\subsection{Kinetic Energy, and the Work-Energy Principle}
\begin{itemize}
    \item Defined traditionally as "the ability to do work"
    \item Note that total energy before and after a process remains the same, it is only converted from one form to another
    \item \textbf{Kinetic Energy} - The energy of motion, calculated by \[KE=\frac{1}{2}mv^2\]
    \item \textbf{Work-energy principle} - The net work on an object is equal to the change in the object's kinetic energy, represented as \[W_{net}=\Delta KE=\frac{1}{2}mv_2^2-\frac{1}{2}mv_1^2\]
    \item Net work means the work done by every force on the object
\end{itemize}

\subsection{Potential Energy}
\begin{itemize}
    \item \textbf{Potential Energy} - Energy waiting to be used
\end{itemize}

\textbf{Gravitational Potential Energy}
\begin{itemize}
    \item An object raised above the ground has \(PE_G\) and has the ability to work, as it falls it converts the \(PE_G\) into KE
    \item Formula for GPE \[PE_G=mpy\]
    Where \(PE_G\) is gravitational potential energy, m is the mass of the object, g is the acceleration due to gravity, and y is its height above the ground
    \item \(PE_G\) is the negative of the work done by gravity on an object
\end{itemize}

\textbf{Potential Energy Defined in General}
\begin{itemize}
    \item Each type of potential energy is associoated with a force and is the negative of the work done by that force
\end{itemize}

\textbf{Potential Energy of Elastic Spring}
\begin{itemize}
    \item Extending a spring so that it is stretched or compressed requires a force, calculated by \[F_{ext}=kx\] Where k is the spring constant (unique to each material) and x is the distance the spring is extended
    \item The force a spring extends on the item extending it is given by \[F_s=-kx\]
    Which is the negative of the force required to extend the spring
    \item The elastic potential energy of a spring is given by \[PE_{el}=\frac{1}{2}kx^2\]
    \item Note that the distance a spring is compressed of stretched is relative to its position when no force is applied, called its natural position
\end{itemize}

\textbf{Potential Energy as Stored Energy}
\begin{itemize}
    \item Potential energy is an object's potential to do work, thus it is stored energy
\end{itemize}

\subsection{Conservative and Nonconservative Forces}
\begin{itemize}
    \item Potential energy depends on the positions of objects thus it must be stated uniquely for each point
    \item Potential energy can only be defined for a conservative force, thus not every force has a potential energy
\end{itemize}

\textbf{Work-Energy Extended}
\begin{itemize}
    \item The work done by a nonconservative force is equal to the sum of the change in kinetic and potential energies, represented by \[W_{NC}=\Delta KE+\Delta PE\]
    \item Note that \emph{all} forces must be included, either in the potential or kinetic terms, but not both
\end{itemize}

\subsection{Mechanical Energy and its Conservation}
\begin{itemize}
    \item For a conservative force, the following relation remains true \[KE_2+PE_2=KE_1+PE_1\]
    or \[E_1=E_2\]
    Where E is the total mechanical energy of a system
    \item \textbf{Principe or Conservation of mechanical energy} - If only conservative forces do work, the total mechanical energy of a system neither increases nor decreases in any process. It stays constant - it is conserved
\end{itemize}

\subsection{Problem Solving Using Conservation of Mechanical Energy}
\begin{itemize}
    \item To find the total mechanical energy of a system involving \(PE_G\), the following equation is used \[E=\frac{1}{2}mv^2+mgy\] Where v is the velocity of a falling object at any point
    \item Just before a dropped object hits the ground, its PE will be 0 as all of it has converted to KE 
    \item Problems involving spring energy are similar, the total energy in a spring system remains the same if other forces such as friciton are ignored, therefore \[E=\frac{1}{2}mv_1^2+\frac{1}{2}kx_1^2=\frac{1}{2}mv_2^2+\frac{1}{2}kx_2^2\]
\end{itemize}

\subsection{Other Forms of Energy and Energy Transformations, the Law of
Conservation of Energy}
\begin{itemize}
    \item Energy can be transformed, such as from one form to another or from one object to another
    \item Work is done when energy is transferred from one object to another
    \item \textbf{Law of Conservation of Energy} - The total energy is neither increased nor decreased in any process. Energy can be transferred from one form to another, and transferred from one object to another, but the total amount remains constant
\end{itemize}

\subsection{Energy Conservation with Dissipative Forces: Solving Problems}
\begin{itemize}
    \item Forces that reduce the mechanical energy (but not total energy) of a system are called dissipative forces, which include friction
    \item Taking friction into account gives \[KE_2+PE_2+F_{fr}d=KE_1+PE_1\]
    Which includes the work done by the force of friction over a distance moved
\end{itemize}

\textbf{Work-Energy versus Energy Conservation} - 
\begin{itemize}
    \item If you study a system where external forces are at play then use the work energy principle to determine change in kinetic energy
    \item If the system has no external forces at play then use the law of conservation of energy
    \item If forces are not constant, then energy may be easier to use than Newton's laws
    \item Use the information you are given to decide on the simplest approach to a problem
\end{itemize}

\subsection{Power}
\begin{itemize}
    \item \textbf{Power} - Rate at which work is done, average power is work divided by time, represented by \[P=\frac{Work}{Time}=\frac{Energy Transformed}{Time}\]
    \item The SI unit of power is watts, defines as \[1W=\frac{1J}{s}\]
    \item For example, a person walking one mile would feel less tired than someone running the same distance because it is easier for the body to transform chemical potential energy to mechanical energy over a longer length of time
    \item Vehicles are limited by the rate at which they can do work
    \item Power can also be calculated by \[P=Fv\] Where F is force and v is velocity
    \item \textbf{Efficiency} - The ratio of useful power output to the power input, represented by  \[e=\frac{P_{out}}{P_{in}}\]
\end{itemize}

\newpage