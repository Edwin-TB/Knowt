\section{Linear Momentum}
There exist other laws of conservation in physics, such as linear momentum discussed in this chapter. Momentum is useful in analyzing a system where multiple objects interact with each other

\subsection{Momentum and its Relation to Force}
\begin{itemize}
    \item \textbf{Linear Momentum} - The product of an object's mass and velocity, represented by \[\vec{p}=m\vec{v}\] Where p is momentum
    \item Forces are required to change an object's momentum, the net force on an object is its change in momentum over time, represented by \[\Sigma F=\frac{\Delta p}{\Delta t}=m\frac{\Delta v}{\Delta t}\]
    \item Note that the quotient of momentum divided by time takes into account possible changes in mass of objects studied
    
\end{itemize}

\subsection{Conservation of Momentum}
\begin{itemize}
    \item If no external forces act on a system, total momentum remains constant, momentum before = momentum after, represented by \[m_Av_A+m_Bv_B=m_Av'_A+m_Bv'_B\] Which is the sum of the momentums of objects A and B before and after the event denoted by the apostrphe
    \item If two objects collide then the force exerted on the first is the negative of the force exerted on the second, by Newton's third law, thus \[F=\frac{\Delta p_B}{\Delta t}\] and \[-F=\frac{\Delta p_A}{\Delta t}\]
    \item \textbf{Law of Conservation of Momentum} - The total momentum of an isolated system of objects remains constant
    \item \textbf{System} - Set of objects chosen to study
    \item \textbf{Isolated System} - A system in which the only forces are those between the objects in the system
\end{itemize}

\subsection{Collisions and Impulse}
\begin{itemize}
    \item Two objects that are moving relative to each other and touch are said to collide
    \item When two objects collide, their velocities change and thus their momentums change
    \item \textbf{Impulse} - The change in momentum of an object after a collision, shown by \[Impulse=F\Delta t\]
    \item Used mainly when dealing with small amounts of time, force is likely to vary during the short time and so the average force is used
\end{itemize}

\subsection{Conservation of Energy and Momentum in Collisions}
\begin{itemize}
    \item \textbf{Elastic Collision} - A collision in which the total kinetic energy is conserved before and after two objects collide, represented by \[KE_A+KE_B=KE'_A+KE'_B\]
    \item Most collisions that can be seen are not elastic as kinetic energy is lost externally due to sound or friction
    \item \textbf{Inelastic Collision} - A collision during which kinetic energy is not conserved, represented by \[KE_A+KE_B=KE'_A+KE'_B+other forms of energy\]
\end{itemize}

\subsection{Elastic Collisions in One Dimension}
\begin{itemize}
    \item Rearranging conservation of momentum gives \[v_A+v'_A=v_B+v'_B\] Which can be rewritten as \[v_A-v_B=-(v'_A-v'_B)\]
\end{itemize}

\subsection{Inelastic Collisions}
\begin{itemize}
    \item \textbf{Completely Inelastic} - A collision during which two objects stick together
    \item KE is not conserved but total energy remains constant
\end{itemize}

\subsection{Collisions in Two Dimensions}
\begin{itemize}
    \item An elastic collision with a moving and stationary object can be modeled by \[m_Av_A=m_Av'_A\cos\theta'_A+m_Bv'_B\cos\theta'_B\] Where $\theta$ is the angle between the path of each objects and the original path of object A
    \item The other component of object A's momentum is 1, thus \[0=m_Av'_A\sin\theta'_A+m_Bv'_B\sin\theta'_B\]
    \item Applying the conservation of kinetic energy gives \[\frac{1}{2}m_Av_A^2=\frac{1}{2}m_Av'^2_A+\frac{1}{2}m_Bv'^2_B\]
\end{itemize}

\subsection{Center of Mass (CM)}
\begin{itemize}
    \item Until now, objects have been assumed to be point particles with no shape, only mass, but in real life objects may have many parts that move relative to each other
    \item The point in an object that moves in the same path a point particle would is its \emph{center of mass} (CM)
    \item The general motion of an extended object is the sum of the translational motion of its CM plus other types o motion around the CM such as vibrational
    \item The center of mass of a two-object system is calculated with \[x_{CM}=\frac{m_Ax_A+m_Bx_B}{M}\] Where x is the position of each object on a coordinate system chosen such that each object is on the x-axis, m is the mass of each object, and M is the total mass of the two objects
    \item The center of mass of a multidimensional system requires coordinates in more than one dimension, the y coordinate of the CM of a system can be calculated with \[y_{CM}=\frac{m_Ay_A+m_By_B}{M}\]
    \item \textbf{Center of Gravity} - The point at which gravity can be considered to act on a system/object
    \item Note that gravity actually acts on all points of a system and it is usually easier to determine CM and CG through experimentation rather than calculation
\end{itemize}

\subsection{CM for the Human Body}
\begin{itemize}
    \item As an example of calculating CM, each of a person's major body parts (upper and lower limbs, neck back, head) are assigned a CM and mass
    \item These assigned values are used to determine the person's CM
\end{itemize}

\subsection{CM and Translational Motion}
\begin{itemize}
    \item The formula for center of gravity can be rewritten as \[Mx_{CM}=m_Ax_A+m_Bx_B\] 
    \item The velocity of a system's CM when the objects have different velocities is \[Mv_{CM}=m_Av_A+m_Bv_B\] 
    \item The acceleration of a system's CM when the objects have different accelerations is \[Ma_{CM}=m_Aa_A+m_Ba_B\]
    \item The net force on all the objects on a system is equal to the product of the total mass of the system and the acceleration of its CM, represented by \[Ma_{CM}=F_{net}\] This is called \emph{Newton's second law for a system of particles}
    \item The CM of a system of particles moves as if all of its mass is centered at one point and forces are applied only to that point
    \item One application of this is in analyzing nearby stars that seem o wobble in their movement, such wobbles can be caused by other stars or by planets orbitting it
\end{itemize}

\newpage