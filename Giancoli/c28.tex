\section{Quantum Mechanics of Atoms}
Bohr's model could not predict the spectra for complex atoms and could not explain why emission lines consist of multiple, very closely spaced lines. It was not theoretically sufficient and showed a more comprehensive theory was needed.

\subsection{Quantum Mechanics- A New Theory}
\begin{itemize}
    \item \textbf{Quantum Mechanics} - Unifies wave-particle duality into a single theory and successfully deals with spectra emitted by cmplex atoms
    \item Quantum mechanics satisfies the correspondence principle as it would give the classical formulas if applied to the macroscopic world
\end{itemize}

\subsection{The Wave Function and Its Interpretation; the Double-Slit Experiment}
\begin{itemize}
    \item \textbf{Wave Function} - Represents EM wave amplitude as a function of time and position, represented by \(\psi\)
    \item When applied to an electron in an atom, \(\psi^2\) at a certain point in space and time represents the probability of finding the electron at a given position and time, thus it is referred to as the \emph{probability density} of the electron
\end{itemize}

\textbf{Double-Slit Interference Experiment for Electrons}
\begin{itemize}
    \item Applying \(\psi^2\) to the double slit experiment gives a maximum where the electrons are more concentrated and a minimum where they are sparse
\end{itemize}

\subsection{The Heisenberg Uncertainty Principle}
\begin{itemize}
    \item Quantum mechanics puts a limit on how precise measurements can get as measuring an object without disturbing it is impossible
    \item Using photons to measure small objects like electrons causes uncertainty in both the electron's position and momentum
    \item \textbf{Heisenberg Uncertainty Principle} - Tells that measuring the position and momentum of an object at the same time is impossible, given by \[(\Delta x)(\Delta\rho_x)\geq\frac{h}{2\pi}\] Where \(\Delta x\) is uncertainty in position and \(\Delta\rho_x\) is uncertainty in momentum
    \item The principle can also be stated in terms of energy and time \[(\Delta E)(\Delta t)\geq\frac{h}{2\pi}\] Where \(\Delta E\) is uncertainty in energy and \(\Delta t\) is uncertainty in time
\end{itemize}

\subsection{Philosophic Implications: Probability Versus Determinism}
\begin{itemize}
    \item The states of extremely small particles are determined with probability rather than exact calculations
    \item Because of this, and the fact that objects are made of these tiny particle, the states of macroscopic objects are expected to be governed by probability
    \item \textbf{Copenhagen Interpretation} - A way of looking at probability as an inherent nature of quantum mechanics
\end{itemize}

\subsection{Quantum-Mechanical View of Atoms}
\begin{itemize}
    \item \textbf{Cloud} - Describes the position of electrons around an atom from a particle view, electrons exist as a cloud of probability rather than a set position
    \item Clouds can be though of as probability distributions of electrons
\end{itemize}

\subsection{Quantum Mechanics of the Hydrogen Atom; Quantum Numbers}
\begin{itemize}
    \item Four quantum numbers specify each electron state in an atom
    \item The first is doneted by \(n\), and is called the \emph{principle quantum number}, from the Bohr model, gives the energy state
    \item The second is the \emph{orbital quantum number}, \(l\), and ranges from 0 to \(n-1\), gives the angular momentum of an electron by \[L=\sqrt{l(l+1)h}\]
    \item The third is the \emph{magnetic quantum number}, \(m_t\) and is related to the direction of the electron's angular momentum, ragning from \(-l\) to \(+l\) in integers
    \item The fourth is the \emph{spin quantum number}, \(m_s\) and can have a value of \(\pm1/2\) for an electron
    \item \textbf{Orbitals} - Different shapes of probability distributions for electrons at different energy levels
\end{itemize}

\textbf{Selection Rules}
\begin{itemize}
    \item Selection rules limit what electrons can do, one such rule is an electron can only transition between states with values of \(l\) that differ by one unit, \(\Delta l=\pm1\)
    \item \textbf{Forbidden Transition} - A transition that violates a selection rule
    \item \textbf{Allowed Transition} - Transitions that do not violate selection rules
    \item Forbidden transitions can actually occur but they are rare
\end{itemize}

\subsection{Multielectron Atoms; the Exclusion Principle}
\begin{itemize}
    \item \textbf{Atomic Number} - The number of protons or electrons in a neutral atom
    \item \textbf{Pauli Exclusion Principle} - no two electrons in an atom can occupy the same quantum state, thus they cannot have the exact same 4 quantum numbers
    \item All protons neutrons and electrons are identical, thus if two of the same particle had the same quantum numbers they would be the same particle
\end{itemize}

\subsection{The Periodic Table of Elements}
\begin{itemize}
    \item \textbf{Periodic Table} - A method of organizing elements by their properties
    \item Electrons with the same value of \(n\) are referred to as being in the same shell
    \item Electron with the same \(n\) and \(l\) are referred to as being in the same subshell
    \item The Pauli exclusion principle limits how many electrons can be in each shell, this number increases each shell
    \item The electron configuration of an element is given by assigning a sperscript to each subshell that tell show many electrons are in it, for example, the ground state configuration for sodium (11 electrons) is \(1s^22s^22p^63s^1\)
    \item Groups (row) in the periodic table have different numbers of electrons in their outermost shells but the transition metals are irregular 
\end{itemize}

\subsection{X-Ray Spectra and the Atomic Number}
\begin{itemize}
    \item X-rays are produced when electrons are accelerated by a high voltage and strike a metal target in the x-ray tube
    \item The graph of wavelength and intensity of x-rays produced gives a minimum wavelength \(\lambda_0\) depend dent on the voltage used, and two peaks, \(K_alpha, K_beta\) which are dependent on the target material
    \item The characteristic x-rays produced are from electrons dropping into lower energy levels, the peaks are produced by electrons dropping into the first shell
    \item \textbf{Bremsstrahlung} - The emission of radiation by a decelerating electron as it loses momentum
    \item The minimum wavelength of x-rays in a vacuum tube is given by \[\lambda_0=\frac{hc}{eV}\] Where \(eV\) is the elctron's charge
\end{itemize}

\subsection{Fluorescence and Phosphorescence}
\begin{itemize}
    \item \textbf{Fluorescence} - The absorption of UV photons and release of visible photons by a material
    \item \textbf{Fluorescent Light bulbs} - Use this phenomenon, they excite a gas that emits UV photons on a coating then then emits visible light
    \item \textbf{Phosphorescence} - The phenomenon of atoms being excited and releasing their energy much later due to the jump being forbidden
\end{itemize}

\subsection{Lasers}
\begin{itemize}
    \item \textbf{Laser} - A device that can produce a narrow beam of monochromatic, coherent light, meaning at any point, all the waves are in phase
    \item \textbf{Stimulated Emission} - The action of a photon pushing an excited atom to release energy at the same wavelength as the original photon, thus two photons of the same wavelength exist
    \item To obtain coherent light, the atoms must be excited so that more excited ones exist than ground state ones, called an \emph{inverted population} and the excited state must be metastable or exist excited longer than usual
\end{itemize}

\textbf{Creating an Inverted Population}
\begin{itemize}
    \item \textbf{Optical Pumping} - Method of creating an inverted population that uses strong flashes of light to put chromium atoms in a metastable state, once these atoms drop down the photons they emit continue to stimulate emission
    \item \textbf{PN Junction Laser} - Uses the inverted population of electrons between the conduction and valence band to stimulate emission
\end{itemize}

\textbf{Applications}
\begin{itemize}
    \item DVDs and CDs use lasers to be read by having the laser reflect off of pits and valleys on their surface
    \item Lasers are also used in medicine as they can destroy a local area
\end{itemize}

\subsection{Holography}
\begin{itemize}
    \item \textbf{Hologram} - A 3-d image produced using lasers
    \item This is achieved by splitting lasers on a half-silvered mirror and having half of it reflect off an object and the other half continue onto a film
    \item The light reflected off the object reaches the film and produces the image
    \item \textbf{Volume Holograms} - Do not require a laser, but can be viewed with white light on a thick emulsion that shows where destructive interference occured when making the hologram
\end{itemize}

\newpage