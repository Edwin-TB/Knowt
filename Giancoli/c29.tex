\section{Molecules and Solids}
Quantum Mechanics has had a great impact on our lives. As we have come to understand more about the topic, we have been able to use it to our advantage to develop and improve the study of semiconductors. 

\subsection{Bonding in Molecules}
\begin{itemize}
    \item \textbf{Chemical Bond} - The attachment of a group of multiple atoms and are held together as a single unit, two main types
\end{itemize}

\textbf{Covalent Bonds}
\begin{itemize}
    \item The \emph{sharing} of electrons between two atoms
    \item Covalent bonds are a result of constructive interference of electron wave functions in the space between two atoms
    \item \textbf{Bond Energy} - The energy required to break a bond
\end{itemize}

\textbf{Ionic Bonds}
\begin{itemize}
    \item Unequal sharing of electrons
    \item For example, in sodium chloride (NaCl) sodium's outer electron spends most of its time around the chlorine atom 
    \item In an ionic bond, atoms become ions, that is, they become electrically charged by gaining or losing electrons
    \item Ionic bonds are caused by the fact that the atoms that gain electrons exert a stronger force on the electrons than the atoms that lose electrons
\end{itemize}

\textbf{Partial Ionic Character of Covalent Bonds}
\begin{itemize}
    \item \textbf{Pure Covalent Bond} - A covalent bond in which electrons are shared equally across atoms 
    \item Many covalent bonds are not purely covalent and share electrons somewhere between a pure covalent and ionic bond, which is called \emph{partial ionic character}
    \item Molecules with partial ionic character are polar, that is they have parts with a net positive and net negative charge
\end{itemize}

\subsection{Potential-Energy Diagrams for Molecules}
\begin{itemize}
    \item \textbf{Potential Energy Diagram} - A plot of the potential energy versus separation distance
    \item For two point charges, the potential energy, \(PE\), is given by \[PE=k\frac{q_1q_2}{r}\] Where \(k\) is a constant \(k=9.0*10^9N*m^2/C^2\)\(q_1\) and \(q_2\) are the strengths of the two charges and \(r\) is the separation distance
    \item As \(PE\) increases, \(KE\) increases
    \item For two polar molecules, the PE diagram is negative and increasing in magnitude as \(r\) decreases since the closer the atoms get, the stronger the electric force get; After a certain point, the diagram curves up sharply as the nuclei of the atoms/molecules repel each other 
    \item There is an optimal separation of atoms that causes potential energy to be at its lowest, called the \emph{Binding Energy}, which tells how much energy must be put in to make the atoms separate to infinity
    \item Some bonds have Pe diagrams that are positive at larger distances and require more energy to get over the initial hump and attract atoms, called the \emph{activation energy}
    \item Other bonds require energy input to form, and release energy when they are broken 
\end{itemize}

\subsection{Weak (van der Waals) Bonds}
\begin{itemize}
    \item \textbf{Strong Bonds} - Bonds that hold atoms together to form molecules
    \item \textbf{Weak Bonds} - An attachment between molecules due to electrostatic attraction, usually cause by attraction between dipoles
    \item \textbf{Electric Dipole} - A pair of point charges with equal magnitude and opposite sign
    \item Dipole-Induced Dipole bonds are caused by a dipole molecule inducing a dipole in a non-polar molecule
    \item \textbf{Hydrogen Bond} - The strongest of the weak (Van der Waals) bonds, occurs in a dipole-dipole bond when one of the dipoles is hydrogen as it has some covalent nature and can form longer lasting bonds
\end{itemize}

\textbf{Protein Synthesis}
\begin{itemize}
    \item Weak bonds are crucial to protein synthesis
    \item Proteins are made of chains of molecules called \emph{amino acids}, the instructions for which are provided by an organism's \emph{genetic code}
    \item 
\end{itemize}

\subsection{Molecular Spectra}
\begin{itemize}
    \item When atoms bond, the probability distributions of their electrons intersect and their energy levels change
    \item \textbf{Band Spectra} - THe overlapped transitions from one energy level to the next in a molecule, these are difficult to discern
\end{itemize}

\textbf{Rotational Energy Levels in Molecules}
\begin{itemize}
    \item The kinetic energy of a diatomic molecule rotating about its center is \[E_{rot}=\frac{1}{2}I\omega^2\] Where \(I\omega\) is \[I\omega=\sqrt{l(l+1)}\] Where \(h\) is plank's constant and \(l\) is an integer
    \item Rotational energy is quantized
    \item Transitions between rotational energy levels are subject to the selection rule
\end{itemize}

\textbf{Vibrational Energy Levels in Molecules}
\begin{itemize}
    \item The vibrational energy of a diatomic molecule is given by \[E_{vib}=(v+\frac{1}{2})hf\] Where v is an integer called the \emph{vibrational quantum number} 
    \item The lowest energy state is given by \(v=0\) which means there is still some energy, called the \emph{zero-point energy}
    \item Transitions between vibrational energy levels are subject to the selection rule, so a change in vibrational energy can only be \(\Delta E_{vib}=hf\)
\end{itemize}

\subsection{Bonding in Solids}
\begin{itemize}
    \item \textbf{Solid-STate Physics} - The study of matter in solid form
    \item \textbf{Amorphous} - Describes matter where particles are not in an arranged pattern
    \item \textbf{Crystalline} - Describes matter where particles are aranged in a pattern, called a \emph{lattice}
    \item In ionic bonding, one atoms does not "belong" to a specific bond, rather the atom is shared by the particles around it
    \item \textbf{Metallic Bond} - The idea that in a metal structure, electrons flow freely around atoms instead of belonging to a specific atom
    \item A metallic structure is held together by the electrostatic attraction between the lattice of metal ions and sea of free roaming electrons
    \item The free electrons are what give metals conductivity; they carry vibrational energy throughout the structure to the atoms and conduct heat, and if they all move in the same direction they can carry electrical potential
    \item There also exist weak bonds in substances such as noble gases that make their particles hold on very weakly
\end{itemize}

\subsection{Free-Electron Theory of Metals; Fermi Energy}
\begin{itemize}
    \item The free electron theory views electrons in a metal structure as always moving no matter what and obey the exclusion principle in that they can only move at explicit energy states
    \item \textbf{Fermi-Dirac Statistics} - Another quantum statistic obeyed by electrons, states that no two electrons can have the same set of quantum numbers
    \item \textbf{Fermi Level} - The highest energy level that can be filled by two electrons in an atom
    \item \textbf{Fermi Energy} - The energy state at the fermi level
    \item In a metal, the average energy in its electrons changes very little when temperature changes, note that this is different from an ideal gas
    \item Thinking of electrons in a metal lattice as a gas provides a good explanation for the conductive properties of metals
\end{itemize}

\subsection{Band Theory of Solids}
\begin{itemize}
    \item As two atoms get closer, their energy electron shell states split into 2; when 6 atoms do the same, their energy levels split into 6
    \item When many atoms come together to form a solid, their electron states split into a band, energy levels so close together they seem continuous
    \item This explains why the spectrum of a heated solid appears continuous
    \item Good conductors have a highest energy band that is only partially filled
    \item A good insulator has a full highest energy level, so that their valence band and conduction bands are far apart
    \item Semiconductors have their valence and conduction bands close together
\end{itemize}

\subsection{Semiconductors and Doping}
\begin{itemize}
    \item \textbf{Doping} - The process of introducing an impurity into the crystal structure of silicon to change its conductive properties
    \item Silicon has 4 valence electrons, and each atom is bonded to 4 others, thus its valence shell is full and there are no empty spots for electrons to move freely, so pure silicon is not a conductor
    \item Adding elements with different amounts of valence electrons is called \emph{doping} and can improve the conductivity of Silicon
    \item Adding an element from group 5 of the periodic table introduces excess electrons and creates an \emph{n-type} Semiconductor
    \item Adding an element from group 3 of the periodic table introduces an empty space in the silicon structure where electrons can move around in and creates a \emph{p-type} Semiconductor
    \item Note that n-type and p-type semiconductors have no net charge
    \item Band theory states that doping has additional energy states between the valence and conduction bands, and thus makes silicon electrons require less energy to jump between bands
\end{itemize}

\subsection{Semiconductor Diodes, LEDS, OLEDs}
\begin{itemize}
    \item \textbf{pn Junction Diode} - Forms when an n- and p-type semiconductor meet, some electrons from the n-type flow over and fill in holes in the p-type, giving the n- a positive charge and the p- a negative charge, forming a potential difference
    \item The junction area where all holes are filled is called the \emph{depletion layer} because all extra electrons and holes are depleted
    \item If the positive terminal of a battery is connected to the p-side and the negative terminal to the n-side of a pn junction diode, the external voltage will oppose the intrinsic potential difference, and the diode is said to be \emph{forward biased}
    \item A great enough voltage will overcome the natural potential difference and allow current to flow
    \item When a diode is reversed biased, the positive holes and extra electrons are pulled on by the battery's terminals in opposite directions and do not get close enough to allow a current to flow
    \item \textbf{Breakdown} - Occurs when too high of a voltage is applied to a reverse biased diode, its atoms ionize and allow electrons to flow in the direction of the batteries terminals
    \textbf{Zener Diode} - A designed to regulate voltage supply as long as voltage is maintained across their breakdown point
    \item Diodes are called \emph{non-linear devices} because voltage an current across them is not proportional 
\end{itemize}

\textbf{Rectifiers}
\begin{itemize}
    \item Diodes can serve as \emph{rectifiers} - devices that convert AC to DC 
    \item \textbf{Half-Wave Rectification} - Occurs when a diode is place on an AC circuit, the diode blocks current flow on the backwards cycle of the AC cycle
    \item \textbf{Full-Wave Rectification} - Uses 2 or 4 diodes to make the entire wave of an AC current flow forwards, these circuits sometimes use an RC circuit to smooth out the wave flow
\end{itemize}

\textbf{Photovoltaic Cells}
\begin{itemize}
    \item \textbf{Photovoltaic Cells} - Heavily doped pn junctions that convert sunlight to electricity
    \item Photons are absorbed and excite an electron to a higher energy level, leaving a hole, the produced electron and hole then allow current to flow
\end{itemize}

\textbf{LED}
\begin{itemize}
    \item \textbf{Light Emitting Diode} - A type of diode where electrons emit a photon when flowing across the pn junction
    \item Light emission is achieved with compound semiconductors which usually involve a group III and group V element bonded together
    \item 
\end{itemize}

\textbf{Pulse Oximeter}
\begin{itemize}
    \item \textbf{Pulse Oximeter} - A device that uses 2 LEDs to measure the \% of oxygen saturation in blood
    \item The diodes release light which is detected by a photodiode, the ratio of absorbed light (red/IR) is used to calculate oxygen saturation
\end{itemize}

\textbf{pn Diode Lasers}
\begin{itemize}
    \item \textbf{Diode Lasers} - use a pn-junction in forward bias 
    \item When one electron drops into a lower energy state, it releases a photon that pushes other electrons to drop in state and release photons as well
\end{itemize}

\textbf{OLED}
\begin{itemize}
    \item Organic LEDs use organic compounds to perform the same basic functions as a conventional LED
    \item They consist of an emissive layer and conductive layer placed between two electrodes
    \item OLEDs can be smaller and more power efficient than conventional LEDs
    \item How they work: At a high enough voltage, electrons move into a higher state in the emissive layer and holes form in the conductive layer, these electrons and holes meet at the junction and combine to emit a photon
\end{itemize}


\subsection{Transistors: Bipolar and MOSFETs}
\begin{itemize}
    \item \textbf{Bipolar Junction Transistor} - Consist of a crystal of one type of doped semiconductor between two of the other type, called npn or pnp
    \item Each semiconductor is given the name \emph{collector, emitter, and base}
    \item \textbf{Amplifier} - One application of an npn transistor, a battery maintains a DC voltage across the collector and emitter and a voltage is aplied to the base called the \emph{base bias voltage}
    \item If the BBV is positive, the electrons in the emitter are attracted into the base, then flow into the collector, thus a large current flows through the collector and emitter but only a small current flows through the base
    \item \textbf{Current Gain} - Defined as \[\frac{output\ ac\ current}{input\ ac\ current}\]
    \item \textbf{Voltage Gain} - Defined as \[\frac{output\ ac\ voltage}{input\ ac\ voltage}\]
    \item \textbf{MOSFET} - A type of transistor commonly used in digital circuits as a switch, the emitter is called the source, the collector is called the drain, and the base is called the gate
\end{itemize}

\subsection{Integrated Circuits, 22-nm Technology}
\begin{itemize}
    \item Circuits can be made by inserting tiny impurities in a silicon wafer to produce components such as diodes, transistors, and resistors (called integrated circuits)
    \item The amount of transistors that can fit on a given surface area of silicon has been doubling every 2 to 3 years 
    \item \textbf{Technology Generation} - Refers to the minimum width of a conducting line in a circuit, but the gate of a MOSFET can be smaller
\end{itemize}

\newpage