\section{Nuclear Energy; Effects and Uses of Radiation}
This chapter focuses on applications and effects of nuclear radiation.

\subsection{Nuclear Reactions and the Transmutation of Elements}
\begin{itemize}
    \item \textbf{Transmutation} - Changing from one element to another, also occurs during nuclear reactions
    \item \textbf{Nuclear Reaction} - Occurs when a nucleus is struck by another nucleus or a simpler particle
    \item In a nuclear reaction, charge and nucleon number are maintained
    \item \textbf{Reaction Energy} - Used to calculate the energy used/expelled when particle a hits nucleus X and turns into nucleus Y and particle b, defined as \[Q=(M_a+M_X-M_b-M_Y)c^2\] This value is negative when energy is used and positive when energy is released
\end{itemize}

\textbf{Neutron Physics}
\begin{itemize}
    \item Enrico Fermi discovered that hitting nuclei with neutrons had a much higher chance of causing a reaction
    \item The first transuranic (elements with atomic number greater than 92) elements were discovered by bombarding uranium nuclei with neutrons, creating heavier nuclei of elemtns 93, 94, ...
\end{itemize}

\textbf{Collision Cross Section}
\begin{itemize}
    \item \textbf{Collision Cross Section} - Used to predict the probability of certain nuclear reactions happening
    \item Imagine a large area made of individual objects, each object has a cross sectional area, \(\sigma\), and is being hit by smaller objects, the area has a total cross section of \[A'=nAl\sigma\] Where the quantity \(nAl\) is the total number of object that comprise the area
    \item \(\sigma\) can be determined by \[\sigma=\frac{R}{R_0nl}\]
\end{itemize}

\subsection{Nuclear Fission: Nuclear Reactors}
\textbf{Nuclear Fission and Chain Reactions}
\begin{itemize}
    \item Fission is the process through which a large nucleus splits into two smaller nuclei roughly half the size of the original
    \item \textbf{Liquid Drop Model} - A uranium-235 nucleus absorbs a neutron and becomes a \emph{compound nucleus}, uranium-236
    \item The extra energy puts the nucleus in an excited state and elongates slightly, weakening the weak nuclear force between the two ends and the nucleus splits in two
    \item \textbf{Fission Fragments} - The resulting nuclei from a fission reaction
    \item A chain reaction could occur as uranium fission releases some neutrons too, which could be used to cause fission in neighboring uranium molecules
    \item Nuclear reactors take advantage of the chain reaction and use the collective energy of each fission event to produce heat
\end{itemize}

\textbf{Nuclear Reactors}
\begin{itemize}
    \item \textbf{Moderator} - A substance used to slow down neutrons as they usually travel too fast to be absorbed by uranium atoms
    \item \textbf{Enrichment} - A process used to increase the percentage of uranium-235 in a sample of uranium in order to increase the likelihood of a nuclear reaction
    \item \textbf{Critical Mass} - The minimum mass of uranium required to achieve a self-sustaining reaction
    \item \textbf{Neutron Multiplication Factor} - The average number of neutrons produces by a reaction,represented by f, a self-sustaining reaction must have a NMF of \(f\geq1\)
\end{itemize}

\textbf{Atom Bomb}
\begin{itemize}
    \item Nuclear Bombs use a similar principle, but do not allow the uranium to reach its critical mass until it is time to detonate, after which it releases its energy quickly in an unhindered burst
    \item Atom bombs produce radioactive fragments called \emph{fallout}
\end{itemize}

\subsection{Nuclear Fusion}
\textbf{Nuclear Fusion; Stars}
\begin{itemize}
    \item \textbf{Nuclear Fusion} - Bringing two nuclei together to build a larger nuclei 
    \item Fusion produces energy because the binding energy per nucleon is less for lighter nuclei than for heavier nuclei
    \item The proton-proton chain where hydrogen nuclei fuse into helium nuclei in the sun goes as follows \[^1_1H\rightarrow ^4_2He+2e^++2v+2\gamma\]
\end{itemize}

\textbf{Possible FUsion Reactors}
\begin{itemize}
    \item One of the primary reactions that humans could use in nuclear fusion reactors to produce energy is \[^2_1H+^3_1H\rightarrow^4_2He+n\]
    \item Using a fission bomb to detonate a fusion bomb is easy but harnessing usable energy from fusion has proven difficult as the extreme conditions necessary to do so are not readily available on Earth
    \item \textbf{Magnetic Confinement} - One such method developed to contain the high temperature plasma necessary for fusion, uses strong magnetic fields to keep the plasma suspended in air and rotate it to increase its kinetic energy
    \item \textbf{Lawson Criterion} - A value that the product of ion density (\(ions/m^3\) and confinement time, \(\tau\), must surpass in order to produce ignition of a fusion reaction, defined as \(3*10^20s/m^3\)
\end{itemize}

\subsection{Passage of Radiation Through Matter: Biological Damage}
\begin{itemize}
    \item \textbf{Ionizing Radiation} - Any type of radiation that can ionize an atom of any material
    \item Radiation damage in organisms is caused by ionization in cells
    \item The radiation disrupts regular processes in the cell  by knocking out electrons that hold together proteins or altering genetic code
    \item If enough radiation is received, it could kill too many cells or cause cancer in the organism
\end{itemize}

\subsection{Measurement of Radiation-Dosimetry}
\begin{itemize}
    \item \textbf{Dosimetry} - The study of measuring dosage of radiation
    \item \textbf{Source Activity} - The rate of nuclear decays per second from a source, unit is the \emph{curie}, defined as \[1Ci=3.70*10^{10}decays/s\]
    \item SI unit for source activity is the becquerel, defined as \[1Bq=1decay/s\]
    \item \textbf{Absorbed Dose} - Another method of measuring exposure to radiation, this measures the effect the radiation has on the absorbing material
    \item Absorbed dose is measured in \emph{rads}, defined as the amount of radiation that deposits \(0.01J/kg\) of energy per units mass, another unit is the gray, defined as \[1Gy=1J/kg=100rad\]
    \item \textbf{Relative Biological Effectiveness} - The number of rads or X-ray or \(\gamma\) radiation that produces the same biological damage as 1 rad of the given radiation
    \item \textbf{Effective Dose} - Defined as the product of doese in rads and RBE, units is rem (rad equivalent man) \[rem=rad*RBE\]
    \item Another unit of effective dose is the sievert, defined as \[1Sv=100rem\]
\end{itemize}

\textbf{Human Exposure to Radiation}
\begin{itemize}
    \item We are regularly exposed to radiation from sources such as cosmic rays and rocks
    \item \textbf{Natural Radioactive Background} - A source of radiation that averages 0.30rem per year per person in the U.S.
    \item People who work around radiation should only receive a maximum of 20mSv of radiation per year, their dosage is measured by a dosimeter they carry around
    \item \textbf{Radiation Sickness} - A disease caused by large doses of radiation, symptoms include loss of body hair and nausea
\end{itemize}

\subsection{Radiation Therapy}
\begin{itemize}
    \item Radiation can be used to treat cancer
    \item \textbf{Radiation Therapy} - A cancer treatment that works by focusing large amount of radiation on cancer cells in order to destroy them
    \item Protons release most of their energy at the end of their path, thus they can be set to destroy cells at a chosen depth by changing the amount of energy behind the proton
    \item Another technique used is placing a radioactive source in the tumor which will eventually kill most of the cells
\end{itemize}

\subsection{Tracers in Research and Medicine}
\begin{itemize}
    \item \textbf{Tracer} - A compound that incorporates a radioactive isotope and can be traced within the body
    \item Tracers can be used to track how food is digested in the body or where certain compounds are diverted to
    \item \textbf{Auto radiography} - Radioactive isotopes are tracked on film
    \item \textbf{Gamma Camera} - Simultaneously record radioactivity at many points which can be displayed on a screen for ease of visualization
\end{itemize}

\subsection{Emission Tomography: PET and SPECT}
\begin{itemize}
    \item \textbf{Single Photon Emission Computed Tomography (SPECT)} - The process of using gamma cameras to detect radioactive intensity at many points
    \item\textbf{Positron Emission Tomography} - Uses positron emitters in a compound that accumulates in a single area to further observe it
    \item Both SPECT and PET give images tha show biochemistry and metablism
\end{itemize}

\subsection{Nuclear Magnetic Resonance (NMR) and Magnetic Resonance Imagine (MRI)}
\textbf{Nuclear Magnetic Resonance (NMR)}
\begin{itemize}
    \item This technique places the sample to be studied in a static magnetic field, photons are applied to the sample, if the frequency of the photons correspond to a difference in energy levels then nuclei in the sample will be excited
    \item For hydrogen nuclei, this frequency is 42.58MHz for a magnetic field of 1.0T
    \item Neighboring nuclei and electrons will affect the frequency at which a nucleus becomes excited
\end{itemize}

\textbf{Magnetic Resonance Imaging}
\begin{itemize}
    \item This is the use of NMR for medical imaging
    \item Hydrogen is often used for MRI as it is the most abundant element in the body and produces the strongest NMR signals
\end{itemize}

\newpage