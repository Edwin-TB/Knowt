\section{Magnetism}
Electricity and magnetism are closely related, electric currents produce a magnetic field

\subsection{Magnets and Magnetic Field}
\begin{itemize}
    \item \textbf{Poles} - The two ends/faces of a magnet, do not confuse them with electric charge
    \item \textbf{Magnetic Monopole} - A magnet with a single poles instead of two
    \item \textbf{Ferromagnetic} - Metals and alloys that are not iron but are influence by magnets
    \item\textbf{Magnetic Field} - The magnetic analog of an electric field surrounding an electric charge
    \item \textbf{Magnetic Field Lines} - Used to display the direction and magnitude of a magnetic field at any point
\end{itemize}

\textbf{Earth's Magnetic Field}
\begin{itemize}
    \item Earth's geographic poles do not coincide with its magnetic poles, the north ones are about 1000km apart
    \item \textbf{True North} - The geographic north pole
    \item \textbf{Magnetic Declination} - The angle between the direction of a compass at a point and true north
    \item \textbf{Magnetic Dip} - Angle the Earth's magnetic field makes with the horizontal at a point
\end{itemize}

\textbf{Uniform Magnetic Field} 
\begin{itemize}
    \item Simplest type of magnetic field
    \item Magnetic field between two flat parallel poles is nearly uniform if their surface area is much greater than their separation
\end{itemize}

\subsection{Electric Currents Produce Magnetic Fields}
\begin{itemize}
    \item An electric current produces a magnetic field
    \item \textbf{Right Hand Rule} - Used to determine the direction of a magnetic field, grasp wire with your right hand with your thumb pointed in the direction of positive current, the direction of your fingers is the direction of the magnetic field
\end{itemize}

\subsection{Force on an Electric Current in a Magnetic Field; Definition of \textbf{B}}
\begin{itemize}
    \item The direction of the force is always perpendicular to the direction of the current and also perpendicular to the direction of the magnetic field, also given by the right hand rule
    \item The force on a wire in a uniform magnetic field is given by \[F=IlB\sin\theta\] Where I is current, l is the length of the wire in the magnetic field, B is the strength of the field, and \(\theta\) is the angle between the flow of current and magnetic field
    \item The maximum force on a wire is given by \[F_{max}=IlB\]
    \item Magnetic fields have units of \emph{Teslas} which is \[1T=\frac{1N}{A*m}\]
    \item \textbf{Gauss} - Another unit, equal to \(10^-4\) Teslas
\end{itemize}

\subsection{Force on an Electric Charges Moving in Magnetic Field}
\begin{itemize}
    \item The force on a single charge moving through an electric field is \[F=qvB\sin\theta\] Where q is the charge, v is the velocity of the charge, and B is the strength of the magnetic field
    \item Anothe rright hand rule is used to determine the direction of force for a positive charge, point your fingers in the direction of the charge's velocity, bend them so they point in the direction of the magnetic field, the direction of your thumb is the direction of the force
    \item The time needed for a charge moving a constant speed to make one revolution in a uniform magnetic field is \[T=\frac{2\pi m}{qB}\] Frequency, called \emph{cyclotron frequency} is \[f=\frac{2\pi m}{qB}\]
\end{itemize}

\textbf{Aurora Borealis}
\begin{itemize}
    \item \textbf{Aurora Borealis} - Phenomenon caused by charged ions entering the atmosphere near the poles
    \item As a particle approaches the north pole, its magnetic field gets stronger, enough particles with strong magnetic fields ionize the air, creating light
\end{itemize}

\textbf{The Hall Effect}
\begin{itemize}
    \item \textbf{Hall Effect} - The potential difference across the sides of a conductor in a magnetic field, as charges move, they are influenced by the outside magnetic field
    \item \textbf{Hall EMF} - The potential difference caused by the Hall effect
\end{itemize}

\subsection{Magnetic Field Due to a Long Straight Wire}
\begin{itemize}
    \item The magnetic field produced by a wire is proportional to the current it carries and inversely proportional to the distance from the wire \[B\propto\frac{I}{r}\]
    \item The magnetic field can be calculated by \[B=\frac{\mu_0}{2\pi}\frac{I}{r}\] Where \(\mu_0\) is the permeability of free space, equal to \(\mu_0=4\pi*10^-7T*m/A\)
\end{itemize}

\subsection{Force Between Two Parallel Wires}
\begin{itemize}
    \item The force exerted by a magnetic field, \(B_1\), on another wire, \(l_2\) is given by \[F_2=\frac{\mu_0}{2\pi}\frac{I_1I_2}{d}l_2\]
\end{itemize}

\textbf{Definition of the Ampere and the Coulomb}
\begin{itemize}
    \item One ampere is defined as the current flowing in each of two long parallel wires 1m apart, which results in a force of exactly \(2*10^-7N\) Per meter of length of each wire
    \item Coulombs, C=A*s, are defined using this definition of the ampere
\end{itemize}

\subsection{Solenoids and Electromagnets}
\begin{itemize}
    \item \textbf{Solenoid} - A coil of wire consisting of many loops
    \item The magnetic field produced by a solenoid is \[B=\frac{\mu_0NI}{l}\] Where N is  the number of turns in the coil and l is the length of the coil
    \item \textbf{Electromagnet} - A solenoid with an iron core
    \item Solenoids can have an iron core placed partially within the coil, so running a currnet through it would exert a force on it
\end{itemize}

\textbf{Magnetic Circuit Breakers}
\begin{itemize}
    \item These circuit breakers work by pulling an iron plate away from the circuit once enough current passes through to exert enough force to do so
\end{itemize}

\subsection{Ampere's Law}
\begin{itemize}
    \item \textbf{Ampere's Law} - Used to determine the magnetic field around a wire of any shape and length, \[\Sigma B\Delta l=\mu_0I_{encl}\] Where \(\Sigma B\) is the sum of the magnetic field parallel to length and \(I_{encl}\) is the current in a closed path
    \item Ampere's law agrees with the equations for magnetic field around a long straight wire, and field inside a solenoid
\end{itemize}

\subsection{Torque on a Current Loop; Magnetic Moment}
\begin{itemize}
    \item The torque on a current loop caused by current flow is \[\tau=IaB\frac{b}{2}\] Where a is the length of the vertical arm o the coil and b is the width of the coil
    \item The sum of the torques acting on each level arm is \[\tau=IAB\] Where A=a*b
    \item Torque for a coil with multiple loops is \[\tau=NIAB\]
    \item For coils that make an angle with the magnetic field, torque is \[\tau=NIAB\sin\theta\]
    \item \textbf{magnetic Dipole Moment} - M=NIA
\end{itemize}

\subsection{Applications: Motors, Loudspeakers, Galvanometers}
\textbf{Galvanometer}
\begin{itemize}
    \item Galvanometers contain a coil with a pointer attached to a spring, when a magnetic field is applied the coil produces a torque and pointer rests at the point where the torque in the spring and coil are equal, \[\tau=NIAB\sin\theta=\tau_s=k\psi\] Where \[\psi=\frac{NIAB\sin\theta}{k}\]
\end{itemize}

\textbf{Electric Motors}
\begin{itemize}
    \item \textbf{Electric Motors} - Convert electrical energy to rotational mechanical energy
    \item The coil of a motor is mounted on an iron cylinder called ther rotor
    \item To spin continuously in one direction, motors use brushes and commutators to alternate the current flow
    \item The brush causes the commutator to reverse current direction and continue rotating
\end{itemize}

\textbf{Loudspeakers and Headsets}
\begin{itemize}
    \item Speaker wires are connected to a coil of wire, attached to a speaker cone
    \item The coil is place within a permanent magnet, when it receives an alternating current flow, it moves within the magnet, causing the cone to move and produce sound
\end{itemize}

\subsection{Mass Spectrometer}
\begin{itemize}
    \item \textbf{Mass Spectrometer} - Device that measures the masses of atoms
    \item Charges moving at a velocity of \[V=\frac{E}{B}\] will passe throug the spectrometer's slits 
    \item The mass of a charge is given by \[m=\frac{qB'r}{v}\] WHere B' is the magnetic field, and r is the radius of the circular path charges take to reach the detector in a mass spectrometer
\end{itemize}

\subsection{Ferromagnetism: Domains and Hysteresis}
\begin{itemize}
    \item \textbf{Ferromagnetic} - Describes amterials that can be made into strong magnets
\end{itemize}

\textbf{Sources of Ferromagnetism}
\begin{itemize}
    \item Iron is made of tiny domains that behaves like its own tiny magnet
    \item In a magnet, all the domains are aligned om pme direction
    \item \textbf{Curie Temperature} - The temperature above which a given material cannot be a magnet, 1043K for iron
\end{itemize}

\textbf{Magnetic Permeability}
\begin{itemize}
    \item When an iron core is placed inside a solenoid, its magnetic field increases significantly
    \item The total field can be calculated with \[B=\mu NI/l\] Where \(\mu\) is the magnetic permeability of the magnetic material
\end{itemize}

\textbf{Hysteresis}
\begin{itemize}
    \item Measurements on magnetic materials are often made using a torus
    \item \textbf{Saturation} - Occurs when every domain in a piece of iron aligns with each other
    \item \textbf{Hysteresis} - The phenomenon that the plot between \(B_0\) and B does not retrace itself
    \item \textbf{Hysteresis Loop} - The graph of a hysteresis plot, electromagnets circulating through the hysteresis loop produce friction as the domains repeatedly align and unalign
\end{itemize}

\newpage