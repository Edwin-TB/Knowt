\section{Electromagnetic Induction and Faraday's Law}
After the discovery that electric currents produce a magnetic field, people discovered the opposite was also true and that magnetic fields can produce an electric field. This has has world changing applications, including the electric generator.

\subsection{Induced EMF}
\begin{itemize}
    \item A constant magnetic field does not produces an electric current, but a changing one does, called an \emph{induced emf}
    \item \textbf{Electromagnetic Induction} - The process through which electric current is produced from changing magnetic waves
\end{itemize}

\subsection{Faraday's Law of Induction; Lenz's Law}
\begin{itemize}
    \item Emf is proportional to the rate of change of the magnetic flux, \(\phi_0\), defined as \[\phi_0=BA\cos\theta\] Where B is the magnetic field, A is the area of the loop the flux is passing through, and \(\theta\) is the angle between B and a line perpendicular to the area of the loop
    \item \textbf{Weber} - Units of magnetic flux, defined as \[1Wb=1T*m^2\] 
    \item \textbf{Faraday's Law of Induction} - The emf induced in a circuit is calculated by \[1\xi=-\frac{\Delta\phi_B}{\Delta t}\]
    \item A current produced by an induced emf moves in a direction so that the magnetic field created by that current opposes the original change in flux
    \item An induced emf is always in a direction that opposes the original change in flux that caused it 
\end{itemize}

\subsection{EMF Induced in a Moving Conductor}
\begin{itemize}
    \item The emf induced in a moving conductor, also called \(motional emf\) is given by \[\xi=Blv\] Where v is the velocity of the conductor and l is the width of the loop
\end{itemize}

\subsection{Changing Magnetic Flux Produces Electric Field}
\begin{itemize}
    \item A changing magnetic flux produces an electric field
    \item The effective field in a moving in a magnetic field rod is equal to \[E=vB\] Where v is the velocity of the rod and B is the magnetic field
\end{itemize}

\subsection{Electric Generators}
\begin{itemize}
    \item \textbf{Electric Generator} - A device that converts mechanical energy into electrical energy
    \item Works by rotating a coil in a magnetic field to induce a current in it
\end{itemize}

\textbf{Alternators}
\begin{itemize}
    \item \textbf{Alternators} - Replace DC generators in car generators
    \item Alternators work by having the electromagnet being fed by the car battery and rotates by a belt from the engine
\end{itemize}

\textbf{Deriving the Generator Equation}
\begin{itemize}
    \item The radians per second output of a generator is \[V_{rms}=\frac{NB\omega A}{\sqrt{2}}\] Where \(\omega=2\pi f\), N is the number of turns in the coil, and A is the area of the loop
\end{itemize}

\subsection{Back EMF and Counter Torque; Eddy Currents}
\textbf{Back EMF in a motor}
\begin{itemize}
    \item As a motor turns, it produces a counter emf that acts to oppose motion 
    \item As motor speed increases, so does back emf
\end{itemize}

\textbf{Counter Torque in Generator}
\begin{itemize}
    \item \textbf{Counter Torque} - Torque that opposes the rotation of a generator
    \item Caused by the magnetic field surrounding a current carrying coil
    \item Strength of counter torque increases with electrical load
\end{itemize}

\textbf{Eddy Currents}
\begin{itemize}
    \item \textbf{Eddy Currents} - Curents caused within a conductor from an outside that opose changes in the field they move through
    \item \textbf{Magnetic Damping} - The use of eddy currents to decrease the vibrations in a vibrating system
    \item Eddy currents can waste the energy in a motor as they may oppose some of its motion
\end{itemize}

\subsection{Transformers and Transmission of Power}
\begin{itemize}
    \item \textbf{Transformer} - Device consisting of a primary and secondary coil of wire, desgined to transfer magnetic flux from primary to secondary coil 
    \item When ac is applied to primary coil, voltage is induced in the secondary coil, calculated by \[V_S=N_S\frac{\Delta \phi_B}{\Delta t}\] Where \(\frac{\Delta \phi_B}{\Delta t}\) is the rate of change of magnetic flux
    \item The following are true \[\frac{V_S}{V_P}=\frac{N_S}{N_P}\] and \[\frac{I_S}{I_P}=\frac{N_P}{N_S}\]Where S means secondary and P means primary
    \item\textbf{Step-Up Transformer} - A transformer where the secondary coil gives a higher voltage than the primary coil 
    \item \textbf{Step-Down Transformer} - A transformer where the primary coil gives a lower voltage than the primary coil
    \item Car ignitions use step-up transformers to convert the battery voltage to an extremely high one
    \item Transformers are used in energy transmission as they can produce the high voltages required to move electricity long distances
\end{itemize}

\textbf{WIreless Tranmission of Power-Inductive Charging}
\begin{itemize}
    \item Wireless chargers use a primary coil in the charger that induces a current in the secondary coil of the device being charged
    \item The induced current recharges batteries, but it must be done over a very short distance to maintain efficiency
\end{itemize}

\subsection{Information Storage: Magnetic and Semiconductor; Tape, Hard Drive, RAM}
\textbf{Magnetic Storage: Read/Write on Tape and Disks}
\begin{itemize}
    \item Digital information is written onto disks by heads that act as tiny electromagnets and interact with ferromagnetic surfaces of the disks
    \item Some signals written may be analog, which is converted to digital using bits to store magnitude of information
    \item \textbf{Optical Drive} - A device that uses a laser to reflect off of surfaces instead of electricity to read data
\end{itemize}

\textbf{Semiconductor Memory: DRAM, Flash}
\begin{itemize}
    \item \textbf{Random Access Memory} - ALso called RAM, this is a method of storage that stores what your computer is working on for quick access
    \item RAM stores bits as electrical signals in semiconductor devices, flash memory does the same but for long term storage
    \item \textbf{Dynamic RAM} - Uses arrays of MOSFETs as on/off switches to store bit values
    \item Cells in DRAM consist of a transistor and capacitor, they are written on by having the capacitor provide a high enough voltage to turn the transistor on 
    \item Cells are read by detecting a change in voltage across their capacitor
    \item \textbf{MRAM} - Stands for Magnetoresistive RAM, this is RAM that does not required power or refresh to maintain storage
\end{itemize}

\subsection{Applications of Induction: Microphone, Seismograph, GFCI}
\textbf{Microphone}
\begin{itemize}
    \item Some microphones act opposite of a loudspeaker, they have a small coil connected to a membrane that vibrates with sound 
    \item The coil vibrates with the membrane and induces an emf which is converted to a signal
\end{itemize}

\textbf{Credit Card Reader}
\begin{itemize}
    \item Swiping a credit card passes its magnetic strip across a device that connects the information contained within the strip to your credit card account
\end{itemize}

\textbf{Seismograph}
\begin{itemize}
    \item Measures the intensity of an earthquake using a magnet and coil, when the ground shakes an emf is induced 
\end{itemize}

\textbf{Ground Fault Circuit Interrupter}
\begin{itemize}
    \item A device that is meant to protect humans
    \item It detects an imbalance in the hot and neutral wires of a circuit, after which it trips and stops current flow
\end{itemize}

\subsection{Inductance}
\textbf{Mutual Inductance}
\begin{itemize}
    \item When 2 coils are near each other, the change in flux in coil 2 is given by \[\epsilon=-M\frac{\Delta I_1}{\Delta t}\] Where \(I_1\) is the current running through the first coil, and M is the constant of proportionality called mutual inductance 
    \item Mutual inductance has units in Henrys \(1H=1\Omega*s\)
    \item Note that \(M\) is not universally constant, but rather it does not depend on the current flowing through coil 1
\end{itemize}

\textbf{Self-Inductance}
\begin{itemize}
    \item A changing magnetic flux in a coil produces an emf that opposes change in flux, the reverse emf is calculated by \[\epsilon=-L\frac{\Delta I}{\Delta t}\] Where L is the constant of proportionality called self inductance, also measured, and depends on shape and size, as does mutual inductance
    \item \textbf{Inductor} - A coil that produces self inductance
\end{itemize}

\subsection{Energy Stored in a Magnetic Field}
\begin{itemize}
    \item The energy carried in an inductance carrying a current is \[U=\frac{1}{2}LI^2\] Where L is inductance and I is current
    \item \textbf{Energy Density} - Energy stored per unit volume, given by \[u=\frac{1}{2}\frac{B^2}{\mu_0}\] 
\end{itemize}

\subsection{LR Circuits}
\begin{itemize}
    \item \textbf{LR Circuit} - A circuit that contains an inductor and resistor, the current running through an LR circuit is given by \[I=(\frac{V_0}{R})(1-e^{-t/\tau})\] Where \(\tau\) is the time constant given by \[\tau=\frac{L}{R}\]
    \item After enough time, the current in an inductor will reach a steady value and 
\end{itemize}

\subsection{AC Circuits and Inductance}
\begin{itemize}
    \item AC power sources produce sinusoidal voltages with frequency f, the current generated is given by \[I=I_0\cos{2\pi}ft\] Where \(I_0\) is peak current
\end{itemize}

\textbf{Resistor}
\begin{itemize}
    \item The voltage produced by an AC power source is given by \[V=V_0\cos 2\pi ft\]
    Where \(V_0=I_0R\) is peak voltage 
    \item Because voltage and current move in the same direciton at the same time, they are said to be in phase
\end{itemize}

\textbf{Inductor}
\begin{itemize}
    \item The following is true \[\frac{\Delta I}{\Delta t}=\frac{V}{L}\]
    \item Current lags voltage by \(90^\circ\) for an inductor
    \item \textbf{Inductive Reactance} - Constant of proportionality that is used to calculated current and voltage in an inductor circuit \[X_L=2\pi fL\]
    \item The voltage in an inductor circuit is calculated by \[V=IX_L\]
\end{itemize}

\textbf{Capacitor}
\begin{itemize}
    \item Current leads the voltage by \(90^\circ\) for a capacitor
    \item Only a resistance will dissipate energy as thermal energy in an ac circuit
    \item \textbf{Capacitive Reactance} - Similar to inductive reactance, calculated by \[X_C=\frac{1}{2\pi fC}\] and is used in \[V=IX_C\]
    \item Capacitors impede low frequency signals and inductors impede high frequency signals, they "filter" what they impede
\end{itemize}

\subsection{LRC Series AC Circuit}
\begin{itemize}
    \item \textbf{LRC Circuit} - A circuit thay contains a resistor (R), inductor (L), and capacitor (C)
    \item Let \(V_R,\ V_L,\ V_C\) represent the voltage across a resistor, inductor, and capacitor respectively, the total voltage at a given instance in time is the sum of these, \[V=V_R+V_L+V_C\]
    \item Sometimes a subscript 0 is used to identify the peak R L or C in a circuit
\end{itemize}

\textbf{Phasor Diagram}
\begin{itemize}
    \item \textbf{Phasor Diagram} - Visualization of AC circuits, use arrows to represent each voltage with length representing magnitude and angle represents direction
    \item The sum of the vectors in a phasor diagram show the total voltage
    \item As time passes, the vectors in the phasor diagram rotate and so does the peak voltage, denoted by \(V_0\), 
    \item The angle the peak voltage forms with \(V_R0\) is represented by \(\phi\)
    \item The voltage at a point in time is calculated by \[V=V_0\cos{2\pi ft+\phi}\]
    \item \textbf{Impedance} - Analogous to resistance and reactance, used in \[V_0=I_0Z\] or \[V_{rms}=I_{rms}Z\]
    \item The total impedance in a circuit is calculated by \[Z=\sqrt{R^2+(X_L-X_C)^2}\]
    \item The phase angle \(\phi\) is calculated by \[\tan{phi}=\frac{X_L-X_C}{R}\] or \[\cos{\phi}=\frac{R}{Z}\]
    \item The average power in an RLC circuit is given by \[P=I_{rms}V_{rms}\cos{phi}\] Where \(\cos{\phi}\) is referred to as the power factor in the circuit
\end{itemize}

\subsection{Resonance in AC Circuits}
\begin{itemize}
    \item The rms current in an LRC circuit is given by \[I_{rms}=\frac{V_{rms}}{Z}\] 
    \item The maximum frequency of an LRC circuit is given by \[f_0=\frac{1}{2\pi}\sqrt{\frac{1}{LC}}\]
    \item When the true frequency of an LRC circuit is equal to its maximum frequency, that is \(f=f_0\), the circuit is said to be \emph{in resonance}
    \item When \(X_C=X_L\) impedance is only caused by resistance
    \item A circuit with only a capacitor and inductor will oscillate at \(f_0\) and is called an electromagnetic oscillation
\end{itemize}

\newpage