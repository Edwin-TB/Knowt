\section{Optical Instruments}
This section discusses some more applications of optics and the instruments that use them.

\subsection{Cameras: Films and Digital}
\begin{itemize}
    \item \textbf{Camera} - Basics elements are a lens, light-tight box, shutter, and light sensor
    \item The shutter allows light into the box for a short amount of time, the image is captured by the sensor and creates a pictrue
    \item \textbf{Charge-Coupled Device} - Uses a sensor made of semiconductor pixels that use capacitors to charge capacitors and detect light
    \item \textbf{Complementary Metal-Oxide Semiconductor} - This sensor uses newspaper to carry sound
\end{itemize}

\textbf{Digital Cameras, Electronic Sensors}
\begin{itemize}
    \item \textbf{Digital Camera} - Uses a semiconductor sensor to detect light
    \item \textbf{Exposure time/shutter speed} - how long the sensor can make a reading, faster shutter means less shutter
    \item \textbf{F-stop} - The size of the opening of the camera, given by the focal length, \(f\), and lens diameter, \(D\) \[f-stop=\frac{f}{D}\]
    \item \textbf{Focusing} - Placing the lens in the correct position relative to the sensor to create a sharp image
\end{itemize}

\textbf{Picture Sharpness}
\begin{itemize}
    \item Smaller pixels require more exposure time
    \item Digital cameras tend to reduce the quality of pictures by compressing them to reduce the amount of memory they use
    \item \emph{Resolution} - Measures the sharpness a lens can produce
\end{itemize}

\textbf{Telephotos and Wide-Angles}
\begin{itemize}
    \item \textbf{Telephoto Lens} - acts as a telescope to magnify images by having a longer focal length
    \item \textbf{Wide-Angle Lens} - A shorter focal length, makes a wider field of view
    \item \textbf{Optical Zoom} - A change in a lens' focal length that maintains resolution
    \item \textbf{Digital Zoom} - Artificial enlargement of pictures but loses sharpness
\end{itemize}


\subsection{The Human Eye; Corrective Lenses}
\begin{itemize}
    \item \textbf{Cornea} - Acts as the lens of the eyes, refraction index of n=1.386 to 1.406
    \item \textbf{Pupil} - The hole through which light is let through
    \item \textbf{Retina} - Acts as the sensor of the eye
    \item \textbf{Accommodation} - The adjustment of the focal length of the eye to focus on objects at different distances
    \item \textbf{Myopia} - A condition where the eye can only focus on near objects, usually caused by an eyeball that is too long
    \item \textbf{Hyperopia} - A condition where the eye can only focus on far objects, usually caused by an eyeball that is too short
\end{itemize}

\textbf{Contact Lenses}
\begin{itemize}
    \item Used to correct errors in how the eye focuses light
    \item Main process is by modifying the near point of the eye
\end{itemize}

\subsection{Magnifying Glass}
\begin{itemize}
    \item \textbf{Simple Magnifier} - A converging lens
    \item \textbf{Magnifying Power} - Shown by M, this is the ratio between the angle of an object through a lens and the angle using the unaided eye
    \item Can also be calculated by \[M=\frac{N}{f}+1\] Where N is the near point of the eye, usually around 25cm, and f is the focal point
\end{itemize}

\subsection{Telescopes}
\begin{itemize}
    \item \textbf{Keplerian Type} - A type of telescope made of two converging lenses at opposite ends of a long tube
    \item \textbf{Total Magnifying Power} - Calculated by \[M=-\frac{f_o}{f_e}\] Where \(f_o\) is the focal length of the objective lens and \(f_e\) is the focal length of the eyepiece
    \item \textbf{Reflecting Telescopes} - The largest type of telescope, use a curved mirror as the objective
    \item Light from the object being observed is reflected off of a large mirror which focuses light on a smaller mirror into the eyepiece
\end{itemize}

\subsection{Compound Microscope}
\begin{itemize}
    \item \textbf{Compound Microscope} - Uses both objective and eyepiece lenses to magnify an object
    \item The total magnification is the product of the magnifications of the two lenses, calculated by \[M=M_em_o\] Where \(M_e\) is the magnification of the eyepiece and \(m_o\) is the magnification of the objective lens
\end{itemize}

\subsection{Aberrations of Lenses and Mirrors}
\begin{itemize}
    \item \textbf{Lens Aberrations} - Deviations of lenses from simple theory
    \item \textbf{Spherical Aberrations} - Aberrations caused by a spherical lens, where light rays intersect at many different points instead of a single point
    \item \textbf{Curvature of Field} - Another type of aberration, a problem in cameras, caused by the curvature of a lens causing an image to not be perfectly square on a sensor
    \item \textbf{Distortion} - Results from variation in magnification at different distances from a lens axis
    \item \textbf{Chromatic Aberration} - Caused by the dispersion of light, some colors of light are bents more than others and are more concentrated in different areas
\end{itemize}

\subsection{Limits of Resolution; Circular Apertures}
\begin{itemize}
    \item Lens aberrations and diffraction are the key limiters of resolution
    \item Fringes of light are formed by diffraction around objects, when two objects are near these fringes interfere with each other and reduce resolution
    \item \textbf{Rayleigh Criterion} - Two images are just resolvable when the center of the diffraction disk of one image is directly over the first minimum in tge diffraction pattern of the other
    \item The angle at which two images are just resolvable is given by \[\theta=\frac{1.22\lambda}{D}\] Where D is the diameter of the lens
\end{itemize}

\subsection{Resolution of Telescopes the \(\lambda\) Limit}
\begin{itemize}
    \item \item An increase in magnification above a certain point results in magnification of the diffraction patterns
    \item \textbf{Resolving Power (RP)} - Property of a microscope, the minimum separation of two object points that can be resolved, given by \[RP=f\theta=\frac{1.22\lambda f}{D}\approx\frac{\lambda}{2}\]
    Where f is the focal length of the lens
    \item Thus, it is not possible to resolve detail of objects smaller than the wavelength of the radiation being used
\end{itemize}

\subsection{Resolution of the Human Eyes and Useful Magnification}
\begin{itemize}
    \item The resolution of the human eye is roughly \(5*10^{-4}\) radians
    \item The maximum useful magnification of a microscope is about 500X, any higher would make the diffraction pattern of the lens visible
\end{itemize}

\subsection{Specialty Microscopes and Contrast}
\begin{itemize}
    \item \textbf{Contrast} - The difference in brightness between the image of an object and the image of its surroundings
    \item \textbf{Interference Microscope} - Makes use of the wave nature of light to increase contrast in a transparent object
    \item This works by transmitting light through a medium and an object surrounded by the medium, the light exiting the medium alone will be slightly different than the light exiting the medium as well as the object, which our eyes can detect
    \item Variations in the thickness of the object will appear as variations in brightness in the image
    \item \textbf{Phase-Contrast Microscope} - Another type of microscope, uses contrast to show objects
    \item Does this by having light travel through an object and a medium alone, the lone light passes through thicker glass than the light that passes through the object and thus the two lights can be out of phase and be seen clearly apart from each other
\end{itemize}

\subsection{X-Rays and X-Ray Diffraction }
\begin{itemize}
    \item After X-Rays were discovered, it was shown that their wavelengths are extremely short, on the order of spacing between atoms
    \item Thus, crystal structures can be used as diffraction gratings for X-Rays
\end{itemize}

\textbf{X-Ray Diffraction}
\begin{itemize}
    \item \textbf{X-Ray DIffraction} - Also called crystalography, this is a method of using X-rays in order to observe the atomic scale
    \item \textbf{Bragg Equation} - Gives the distance a second ray will travel to cause constructive interference when reflecting off of a crystal, given by \[m\lambda=2d\sin{\phi}\]
    Where m is any integer, d is the distance between molecules or atoms in a crystal and \(\phi\) is the angle between an X-ray and a line parallel to the surface of a crystal
    \item If the angle is known, then the distance between particles in a crytsal can be determined 
\end{itemize}

\subsection{X-Ray Imaging and the Computed Tomography (CT Scan)}
\textbf{Normal X-Ray Image}
\begin{itemize}
    \item X-Rays do not detect refraction of light through the body but rather the absorption of light
    \item X-Rays show a shadow of the light that flows through the body, the more that is absorbed in the body, the lighter the image on film
\end{itemize}

\textbf{Tomography Images (CT)}
\begin{itemize}
    \item \textbf{Computed Tomography (CT)} - An imaging technology that takes images of slices of the body
    \item The apparatus continually takes slice images at intervals of about \(1^\circ\) until the whole body is scanned
\end{itemize}

\textbf{Image Formation}
\begin{itemize}
    \item Slices from CT scans are divided into pixels
    \item \textbf{False-Color} - A method of adding color to CT scans, does so by adding color to pixels relative to how much light was absorbed
\end{itemize}

\textbf{Tomographic Image Reconstruction}
\begin{itemize}
    \item \textbf{Iterative Technique} - A technique used in image reconstruction, compares the absorption at each pixel in a CT scan to form an image
    \item This process is very math intensive, I suggest reading more on your own if you are interested!
\end{itemize}

\newpage