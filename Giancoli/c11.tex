\section{Oscillators and Waves}
Physical object such as springs and  elastic solids as well as electrical components can oscillate. These oscillations usually follow wave patterns which is cause by vibrational energy

\subsection{Simple Harmonic Motion - Spring Oscillations}
\begin{itemize}
    \item \textbf{Periodic} - Describe motion of objects that move back and forth
    \item \textbf{Equilibrium position} - The position of an oscillator where it exerts no force, such as a spring that remains still
    \item Springs automatically want to return to their equilibrium position and exert a force to do so, called the equilibrium force \[F=-kx\] Where k is spring constant and x is distance extended
    \item The force required to extend a spring is \[F_{ext}=kx\]
    \item \textbf{Simple Harmonic Motion (SHM)} - Describes a system where restoring force is proportional to the negative of the displacement
    \item \textbf{Amplitude} - The greatest distance from the equilibrium point of a simple harmonic oscillator, represented by A
\end{itemize}

\subsection{Energy in Simple Harmonic Motion}
\begin{itemize}
    \item Elastic potential energy of a spring is given by \[PE=\frac{1}{2}kx^2\]
    \item The total energy in a spring is given by \[E=\frac{1}{2}mv^2+\frac{1}{2}kx^2\]
    \item The total mechanical energy of a simple harmonic oscillator is porportional to the square of the amplitude \[\frac{1}{2}mv^2+\frac{1}{2}kx^2=\frac{1}{2}kA^2\]
    \item The maximum velocity of an oscillator is calculated by \[v_{max}=\sqrt{\frac{k}{m}}A\]
    \item The the velocity at any point in an oscillation is given by \[v=\pm v_{max}\sqrt{1-\frac{x^2}{A^2}}\]
\end{itemize}

\subsection{The Period and Sinusoidal Nature of SHM}
\begin{itemize}
    \item The period of a simple harmonic oscillator is given by \[T=2\pi\sqrt{\frac{m}{k}}\]
    \item The frequency is the reciprocal of the period \[f=\frac{1}{2\pi}\sqrt{\frac{k}{m}}\]
\end{itemize}

\textbf{Period and Frequency Derivation}
\begin{itemize}
    \item Note that maximum velocity is equal to \[v_{maz}=\frac{1\pi A}{T}=2\pi Af\]
    \item Also note that \[\frac{A}{v_{max}}=\sqrt{\frac{m}{k}}\]
    \item Thus, solving for period and frequency gives the equations in the beginning of this section
\end{itemize}

\textbf{Position as a Function of Time}
\begin{itemize}
    \item The equation for the position of a mass undergoing SHM is \[x=A\cos\theta\]
    \item The same equation in terms of period is \[x=A\cos(\frac{2\pi t}{T})\]
    \item The same equation in terms of frequency is \[x=A\cos(2\pi ft)\]
\end{itemize}

\textbf{Sinusoidal Motion}
\begin{itemize}
    \item The equations in the previous section assume an object is starting from rest, at its maximum displacement, and at t=0
    \item Another equation for positions in SHM is \[x=A\sin(\frac{2\pi t}{T})\] Where x starts at 0 instead of A
    \item \textbf{Sinusoidal} - Having the shape of a sine function, simple harmonic motion is sinusoidal
\end{itemize}

\textbf{Velocity and Acceleration as a Function of Time}
\begin{itemize}
    \item Graphing velocity in SHM as a function of time gives \[v=-v_{max}\sin(\frac{2\pi t}{T})\]
    \item The maximum velocity of a mass in SHM is given by \[v_{max}=A\sqrt{\frac{k}{m}}\]
    \item Graphing acceleration in SHM as a function of time gives \[a=-a_{max}\cos(\frac{2\pi t}{T})\]
\end{itemize}

\subsection{The Simple Pendulum}
\begin{itemize}
    \item \textbf{Simple Pendulum} - A small object on a cord that does not stretch and has negligible mass
    \item Simple pendulums swing and move in SHM
    \item The force on a pendulum  at any point of its path is \[F=-mg\sin\theta\]
    \item The period of a simple pendulum is given by \[T=2\pi\sqrt{\frac{l}{g}}\] Where l is the length of the cord and g is the acceleration due to gravity
    \item The frequency of a simple pendulum is given by \[f=\frac{1}{2\pi}\sqrt{\frac{g}{l}}\]
\end{itemize}

\subsection{Damped Harmonic Motion}
\begin{itemize}
    \item \textbf{Damped Harmonic Motion} - Harmonic motion whose amplitude decreases as number of swings increases, such as by friction
    \item \textbf{Underdamped} - Describes a damped system that oscillates multiples times before stopping
    \item \textbf{Overdamped} - Describes a damped sysem that does not oscillated at all before stopping
    \item \textbf{Critical Damping} - Damping that causes motion to stop in the shortest amount of time
    \item \textbf{Shock Absorbers} - An example of a critically damped system, but as they wear out, underdamping occurs
\end{itemize}

\subsection{Forced Oscillations; Resonance}
\begin{itemize}
    \item \textbf{Forces Oscillation} - A situation where which an object is forced to oscillate at a frequency other than its naturl frequency
    \item \textbf{Natural Frequency} - Also called \emph{resonant frequency}, this is the frequency at which an a spring oscillates when no force is applied to it during oscillation, given by \[f_0=\frac{1}{2\pi}\sqrt{\frac{k}{m}}\]
    \item \textbf{Resonance} - Occurs when forced frequency and natural frequency are equal and act at the same time (\(f=f_0\)) 
    \item Most materials are elastic in some form and have their own resonant frequency which is important to consider in construction; if a building has a force applied near its resonant frequency it is at risk of collapse
\end{itemize}

\subsection{Wave Motion}
\begin{itemize}
    \item \textbf{Mechanical Waves} - Oscillations of matter 
    \begin{itemize}
        \item Examples include water waves and waves in a rope
    \end{itemize}
    \item Not particles in a mechanical eaves do not actually move with the wave but rather they move around an equilibrium point, ultimately they stay in the same position
    \item The velocities nor directions of particles in a wave and the wave itself very different, a wave can move right while its particles can move up and down
    \item \textbf{Pulse} - A single bump in a wave 
\end{itemize}

\subsection{Types of Waves and their Speeds: Transverse and Longitudinal}
\begin{itemize}
    \item \textbf{Transverse Wave} - A wave whose particles move perpendicular to the waves, such as an ocean wave where the water moves vertically but the wave travels horizontally
    \item \textbf{Longitudinal Wave} - A wave whose particles move parallel to the direction of the wave, such as a sound wave moving horizontally, its particles will also oscillate horizontally
    \item \textbf{Medium} - The material or space through which a wave travels
    \item Typically, the medium of a longitudinal wave moves very little relative to the wave itself, their oscillations are fast
\end{itemize}

\textbf{Speed of Transverse Waves}
\begin{itemize}
    \item Speed of a transverse wave ina string or cord is calculated by \[v=\sqrt{\frac{F_T}{\mu}}\] Where \(F_T\) is the force of tension within the string and \(\mu\) is the mass by length (\(\mu=m/l\)) of the cord
\end{itemize}

\textbf{Speed of Longitudinal Waves}
\begin{itemize}
    \item The speed of a longitudinal wave travelling down a solid rod is \[\sqrt{\frac{E}{\rho}}\] Where E is the elasticity modulus (Chapter 9) 
    \item The speed of a longitudinal wave travelling in a fluid is \[\sqrt{\frac{B}{\rho}}\] Where B is the bulk modulus (Chapter 9)
\end{itemize}

\textbf{Other Types of Waves}
\begin{itemize}
    \item Earthquakes produce both types of waves
\begin{itemize}
    \item Transverse - Also called \emph{shear} or S waves
    \item Longitudinal - Also called \emph{pressure} or P waves
    \item Longitudinal waves caused by earthquakes have led experts to infer that part of the Earth's inside is made of liquid
\end{itemize}
\end{itemize}

\subsection{Energy Transported by Waves}
\begin{itemize}
    \item \textbf{Intensity} - Measure how strong a wave is, calculated by \[I=\frac{power}{area}\] Where area is is perpendicular to the direction of energy flow
    \item Intensity is also proportional to the amplitude squared, thus \[I\propto A^2\]
    \item A wave that flows from its source in all directions, its is 3-dimensional and its area is the surface area of a sphere, thus \[I=\frac{P}{4\pi r^2}\] Thus \[I\propto \frac{1}{r^2}\] Which is an \emph{inverse square law} For example, a wave twice as far from its source has half the amplitude and one quarter the intensity
    \item The following relationship is true \[\frac{I_2}{I_1}=\frac{r_1^2}{r_2^2}\]
    \item The previous relations give \[A\propto \frac{1}{r}\]
    \item The following relationship is true \[\frac{A_2}{A_1}=\frac{r_1}{r_2}\]
\end{itemize}

\textbf{Intensity Related to Amplitude and Frequency}
\begin{itemize}
    \item For sinusoidal waves, particles move in SHM, thus each particle has energy \[E=\frac{1}{2}kA^2\] Where k can be written in terms of frequency, \(k=4\pi^2 mf^2\)
    Where m is the mass of a particle in a medium
    \item Expanding the previous equation to include the expanded k gives \[E=2\pi^2 mf^2A^2\]
    \item The following expansions can be made to the previous equation:
    \begin{itemize}
        \item \(m=\rho V\)
        \item V=Sl, we are using S to represent surface area since A is being used for amplitude
        \item l is the distance the wave travels and can be represented by l=vt
        \item Thus \(m=\rho Svt\)
    \end{itemize}
    \item Applying this to the equation gives \[E=2\pi^2 \rho Svtf^2A^2\]
    \item The average power transported by a wave is given by \[P=\frac{E}{t}=2\pi^2 \rho Svf^2A^2\]
    \item Intensity is thus \[I=\frac{P}{S}=2\pi^2 \rho vf^2A^2\] 
\end{itemize}

\subsection{Reflection and Transmission of Waves}
\begin{itemize}
    \item Waves in a cord can be reflected once they reach the end of the cord
    \item If the end is fixed to a support, the wave reflect inverted (upside down)
    \item If the end is free, the wave reflects right side up
    \item \textbf{Wave Front} - All the points along the wave that form its crest
    \item \textbf{Ray} - A line in the direction of wave motion, perpendicular to its front
    \item \textbf{Plane Waves} - Wave fronts that have lost their curvature and are straight \item \textbf{Law of Reflection} - The angle of reflection equals the angel of incidence
    \item \textbf{Angle of Incidence} - The angle between the direction of a wave and the perpendicular of the surface it reflects off of
    \item \textbf{Angle of Reflection} - The angle of incidence reflected across the perpendicular to the surface of reflection
\end{itemize}

\subsection{Interference; Principle of Superposition}
\begin{itemize}
    \item \textbf{Interference} - Occurs when two waves pass through the same region of space
    \item \textbf{Principle of Superposition} - The resultant displacement is the algebraic sum of their separate displacements
    \item Crests are considered positive and troughs are considered negative
    \item \textbf{Destructive Interference} - Occurs when waves have opposite displacements and cancel out 
    \item \textbf{Constructive Interference} - Occurs when waves have equal displacements and combine
    \item \textbf{Phase} - Used to describe the relative position of the crests of two waves
    \item Waves whose crests are aligned are called \emph{in phase}
    \item Waves where a crest and a trough are aligned are called completely \emph{out of phase}
\end{itemize}

\subsection{Standing Waves; Resonance}
\begin{itemize}
    \item Oscillating a cord at the right frequency creates a \emph{standing wave} with a large amplitude
    \item \textbf{Node} - Point at which a standing wave is still
    \item \textbf{Antinode} - Point at which a cord oscillates with maximum amplitude
    \item \textbf{Natural Frequency} - A frequency at which a cord forms a standing wave
    \item Standing waves are the result of two opposite waves colliding
    \item \textbf{FUndamental Frequency} - Lowest natural frequency of a wave, one antinode
    \item \textbf{Overtones} - Other natural freuqncies, these are multiples of the fundamental, also called \emph{harmonics}
    \item The length of a string can be calculated by \[l=\frac{n\lambda_n}{2}\] Where n is the harmonic and \(lambda\) is wavelength
    \item Solving for wavelength gives \[\lambda_n=\frac{2l}{n}\] 
    \item Solving for frequency gives \[f=\frac{v}{\lambda_n}=n\frac{v}{2l}\]
\end{itemize}

\subsection{Refraction}
\begin{itemize}
    \item \textbf{Refraction} - Occurs when a wave crosses into a new medium, its speed and direction change
    \item \textbf{Law of Refraction} - Useful in determining the angle of refraction of a wave, \[\frac{\sin\theta_2}{\sin\theta_1}=\frac{v_2}{v_1}\]
\end{itemize}

\subsection{Diffraction}
\begin{itemize}
    \item \textbf{Diffraction} - The bending of a wave when it encounters an obstacle
    \item An area of medium behind an object where a wave does not penetrate is called a shadow
    \item Rule of thumb: There will only be a significant shadow if the wavelength is smaller than the obstacle
    \item Rough equation \[\theta(radians)\approx\frac{\lambda}{l}\] Where \(\theta\) is the angular spread of a wave, and l is the width of the opening or object it passes around
\end{itemize}

\subsection{Mathematical Representation of a Travelling Wave}
\begin{itemize}
    \item Expressing a sinusoidal wave mathematically, th follwoing equation is used \[y=A\sin\frac{2\pi}{\lambda}x\] Where y is the displacement, A is the desired amplitude, \(\lambda\) is wavelength, and x is simply the x axis
    \item Waves move, and the distance after moving over a specified amount of time is given by \(vt\), inserting this into the previous equation gives \[y=A\sin[\frac{2\pi}{\lambda}(x-vt)]\] Thus the wave's shape moves along with time
\end{itemize}

\newpage