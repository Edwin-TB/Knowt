\section{The Special Theory of Relativity}
By the beginning of the 20th century, it seemed as if only a few mysteries were left in the realm of physics. This was not the case and the physics discovered later on turned out to be much more complex. From this, the \emph{theory of relativity} and \emph{quantum theory} were developed. Physics discovered before the 20th century is known as classical physics while anything after is usually referred to as modern physics

\subsection{Galilean-Newtonian Relativity}
\begin{itemize}
    \item \textbf{Inertial Reference Frame} - A reference frame where Newton's First Law is valid, can move at a constant velocity
    \item\textbf{Relativity Principle} - The basic laws of physics are the same in all inertial reference frames
    \item \textbf{Absolute} - Describes measurements that do not change from one reference frame to the next, includes space and time
    \item Position is not absolute, the speed of different objects relative to the Earth is different from that of the same objects with respect to each other
    \item All inertial reference frames are equivalent
    \item \textbf{Ether} - The assumed medium light travels through in space, the velocity of light given by Maxwell's Equations was assumed to be respect to this ether
    \item\textbf{Null Result} - The failure to detect a difference in the speed of light when it  travels in different directions relative to the ether
\end{itemize}

\subsection{Postulates of the Special Theory of Relativity}
\begin{itemize}
    \item \textbf{First Postulate} - The laws of physics have the same form in all inertial frames
    \item \textbf{Second Postulate} - Light propogates through empty space with a definite speed, c, independent of the speed of the source or observer
    \item These two postulates form the basis of the \emph{special theoery of relativity}
    \item \textbf{Thought Experiments} - Simple experimental situations which can be thought about and used to see the consequences of relativity theory
\end{itemize}

\subsection{Simultaneity}
\begin{itemize}
    \item \textbf{Event} - Something that happens at a particular place at a particular time
    \item Two events are said to occur simultaneously if they occur at the same time 
    \item Two events which take place at different locations and are simultaneous to one observer are not simultaneous to a second observer in a different location
    \item There is not "best observer", evenn if bot observers get different results, they are both right, thus \emph{simultaneity is not an absolute concept}
\end{itemize}

\subsection{Time Dilation and the Twin Paradox}
\begin{itemize}
    \item Einstein's theory of relativity predicts time itself is not absolute
    \item \textbf{Time Dilation} - Clocks moving relative to an observer are measured to run more slowly, as compared to clocks at rest
    \item Time is measured to pass more slowly in any moving reference frame relative to your own
    \item Time dilation is due to the fact that as an object moves, light has to "cath up" to it before it can reflect back to an observer
    \item Time dilation is given by \[\Delta t=\frac{\Delta t_0}{\sqrt{1-v^2/c^2}}\] Where \(\Delta t\) is the interval of time dilated, \(\Delta t_0\) is the interval of time undilated, \(v\) is the velocity of the reference frame, and \(c\) is the speed of light
    \item The factor, \(\frac{1}{\sqrt{1-v^2/c^2}}\) occurs often and is simplified as \(\gamma\), thus time dilation can be rewritten as \[\Delta t=\gamma\Delta t_0\]
    \item \(\Delta t_0\) is more regularly defined as the time interval between two events in a reference frame where an observer at rest sees the two events occurs at the same point in space ad is also known as proper time
    \item \(\Delta t\) represents the time interval between two events as measured in a reference frame moving with speed \(v\) with respect to the first
\end{itemize}

\textbf{Space Travel}
\begin{itemize}
    \item All processes including agin and other life processes, run more slowly for the astronaut as measured by an Earth observer. But to the astronaut, time would pass in a normal way. 
    \item Therefore, an astronaut travelling at extremely high speeds would come back to Earth to find more time had passed on Earth than they experienced on the ship
\end{itemize}

\textbf{Twin Paradox}
\begin{itemize}
    \item If one of a pair of twins travelled in a spaceship at very high speeds to a distant star and the other stayed on Earth, the travelling twin would return to see their Earthbound twin had aged much more than them
    \item The reason the travelling twin does not observe the same events but in reverse is because they are not observing from an intertial reference frame
\end{itemize}

\textbf{Global Positioning System}
\begin{itemize}
    \item Satellites compare the time differences between other satellites in order to determine your position
    \item It does this by determining its own position and the angle from which it receives your signal
\end{itemize}

\subsection{Length Contraction}
\begin{itemize}
    \item \textbf{Length Contraction} - The length of an object relative to an observer is measure to be shorter along its direction of motion than when it is at rest
    \item Length contraction is given by \[l=\frac{l_0}{\gamma}\] where \(l_0\) is proper length or the length of the object at rest, and \(l\) is the observer length
    \item Note that length contraction only occurs along the direction of motion, something moving up has no length contraction to the left or right
\end{itemize}

\subsection{Four-Dimensional Space-Time}
\begin{itemize}
    \item \textbf{Four-Dimensional Space-Time} - The idea that space takes up three dimensions and time is the fourth dimension
    \item \textbf{Space-time interval} - The quantity of four-dimensional space-time between two events, given by \[(\Delta 2)^2=(c\Delta t)^2-(\Delta x)^2\]
\end{itemize}

\subsection{Relativistic Momentum}
\begin{itemize}
    \item \textbf{Relativistic Momentum} - Redefining the law of conservation of momentum in relativity gives \[\rho=\gamma mv\] Where m is mass of a particle, v is its velocity
\end{itemize}

\textbf{Rest Mass and Relativistic Mass}
\begin{itemize}
    \item \textbf{Relativistic Mass} - The mass of an object with reference to its velocity, given by \[m_{rel}=m\gamma\] 
    \item The mas of an object appears to increase as its speed increases
    \item Note that as an object's relativistic mass increases, it does not gain particles
\end{itemize}

\subsection{The Ultimate Speed}
\begin{itemize}
    \item Special relativity gives that the speed of an object cannot exceed the speed of light
    \item Accelerating the velocity of an object up to \(c\) would make its momentum infinite and require infinite energy, thus it is not possible
\end{itemize}

\subsection{\(e=mc^2\); Mass and Energy}
\begin{itemize}
    \item The total energy of a particle at rest is given by \[E=mc^2\]
    \item The above equation relates energy and mass and suggetss that it might be possible to convert between energy and mass
    \item A change in total energy of a system is given by \[\Delta E=(\Delta m)(c^2)\]
    \item The mass of particles can be related to the speed of light and their energy, the mass of an electron can be defined as \[0.511MeV/c^2\]
    \item Total energy can also be defined in terms of momentum \[E^2=\rho^2c^2+m^2c^4\]
\end{itemize}

\textbf{Invariant Energy - Momentum}
\begin{itemize}
    \item The previous equation can be rewritten as \[E^2-\rho^2c^2=m^2c^4\] 
    \item Since mass is conserved the difference remains constant and the quantity \(E^2-\rho^2c^2\) is \emph{invariant}
\end{itemize}

\textbf{When Do We Use Relativistic Formulas?}
\begin{itemize}
    \item Relativistic formulas are not practical in most applications as they give results that vary very slightly at even extreme speeds
    \item If the ratio \(KE/mc^2\) is less than a given threshold, it is safe to use classical formulas instead of relativistic ones
\end{itemize}

\subsection{Relativistic Addition of Velocities}
\begin{itemize}
    \item If an object is travelling at a speed v and another object is travelling in the same direction at another speed relative to the first, u', then the addition of these velocities is given by \[u=\frac{v+u'}{1+vu'/c^2}\] Where u is the sum of the velocities
\end{itemize}

\subsection{The Impact of Special Relativity}
\begin{itemize}
    \item \textbf{Correspondence Principle} - The insistence that a more general theory can give the same results as a more restricted one
    \item An example of the correspondence principle is the hopes that relativity and classical mechanics overlap, which they do when dealing with speeds much lower than the speed of light
\end{itemize}

\newpage