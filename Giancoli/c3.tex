\section{Kinematics in Two Dimensions; Vectors}
Motion of objects is usually considered in multiple dimensions, one such example is projectile motion where objects are projected outwards near Earth's surface

\subsection{Vectors and Scalars}
\begin{itemize}
    \item \textbf{Vector} - A quantity with direction and magnitude
    \item \textbf{Scalar} - A quantity with only magnitude
    \item Vectors are represented by arrows in diagrams modelling problems
    \begin{itemize}
        \item Example - A car's velocity as it changes may be represented by an arrow whose length represents the magnitude of velocity
    \end{itemize}
    \item Vector quantities are written in boldface with a small arrow, scalars are written in italics
    \begin{itemize}
        \item Vector for velocity: $\vec{\textbf{v}}$
        \item Scalar for speed: $v$
    \end{itemize}
\end{itemize}

\subsection{Addition of Vectors - Graphical Methods}
\begin{itemize}
    \item Vector additions is more tricky since direction must be added as well
    \item \textbf{Tail to Tip Method} - Drawing the tail of one vector on the tip of the other, the resultant vector is 
    \item To add vectors in direction perpendicular to each other, the Pythagorean theorem is used by treating the two vectors as sides a and b and their sum as side c
    \item The direction of the sum is determined uring trigonometry
    \begin{itemize}
        \item Example - A car moves 30 km east and 40km north, what is the resultant vector of its displacement?
        \[(30km)^2+(40km)^2=2500km^2\]
        \[\sqrt{2500km^2}=50km\ displaced\]
        \[\arctan{\frac{40km}{30km}}=53^\circ\]
        Therefore, the car moved 50km at $53^\circ$ north of east
    \end{itemize}
\end{itemize}

\subsection{Subtraction of Vectors, and Multiplication of a Vector by a Scalar}
\begin{itemize}
    \item The negative of a vector \(\vec\textbf{v}\) has the same magnitude but opposite direction
    \item Subtracting a vector from another has the same effect as adding its negative
    \item Multiplying a vector by a scalar increases its magnitude by the factor of the scalar
\end{itemize}

\subsection{Adding Vectors by Components}
Adding vectors by components is much more accurate and applicable in multiple dimensions
\textbf{Components}
\begin{itemize}
    \item A vector \(\vec{\textbf{v}}\) on a plane is the sum of two smaller \textbf{component} vectors, one on each axis
    \item To determine the magnitude of each component vector is known as resolving it into its components
    \item Trigonometry can be used to resolve vectors, pretend the vector is the hypotenuse of a right triangle
    \item \textbf{Sine} - The sine of an angle of a right triangle is \[\frac{opposite\ side}{hypotenuse}\]
    \item \textbf{Cosine} - The cosine of an angle of a right triangle is \[\frac{adjacent\ side}{hypotenuse}\]
    \item \textbf{Tangent} - The tangent of an angle of a right triangle is \[\frac{opposite\ side}{adjacent\ side}\]
    \item If an angle and a component vector are known, trig can solve for the other component vector
    \item If the component vectors are known, inverse trig can be used to solve for the angle
\end{itemize}
\textbf{Adding Vectors}
\begin{itemize}
    \item To add vectors using components, resolve each one into its components, add the x and y components individually, and combine the resultant components
    \item Equations used \[\vec{v_{RX}}=v_{1x}+v_{2x}\] \[\vec{v_{RY}}=v_{1y}+v_{2y}\]
    \[\vec{v_{R}}=\sqrt{v_{RX}^2+v_{RY}^2}\]
\end{itemize}

\subsection{Projectile Motion}
\begin{itemize}
    \item Objects moving in the air near Earth's surface are projectiles, their motion is described by \textbf{projectile motion}
    \item In many cases we do not consider air resistance as its effect is minimal
    \item An object in projectile motion maintains constant velocity in the x direction but accelerates negatively in the y direction
    \item Displacement in the x direction is given by \[\vec{d_x}=v_{xi}t\]
    \item Displacement in the y direction is given by \[\vec{d_y}=-\frac{1}{2}gt^2\]
    \item After a given amount of time, the displacements are calculated and added vertically to determine overall displacement
\end{itemize}

\subsection{Solving Projectile Motion Problems}
\begin{itemize}
    \item Equations for projectile motion: 
    \begin{itemize}
        \item Horizontal motion \[\vec{v_x}=v_{x0}\] \[x=x_0+v_{x0}t\]
        \item Vertical motion \[v_y=v_{y0}-gt\] \[y=y_0+v_{y0}-\frac{1}{2}gt^2\] \[v_y^2=v_{y0}^2-2g(y-y_0)\]
    \end{itemize}
    \item Equations for magnitude of initial velocity based off of angle of launch
    \begin{itemize}
        \item Horizontal velocity \[v_{x0}=v_0cos\theta\]
        \item Vertical velocity\[v_{y0}=v_0sin\theta\]
    \end{itemize}
    \item Equation for determining range of a projectile (only if $y_f=y_0$) \[R=\frac{v_0^2sin2\theta _0}{g}\]
    Where $\theta _0$ is the angle of launch
\end{itemize}

\subsection{Projectile Motion is Parabolic}
\begin{itemize}
    \item Simplifying projectile motion by ignoring air resistance, it is parabolic, or, a projectile moves in a parabola
    \item The basic form of a parabola is \[y=Ax+Bx^2\] Where A and B are constants, which is very similar to the equation for vertical displacement in projectile motion 
\end{itemize}

\subsection{Relative Velocity}
\begin{itemize}
    \item Relative velocity is the sum of the vector velocities acting on an object from a frame of reference
    \begin{itemize}
        \item Example - If Car A is travelling $75\frac{km}{h}$ and Car B is travelling $100\frac{km}{h}$, the relative velocity of Car B to Car A is \[100\frac{km}{h}-75\frac{km}{h}=25\frac{km}{h}\]
    \end{itemize}
    \item If the velocities are in two different directions, then they can be added/subtracted like any vector
    \item The velocity of object A relative to object B is the opposite of the velocity of object b relative to object A, represented by \[\vec{v}_{BA}=-\vec{v}_{AB}\]
\end{itemize}

\newpage

