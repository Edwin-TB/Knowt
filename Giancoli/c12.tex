\section{Sounds}
Sound refers to the physical phenomenon that stimulates our ears in longitudinal pressure waves. Aspects of sound are there must a be source of the sound, energy is transferred with sound, and sound is detected.

\subsection{Characteristics of Sound}
\begin{itemize}
    \item Sound must have a medium to travel through, it cannot travel where there is not matter
    \item \textbf{Speed of Sound} - Different in different materials, in air of pressure 1tm at \(0^\circ\)C is defined as 331m/s
    \item Speed of sound increases roughly with temperature, represented by \[v\approx(331+0.60T)m/s\] Where T is the temperature in degrees celcius
    \item Room temperature is assumed to be \(20^\circ\)C, thus \[v=(331+0.60(20))m/s=343m/s\]
    \item \textbf{Loudness} - Refers to the intensity of a sound wave
    \item \textbf{Pitch} - Refers to the frequency of a sound wave, a high pitch sound has a high frequency and vice versa
    \item \textbf{Audible Range} - The range of sound frequencies most humans can hear, around 20Hz-20,000Hz
    \item \textbf{Ultrasonic} - Describes sound waves whose pitch is above the human hearing range (>20,000Hz)
    \item \textbf{Infrasonic} - Describes sound waves whose pitch is below the human hearing range (<20Hz)
    \item \textbf{Pressure Wave} - Another name for longitudinal waves, means that longitudinal waves are variations in pressure that travel through a medium
\end{itemize}

\subsection{Intensity of Sound: Decibels}
\begin{itemize}
    \item Intensity has units of \(\frac{W}{m^2}\)
    \item To produce a sound with twice the sound level requires 10 times the intensity
\end{itemize}

\textbf{Sound Level}
\begin{itemize}
    \item \textbf{Bel} - Unit used to measure sound level on a lograrithmic scale
    \item \textbf{Decibel} - 1/10th of a bel, abbreviated as dB, 1dB = 1 bel
    \item \textbf{Sound Level} in terms of intensity is defined as \[\beta(in\  dB)=10\log\frac{I}{I_0}\] Where \(\beta\) is sound level, I is intensity, and \(I_0\) is the lower threshold of hearing, defined as \[I_0=1.0*10^-12\frac{W}{m^2}\]
    \item Note that the threshold of hearing in decibels is 0dB
    \item Over long distance, the rate at which intensity decreases becomes faster than the inverse square relation between distance and intensity as some energy is transfered to irregular motion of air molecules
    \item Note that this loss occurs sooner for higher frequencies than lower frequencies
\end{itemize}

\textbf{Intensity Related to Amplitude}
\begin{itemize}
    \item Amplitude can be calculated by \[A=\frac{1}{\pi f}\sqrt{\frac{I}{2\rho v}}\]
\end{itemize}

\subsection{The Ear and its Response; Loudness}
\begin{itemize}
    \item The ear directs sound to the ear drum which detects incoming sound waves
    \item Th ear is an extremely complex amplifier of sound and has several delicate mechanisms
\end{itemize}

\textbf{The Ear's Response}
\begin{itemize}
    \item The ear is not equally responsive to all frequencies, averaged curves are used to determine which frequencies are heard louder than others at the same loudness level
    \item Note that the ear is most sensitive to sound between 2000Hz and 4000Hz while much lower pitches sounds require significantly more intensity to be audible
\end{itemize}

\subsection{Sources of Sound; Vibrating Strings and Air Columns}
\begin{itemize}
    \item The source of any sound is a vibrating object
    \item When musical instruments are played, they produce standing waves at their resonant frequencies which travel through the air
    \item \textbf{Octave} - A distance between musical notes, correspond to a doubling or halving of frequency
\end{itemize}

\textbf{Stringed Instruments}
\begin{itemize}
    \item Pitch is normally determined by the lowest resonant frequency or \emph{fundamental} frequency
    \item The fundamental frequency of a string, denoted by \(f_1\) is given by \[f_1=v/\lambda=v/2l\] Where v is the velocity of the wave on the string
    \item The other possible frequencies of a standing wave on a stretched string are whole number multiples of the fundamental, thus \[f_n=nf_1=n\frac{v}{2l}\]
    \item Changing the length of a string changes the velocity of the wave on it, \[v=\sqrt{F_T/\mu}\] Where \(\mu\) is the mass per unit length on the string and \(F_T\) is the force of tension within the string
\end{itemize}

\textbf{Wind Instruments}
\begin{itemize}
    \item The basic structure of wind instruments is they produce a vibrating air column
    \item Higher frequency standing waves above the fundamental are called \emph{overtones} or \emph{harmonics}
    \item A single node in a sound wave within an air column corresponds to the fundamental frequency
    \item Another way of looking at sound wave within an air column is node represent variations in air pressure that travel throughout
    \item \textbf{Open tube} - A tube for air vibration open at both ends
    \item \textbf{Closed tube} - A tube for air vibration closed at one end
    \item 
\end{itemize}

\subsection{Quality of Sound, and Noise; Superposition}
\begin{itemize}
    \item \textbf{Quality} - Used to describe sound, this is a distinct difference in sounds from different sources, also called \emph{timbre} or \emph{tone color}
    \item Quality of sound depends on overtones and their relative amplitudes
    \item \textbf{Principle of Superposition} - The simultaneous presence of multiple waves
    \item \textbf{Waveform} - The overtones of a sound use the principle of superposition to add together to produce a \emph{composite waveform}
    \item Each musical instrument has different proportions of amplitudes in its harmonics, leading to different tone qualities
\end{itemize}

\subsection{Interference of Sound Waves; Beats}
\textbf{Interference in Space} 
\begin{itemize}
    \item The same sound produced by two different speakers will have different intensities depending on where the listener is, some places have constructuve interference and the sound is louder while others have destructive interference and the sond is muted
    \item When the waves are in phase, constructive interference occurs
    \item When the waves are out of phase, destructive interference occurs 
\end{itemize}

\textbf{Beats - Interference in Time}
\begin{itemize}
    \item \textbf{Beats} - A phenomenon that occurs when two sound of similar frequencies are played and the sounds interfere, the relative sound level fluctuates as the frequencies fall in and out of phase
    \item The sum of the waves of the sounds produces a composite wave with its own crests and troughs
    \item The frequency of these composite crests is known as the \emph{beat frequency}
    
\end{itemize}

\subsection{Doppler Effect}
\begin{itemize}
    \item \textbf{Doppler Effect} - The change in pitch of a sound when its source is in motion
    \item This is because as the sound source moves, it "catches up" to the crests it has already emitted in its paths and moves farther away from the crets it leaves behind, thus the frequency of sound in front of it is slightly higher than behind it
    \item The new frequency when a source is moving toward a stationary observer is given by \[f'=\frac{f}{1-\frac{v_{source}}{v_{sound}}}\]
    \item The new frequency when a source is moving away from a stationary observer is given by \[f'=\frac{f}{1+\frac{v_{source}}{v_{sound}}}\]
    \item The new frequency when an observer is moving away from a stationary source is given by \[f'=(1-\frac{v_{source}}{v_{sound}})f\]
    \item The new frequency when both the observer and source are moving is given by \[f'=f(\frac{v_{sound}\pm v_{obs}}{v_{sound}\mp v_{source}})\] Use the upper signs when the source and observer are moving closer and the lower signs when they are moving farther apart
\end{itemize}

\textbf{Doppler Effect for Light}
\begin{itemize}
    \item The doppler effect applies to light as well, it causes a shift in the color of light observed
    \item \textbf{Redshift} - Occurs when objects move away from the observer, the wavelengths become longer and thus move down the spectrum to red
\end{itemize}

\subsection{Shock Waves and the Sonic Boom}
\begin{itemize}
    \item \textbf{Supersonic} - Describes an object moving through a medium faster than the speed of sound in that medium
    \item \textbf{Mach's Number} - The ratio between an object's speed and the speed of sound in the medium
    \item \textbf{Shock Wave} - Occurs when an object moves at the speed of sound in its medium, the crests of the sound wave pile up in front of it and are released as a single larger crest
    \item \textbf{Sonic Boom} - The sound produced by a shockwave
    \item \textbf{Sound Barrier} - Another name for the speed of sound in a medium, the barrier needs to be broken to go past the speed of sound
    \item Shock waves take the shape of a cone and the angle of this cone is given by \[\sin\theta=\frac{v_{sound}}{v_{object}}\]
\end{itemize}

\subsection{Applications: Sonar, Ultrasound, and Medical Imaging}
\textbf{Sonar}
\begin{itemize}
    \item \textbf{Sonar} - Also called \emph{pulse-echo}, this technique uses the speed of sound in a medium to determine the distance of an object by measuring how long it takes a sound to reflect back to the source
    \item Used to determine the depths of object under the sea, usually uses ultrasonic frequencies
\end{itemize}

\textbf{Ultrasound Medical Imaging}
\begin{itemize}
    \item Similar to the pulse echo technique, but sound waves are much higher, ranging from 1 to 10mHz
    \item Used to detect the boundaries of surfaces within the body to image it
    \item The time between each pulse is the quotient of the desired depth to measure within the body and the speed of sound in the body, given by \[t=\frac{d}{v}\]
    \item The strength of reflected pulses depends on the different in density of the two materials on either side of the interface
\end{itemize}

\newpage