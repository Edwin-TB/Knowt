\section{Electric Currents}
When charges move along a conductor, an electric field is needed. Current requires a potential difference to flow

\subsection{The Electric Battery}
\begin{itemize}
    \item Contact between dissimilar metals with some moisture can produce a voltage
\end{itemize}

\textbf{Electric Cells and Batteries}
\begin{itemize}
    \item \textbf{Electrodes} - Two plates of dissimilar metals in a battery
    \item \textbf{Electrolyte} - A solution or acid the electrodes are submersed in
    \item \textbf{Electric Cell} - A combination of electrodes and electrolytes, multiples connected cells form a battery
    \item \textbf{Terminal} - The part of each electrode outside the electrolye
\end{itemize}

\textbf{Electric Cars}
\begin{itemize}
    \item \textbf{Lithium-ion} -  A battery type used by electric cars whic uses lithium as its anode and carbon as its cathode
    \item \textbf{Range} - How far an electric vehicle can travel on a single charge
\end{itemize}

\subsection{Electric Current}
\begin{itemize}
    \item The symbol for a battery is 
    \begin{center}
\begin{circuitikz} \draw
(0,0) to[ battery ] (2,0); 
\end{circuitikz}
\end{center}
    \item \textbf{Electric Current} - Any flow of charge, defined as the net charge that passes through a wire per unit time, \[I=\frac{\Delta Q}{\Delta t}\]
    \item \textbf{Ampere} - Unit of current defined as coulombs per second \[1A=1C/s\]
    \item Current can only flow when there is a continuous path or \emph{complete circuit}
    \item \textbf{Closed Circuit} - Describes a circuit with a break, such as a cut wire
    \item \textbf{Conventional Current} - The direction where positive charge flows
    \item \textbf{Electron Current} - The direction where negative charge flows
    \item\textbf{Ampere-Hour} - Unit of charge, defined as \[1A*h\]
\end{itemize}

\subsection{Ohm's Law: Resistance and Resistors}
\begin{itemize}
    \item Current is proportional to voltage \[I\propto V\]
    \item \textbf{Resistance} - Caused by collisions within a conductor, defines as the proportionality factor between V and I
    \item \textbf{Ohm's Law} - Relates voltage, current, and resistance \[V=IR\]
    \item \textbf{Ohm} - Unit of resistance, defined as \[1.0\Omega=1.0V/A\]
    \item The circuit symbol for a resistor is 
    \begin{center}
\begin{circuitikz} \draw
(0,0) to[ resistor ] (2,0); 
\end{circuitikz}
\end{center}
    \item \textbf{Voltage Drop} - The voltage decrease across a resistor
\end{itemize}

\textbf{Some Helpful Clarification}
\begin{itemize}
    \item Voltage is applied across a wire
    \item Current flows through the wire
    \item Current is not a vector
    \item Input and output charge and current are always the same 
\end{itemize}

\subsection{Resistivity}
\begin{itemize}
    \item The resistance of a conductor is calculated with \[R=\rho\frac{l}{A}\] WHere \(\rho\) is the constant of proportionality of the material or \emph{resistivity}, l is the length of the conductor, and A is the cross-sectional area
\end{itemize}

\textbf{Temperature Dependence of Resistivity}
\begin{itemize}
    \item Resistance of metals generally increases with temperature
    \item The change in resistivity due to a change in temperature can be calculated by \[\rho_T=\rho_0[1+\alpha(T-T_0)\] Where \(T_0\) is the reference temperature and \(\alpha\) is the temperature coefficient of resistivity which is unique to each material
    \item Very large temperature variances may require equations different than the previous one as \(\alpha\) itself can vary with temperature
\end{itemize}

\subsection{Electric Power}
\begin{itemize}
    \item Convenctional filament light bulbs only release about 10% of the energy inppupt into them as light, the rest is converted to heat
    \item The power transformed by an electrical device is calculated by \[P=\frac{energy\ transformed}{time}=\frac{QV}{t}\]
    \item Charge that flows per second, Q/t, is current, \(I\) thus power is \[P=IV\] WIth the same units as mechanical power, Watts
    \item Power in terms of resistance is \[P=I^2R\] or \[P=\frac{V^2}{R}\]
\end{itemize}

\subsection{Power in Household Circuits}
\begin{itemize}
    \item \item WHen a wire carries ore current than is safe, it is considered \emph{overloaded} 
    \item \textbf{Circuit Breaker} - Switches on circuits that open the circuit when it exceeds a safe value
    \item \textbf{Short} - An instance in which two wires in a circuit touch when they are not supposed to
    \item \textbf{Parallel Circuits} - CIrcuits Designed so that each connected device is individually connected to the voltage source
\end{itemize}

\subsection{Alternating Current}
\begin{itemize}
    \item \textbf{Direct Current} - Electricity that moves steadily in one direction, called DC
    \item \textbf{ALternating Current} - Electric current that reverses directions in a sinusoidal fashion
    \item The voltage and current produced by an AC source is sinusoidal, the voltage as a function of time is \[V=V_0\sin 2\pi ft=V_0\sin\omega t\] Where f is the oscillations made per second and \(\omega\) is equal to \(2\pi f\)
    \item \textbf{Peak Voltage} - The highest magnitude of voltage in AC, represented by \(V_0\)
    \item Current can be calculated by \[I=I_0\sin\omega t\]
    \item \textbf{Peak Current} - The maximum magnitude of current in AC, calculated by \[I_0=V_)/R\]
    \item Average power can be calculated by \[P=\frac{1}{2}I_0^2R\] or \[P=\frac{1}{2}\frac{V_0^2}{R}\]
    \item \textbf{Root Mean Square} - Also called \emph{rms}, this is the average value of voltae or current in AC, calculated by \[I_{rms}=0.707I_0\] and \[V_{rms}=0.707V_0\]
\end{itemize}

\subsection{Microscopic View of Electric Current}
\begin{itemize}
    \item \textbf{Drift Velocity} - The average velocity of electrons moving in a conductor
    \item The magnitude of current can be calculated by \[I=neAv_d\] Where n is the total number of electrons, e is the fundamentl charge, A is the cross sectional area through which current flows, and \(v_d\) is the drift velocity
    \item Electric fields travel at the speed of light
\end{itemize}

\subsection{Superconductivity}
\begin{itemize}
    \item \textbf{Superconducting} - Occursat very low temperatures, this is when the resistivity of a material becomes nearly 0
    \item Occurs at a critical temperature, \(T_C\) usually within a few degrees of absolute 0
    \item Current work is focused on making materials with higher critical temperature to remove the need for intense cooling 
\end{itemize}

\subsection{Electrical Conduction in the Human Nervous System}
\begin{itemize}
    \item \textbf{Neuron} - the most basic element of the nervous system, transmits electricl signals
    \item \textbf{Dipole Layer} - Difference in charge across the cell membrane of a neuron, resting potential across the layer is \[V_{inside}-V_{outside}\]
    \item \textbf{Action Potential} - The voltage across a neuron when it is "fired"
    \item Ions are the main driver behind neuron activation
\end{itemize}

\newpage