\section{Introduction, Measurement, Estimating}
Physics is the most basics of sciences, divided into \textbf{classical physics} (ex: motion, fluids, light) and \textbf{modern physics} (ex: relativity, quantum theory, astrophysics). It is important to review how science is done before getting into it.

\subsection{The Nature of Science}
    \begin{itemize}
        \item \textbf{Observations} - Things we notice, can include experiments, require imagination
        \item \textbf{Theories} - Formed Around observations in an attempt to explain phenomena
        \item \textbf{Testing} - Experiments or other assessments to determine if predictions based off of theories are true
        \item Theories are accepted or rejected based off of data from testing
    \end{itemize}
	
\subsection{Physics and its Relation to Other Fields}
    \begin{itemize}
        \item Over time as science developed, different fields broke away from physics and established themselves (ex. life sciences and chemistry which are not defined under physics but were once part of it)
        \item Other sciences can apply physics, but that does not mean they are the same field (ex. architecture uses physics but is its own field)
    \end{itemize}
	
\subsection{Models, Theories, and Laws}
    \begin{itemize}
        \item \textbf{Model} - Analogy used to approximate a mental image 
        \item \textbf{Theory} - More broad than a model, gives more detail and testable predictions for phenomena
        \item \textbf{Law} - Used for general statements about nature's behavior, often in the form as a relation between 2 quantities
        \item \textbf{Principle} - Used for less general, more precise statements, such as Archimedes' Principle
        \item Scientific laws describe how nature \emph{does} behave rather than how it \emph{should} behave, and are subject to change if new evidence is presented.
    \end{itemize}
	
\subsection{Measurement and Uncertainty; Significant Figures}
\textbf{Uncertainty}
    \begin{itemize}
		\item \textbf{Estimated Uncertainty} - Gives a range of values that a measurement could fall between
        \begin{itemize}
            \item Example: A ruler's who's smallest measurement is 1mm could have an uncertainty of \(\pm\)0.1cm and a measurement of 5.3cm could have a true value between 5.2cm and 5.4cm
        \end{itemize}
        \item \textbf{Percent Uncertainty} - The ratio between the uncertainty and measured value expressed as a percent. Can be found with
        \[\frac{Uncertainty}{Measured Value}*100\%\]
    \end{itemize}
\textbf{Significant Figures}
	\begin{itemize}
		\item SigFigs (or significant digits) Represent reliably known digits in measurements and calculations
		\item To count SigFigs, there are several rules:
		\begin{itemize}
		   \item Any digits from 1-9 are significant
		   \item Any 0's between digits from 1-9 are significant
		   \item Any 0's to the right of a sigfig and to the left of a decimal point are significant
		   \item Any 0's trailing after a decimal  point are significant
        \end{itemize}
        \item Examples -
        \begin{itemize}
            \item \textbf{505} has \textbf{3} significant digits
            \item \textbf{1}00 has \textbf{1} significant digit
            \item \textbf{1200.} has \textbf{4} significant digits
            \item 0.0\textbf{16000} has \textbf{5} significant digits
        \end{itemize}
	\end{itemize}
\textbf{Scientific Notation}
    \begin{itemize}
        \item Used to display very large or very small numbers using powers of 10
        
        \begin{itemize}
            \item Example: 45,000 in scientific notation is expressed as \(4.5*10^4\)
        \end{itemize}
        \item Remember to always have one significant figure to the left of the decimal
    \end{itemize}
    
\textbf{Percent Uncertainty vs. Significant Figures}
    \begin{itemize}
        \item There are exceptions to the significant figures rules
        \item When calculating percent uncertainty display the digits to the same precision as the uncertainty
        \item Add an extra digit if it gives a more realistic \% uncertainty
    \end{itemize}
    
\textbf{Approximations}
    \begin{itemize}
        \item We often cannot solve problems precisely
        \item Be aware an approximation will be nowhere near as precise as a true result
    \end{itemize}

\textbf{Accuracy vs. Precision}
    \begin{itemize}
        \item \textbf{Accuracy} - How close a value is to the "true" or "accepted" value
        \item \textbf{Precision} - how close different measured values are to each other
    \end{itemize}

\subsection{Units, Standards, and the SI System}
    \begin{itemize}
        \item \textbf{Unit} - Tells what is being measured and must be stated along with values (ex. kilograms tells mass is being measured)
        \item \textbf{Standard} - A defined quantity of a unit that can be reproduced easily for use in experiments and testing
    \end{itemize}
\textbf{Length}
    \begin{itemize}
        \item International standard unit is the \textbf{meter}
        \item Defined as how long light travels in 1/299,792,458 seconds.
    \end{itemize}
    
\textbf{Time}
    \begin{itemize}
        \item Standard is the \textbf{second}
        \item Currently defined as the amount of time a cesium atoms takes to oscillate 9,192,631,770 times.
    \end{itemize}
    
\textbf{Mass}
    \begin{itemize}
        \item Standard is the \textbf{kilogram}
        \item Defined a platinum-iridium cylinder whose mass is exactly 1kg
    \end{itemize}
    
\textbf{Unit Prefixes}
    \begin{itemize}
        \item Used to quantify very large or very small amounts of a unit
        \item Example
        \begin{itemize}
            \item Kilo- means 1000 so 1 kilometer is 1000 meters
        \end{itemize}
    \end{itemize}

\textbf{Systems of Units}
    \begin{itemize}
        \item Used to make equation writing easy by having consistent units
        \item \textbf{Systeme International} - French standardized system of units used globally 
    \end{itemize}

\textbf{Base vs. Derived Quantities}
    \begin{itemize}
        \item \textbf{Base Units} - Fundamental units of measurement used to derive other units, listed as:
        \begin{itemize}
            \item Meter (length)
            \item Second (time)
            \item Kilogram (mass)
            \item Ampere (electricity)
            \item Kelvin (temperature)
            \item Mole (amount of a substance, similar to a dozen)
            \item Candela (luminous intensity)
        \end{itemize}
        \item \textbf{Derived Quantity} - Quantities defined in terms of the 7 base quantities listed before
        \begin{itemize}
            \item Example - The newton is defined as: \[\frac{kg*m}{s^2}\]
        \end{itemize}
        \item \textbf{Operational Definition} - The rule or procedure used to define a base or derived unit
    \end{itemize}

\subsection{Converting Units}
    \begin{itemize}
        \item Measurements consist of a value and a unit, to go from one unit to another, use a conversion factor
        \item \textbf{Conversion Factor} - Ratio between two units that can be used to go from one to the other
        \begin{itemize}
            \item Example - The conversion factor between inches and centimeters is \(\frac{2.54cm}{in}\), to determine how many cm are in 8.00 inches, use the conversion factor: \[8.00in*\frac{2.54cm}{in}=20.3cm\]
            \item Note that the inches units cancel out leaving only centimeters, and that the answer was adjusted to 3 significant figures
        \end{itemize}
    \end{itemize}
	
\subsection{Order of Magnitude: Rapid Estimating}
    \begin{itemize}
        \item \textbf{Order-of-Magnitude Estimating} - A technique of estimating that rounds the answer to a single sigfig and its power of 10 (order of magnitude)
        \begin{itemize}
            \item Example - Estimate the thickness of a sheet of paper from a book that is 300. pages long and 2.40cm thick
            \item To solve, divide the thickness by the number of pages: \[\frac{2.40cm}{300. sheets}=0.00800\frac{cm}{sheet}=8.00*10^{-3}\frac{cm}{sheet}\]
        \end{itemize}
    \end{itemize}
	
\subsection{Dimensions and Dimensional Analysis}
    \begin{itemize}
        \item \textbf{Dimensions} - "Type" of quantity being measured 
        \begin{itemize}
            \item Kilograms measure the dimension of mass, abbreviated as \([M]\)
            \item Square meters measure the dimension of area, abbreviated as \([L^2]\)
        \end{itemize}
        \item \textbf{Dimensional Analysis} - Used to verify a relationship, both sides of an equation must have the same dimensions
        \begin{itemize}
            \item Example - To verify a relationship between Newtons ($\frac{kg*m}{s^2}$) vs mass (kg) and acceleration ($\frac{m}{s^2}$), write the units in their base dimensions and compare:
            \[\frac{kg*m}{s^2}=kg*\frac{m}{s^2}\Rightarrow\frac{kg*m}{s^2}=\frac{kg*m}{s^2}\]
            The dimensions are correct, the left and right sides are equal
        \end{itemize}
        \item Dimensional analysis can also be used to verify derived equations you are unsure about
        \item Note that some units, such as radians, are dimensionless and should therefore not be included in dimensional analysis
    \end{itemize}
	
\newpage