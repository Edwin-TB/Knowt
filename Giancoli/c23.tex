\section{Light: Geometric Optics}
AN object can be seen by either seeing light directly from it or seeing a reflection. The interactions between light and object are discussed in this section.

\subsection{The Ray Model of Light}
\begin{itemize}
    \item \textbf{Ry Model of Light} - A model that assumes light travels in stright lines called rays
	\item When we see an ibject, light travels in countless directions when it hits the object and reflects back
	\item \textbf{Geometric Optics} - The subject that studies the ray nature of light and how it interacts with objects
\end{itemize}

\subsection{Reflection; Image Formation by a Plane Mirror}
\begin{itemize}
    \item \textbf{Angle of Incidence} - The angle between a ray of light striking a surface and a line normal to the surface being hit
    \item \textbf{Angle of Reflection} - The angle between a reflected light ray and a line normal to the surface being reflected
    \item \textbf{Law of Reflection} - Angle of reflection equals angle of incidence \[\theta_r=\theta_i\]
    \item \textbf{Diffuse Reflection} - Occurs when light hits a rough surface and reflects in many directions since the angle of the surface is different across
    \item \textbf{Specular Reflection} - Reflection on a mirror, light reflects at the same angles and will only hit your eyes at the right angle
    \item \textbf{Image} - A reflection of objects in a mirror
    \item \textbf{Plane Mirror} - A mirror with a single flat reflective surface
    \item \textbf{Image Point} - THe point from which a set of diverging rays appears to come from when an object is reflected in a mirror
    \item \textbf{Image Distance} - The distance at which an image in a mirror appears to be from the mirror
    \item Image distance = Object distance
    \item \textbf{Virtual Image} - The image seen and reflected from a plane mirror
    \item \textbf{Real Image} - An image where light passes through the image and can appear on a white surface, or on film
\end{itemize}

\subsection{Formation of Images by Spherical Mirrors}
\begin{itemize}
    \item \textbf{Spherical Mirror} - A mirror whose surface is curved i the shape of a sphere or a section of one
    \item \textbf{Convex} - Describes a spherical mirror where the reflective surface is on the outside of the spherical shape
    \item \textbf{Concave} - Describes a spherical mirror where reflection takes place on the inside
\end{itemize}

\textbf{Focal Point and Focal Length}
\begin{itemize}
    \item Light rays from a distant object will strike a concave mirror precisely parallel
    \item \textbf{Focus} - The point at which rays that strike a concave mirror will cross
    \item \textbf{Principal Axis} - The straight line perpendicular to the curved surface at its center, light rays must be parallel to this axis in order to have a focus
    \item \textbf{Focal Point} - The point where reflected rays come to a focus
    \item \textbf{Focal Length} - The length between the focal point and center of a concave mirror
    \item \textbf{Paraxial Rays} - Rays that make small angles with the principal axis
    \item Focal length is half the radius of curvatuce \[f=\frac{r}{2}\]
    \item Focii are approximations, defects in focus on a spherical mirror are called \emph{spherical aberrations}
    \item \textbf{Parabolic Reflector} - A mirror shape that will reflect rays at a perfect focus
    \item In diagrams of light reflections, the focal point is labelled F and the center of curvature is labelled C
\end{itemize}

\textbf{Image Formation - Ray Diagrams} 
\begin{itemize}
    \item For an object at infinity, the image is locaed at that focal point of a concave spherical mirror
    \item Constructing a real image of an object requires having a point on it be parallel to the principal axis bet wen points C and F
    \item A point, O' is used to project rays and form a real image and is right above point O, where the top of the object at O is
    \item To form an image from O', draw 3 rays:
    \begin{itemize}
        \item A ray leaving O' parallel to the principal axis
        \item A ray leaving O' and crossing through point F
        \item A ray leaving O' along the radius of the spherical surface
    \end{itemize}
    \item These three rays then converge at a point, I', which is the real image
\end{itemize}

\textbf{Mirror Equation and Magnification}
\begin{itemize}
    \item \textbf{Object Distance} - The distance an object is from the center of a spherical mirror, represented by \(d_o\)
    \item \textbf{Image Distance} - The distance an object's virtual image is from the center of a spherical mirror, represented by \(d_i\)
    \item The height of the object is represented by \(h_o\) 
    \item The height of the image, \(I'I\) is represented by \(h_i\) 
    \item \textbf{Mirror Equation} - Relates the object and image distances to the focal length, \[\frac{1}{d_o}+\frac{1}{d_i}=\frac{1}{f}\]
    \item \textbf{Magnification} - Defined as the height of the image divided by the height of the object \[m=\frac{h_i}{h_o}=-\frac{d_i}{d_o}\]
    \item Two sign conventions when dealing with magnification are:
    \begin{itemize}
        \item The image height is positive if the image is upright and negative if it si s inverted relative to the object
        \item \(d-i\) or \(d_o\) is positive if the image or object is in front to of the mirror, negative if either is behind
    \end{itemize}
\end{itemize}

\textbf{Seeing the Image; Seeing Yourself}
\begin{itemize}
    \item Your eyes must intercept the rays appraoching you in order fo ryou to see an image f yourself in a mirror
    \item For a concave mirror, placing your eyes between points O and I, \emph{converging} rays would hit your eyes and the image would be diffifcult to discern
    \item If you are behind point C then you will see a clear, inverted image of yourself
\end{itemize}

\textbf{Convex Mirrors}
\begin{itemize}
    \item Analysis of concave mirrors can be applied to convex mirrors, even the mirror equation
    \item Any reflected rays diverge but seem to com from a point F behind the mirror, called its focal point
    \item \textbf{Focal Length} - The distance an focal point is from the center of the mirror
    \item The mirror equations hold but focal length and radius of curvature are considered to be negative
    \item To solve mirror problems:
    \begin{itemize}
        \item Draw a ray diagram
        \item Apply the mirror and magnifications equations
        \item Check for sign conventions
    \end{itemize}
\end{itemize}

\subsection{Index of Refraction}
\begin{itemize}
    \item The speed of light in air of other materials is slower than it is in a vacuum
    \item \textbf{Index of Refraction} - The ration between the speed of light in a vacuum and a given material, never less than 1, given by \[n=\frac{c}{v}\]
    \item Light travels slower in materials than in a vacuum due to absorption and re emission of light by atoms and molecules in the material
\end{itemize}

\subsection{Refraction: Snell's Law}
\begin{itemize}
    \item \textbf{Refraction} - The change in direction of a light ray due to it entering a new medium
    \item \textbf{Angle of Incidence} - The angle at which a ray strikes a new medium and a line normal to its surface
    \item \textbf{Angle of Refraction} - The angle between a ray's new direction in the new medium and a line normal to its surface
\end{itemize}

\textbf{Snell's Law}
\begin{itemize}
    \item \textbf{Law of Refraction} - Also called Snell's law, used to relate the angles of incidence and refraction in optics, given by \[n_1\sin{\theta_1}=n_2\sin{\theta_2}\]
    Where \(\theta_1\) is the angle of incidence and \(\theta_2\) is the angle of refraction and \(n_1\) and \(n_2\) are the indices of refraction of the materials in question
    \item Note that when \(n_2>n_1\), light bends towards the normal, and vice versa
\end{itemize}


\subsection{Total Internal Reflection; Fiber Optics}
\begin{itemize}
    \item At a certain angle of incidence, the angle of refraction will be \(90^\circ\) and the ray would skim the surface instead of refracting out into the new medium
    \item \textbf{Critical Angle} - The angle of incidence at which this occurs, given by \[\sin{\theta_c}=\frac{n_2}{n_1}\]
    \item \textbf{Total Internal Reflection} - Occurs when all light is reflected back into the medium it originated from, can only occurs when \(n_2<n_1\)
\end{itemize}

\textbf{Fiber Optics; Medical Instruments}
\begin{itemize}
    \item \textbf{Fiber-Optic Cable} - A bundle of plastic or blass fibers that use total internal reflection to transmit signals in the form of light
    \item Light only glances off the boundaries of the cable and thus is maintained within it
    \item This has been used in transmitting data extremely quickly such as for high-res medical imaging
\end{itemize}

\subsection{Thin Lenses; Ray Tracing}
\begin{itemize}
    \item The axis of a lens is a straight line passing through its center
    \item \textbf{Focal Point} - THe point at which rays travelling parallel to the axis of a lens will intersect after passing through it
    \item Thins lenses are those whose diameter is small relative to its radius of curvature
    \item \textbf{Focal Length} - Distance between the focal point and the center of the lens, the same on both sides of a double-convex lens
    \item \textbf{Focal Plane} - The plane of a double convex lens containing all of its focal points
    \item \textbf{Converging Lens} - A lens that is thicker in the center than at its edges, they focuses light
    \item \textbf{Diverging Lens} - A lens that is thinner in the center than at its edges, they spread light out
    \item \textbf{Lens Power} - Used to determine the strength of a lens, given by \[P=\frac{1}{f}\] Where f is focal length
    \item \textbf{Diopeter} - A unit for lens power, defined as the inverse of a meter, \(1D=1M^{-1}\)
    \item \textbf{Real Image} - Formed by a double convex lens, since rays actually pass through the image
\end{itemize}

\textbf{Diverging Lens}
\begin{itemize}
    \item The image formed by a diverging lens is \emph{virtual}
\end{itemize}

\subsection{The Thin Lens Equation}
\begin{itemize}
    \item \textbf{Thin Lens Equaiton} - Relates image distance to object distance and focal length, given by \[\frac{1}{d_o}+\frac{1}{d_i}=\frac{1}{f}\] Where \(d_o\) is object distance and \(d_i\) is image distance
    \item f is negative for a diverging lens
    \item \(d_i\) is negative when the image i on the same side of the lens as the light comes from
    \item \textbf{Magnification} - Same as the magnification of a mirror, given by \[m=\frac{h_i}{h_o}=-\frac{d_i}{d_o}\]
\end{itemize}

\subsection{Combinations of Lenses}
\begin{itemize}
    \item When analyzing multiples lenses, the image formed by the first lens then becomes the object of the second lens, the image of the second lens becomes the object of the third, and so on
    \item The total magnification of multiple lenses is the product of the individual magnifications
\end{itemize}

\subsection{Lensmaker's Equation}
\begin{itemize}
    \item \textbf{Lensmaker's Equation} - Relates focal length of a lens to its radii of curvature, and index of refraction, given by \[\frac{1}{f}=(n-1)(\frac{1}{R_1}+\frac{1}{R_2})\] Where n is the index of refraction and \(R_1\) and \(R_2\) are the radii of curvature of both sides
\end{itemize}

\newpage
