\section{Forces and Newton's Laws}

\subsection{Vocabulary}
\begin{itemize}
    \item Force - A push or pull applied by one object and experienced by another
    \item Net Force - The sum of the force acting on an object. If the net force were applied instead of individual force, it would produce the same effect as the individual forces
    \item Weight - The force an object experiences due to gravity
    \item Friction - A force that acts on an object from a surface, acts parallel to the surface of an object
    \item Kinetic Friction - Friction that acts on an object in motion along a surface, acts opposite to the direction of motion
    \item  Static Friction - Friction that acts on an object at rest
    \item Normal Force - The force of a surface on an object, acts perpendicular to the surface
    \item Coefficient of friction - A constant between two surface that tells how strong friction between them is
    \item - Newton's third Law - States the force of object A on object B is equal and opposite to the force of object B on object A
    \item - Newton's Second Law - States an object's acceleration is is the net force acting on it divided by its mass
\end{itemize}

\subsection{Free Body Diagrams}
\begin{itemize}
    \item Objects cannot "have" force, the can only experience and apply it
    \item Free-body diagrams are used to show what forces act on an object in what direction
    \item Free-body diagrams should have:
    \begin{itemize}
        \item A labeled arrow for each force, with its tail starting at the object and its head pointing in the direction of force
        \item A list of all forces acting on the object, indicating the object applying the force and the object experiencing the force
        \item The phrase "draw and label the forces that act on [object]" means "draw a free-body diagram"
        \item The only forces that act on an object without contact are the gravitational and electrical force
        \item When listing and determining forces, begin with the force of gravity acting on the object and call it "weight", then, think of everything else the object is touching and any force they may exert on the object
    \end{itemize}
\end{itemize}

\subsection{Determining Net Force}
\begin{itemize}
    \item To add forces, treat each direction separately, add forces that point in the same direction, and subtract forces that point in opposite directions
    \item When an object moves along a surface, it is not accelerating perpendicular to the surface and thus experiences no net force perpendicular to the surface
    \item Note that the normal force is not always equal to an object's weight as there may be other forces acting perpendicular to the surface
    \item The force of kinetic friction is defined as the coefficient of friction times the normal force: \[F_f=\mu_k*F_n\]
    \item Make sure you use the coefficient of static friction for a stationary object and the coefficient of kinetic friction for a moving object
\end{itemize}

\subsection{Newton's Third Law}
\begin{itemize}
    \item The equal and opposite force given by Newton's Third Law is called the "companion force"
    \item To find the companion force, reverse the objects acting on each other and reverse the direction of the original force
\end{itemize}

\subsection{Forces at Angles}
\begin{itemize}
    \item If a net force has vertical and horizontal components, use the Pythagorean theorem to determine the value of the combined force
    \item Drawing a force at an angle on a free body diagram is the same as any other force, but you break the force into its components before doing analysis
    \item When an angle, \(\theta\) if formed between the direction of the force and the horizontal:
    \begin{itemize}
        \item The horizontal component of the force is: \(\cos{\theta}*F\)
        \item The vertical component of the force is: \(\sin{\theta}*F\)
    \end{itemize}
    \item Using \(F=ma\), we see that the acceleration of an object with multiple forces acting on it is: \[\frac{F_{net}}{m}=a\]
    \item ALWAYS remember that only the net force is equal to an object's mass times its acceleration, individual forces cause individual accelerations that cancel out
\end{itemize}

\subsection{Inclinced Planes}
\begin{itemize}
    \item A similar process occurs for analysis of objects on an incline
    \item The normal force will be perpendicular to the incline
    \item Friction will act parallel to the incline
    \item Gravity will act at an angle, so it will be broken into components
\end{itemize}

\subsection{Multiple Objects}
\begin{itemize}
    \item When two objects are connected over a pulley, it is easy to consider them one system and use \(a=
    \frac{F_{net}}{m}\) to find the acceleration
    \item If you need to describe one object, then you can draw a free-body diagram for just that one
    \item Note that one rope has one force of tension
\end{itemize}