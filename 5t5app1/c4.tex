\section{Work and Energy}

\subsection{Vocabulary}
\begin{itemize}
    \item Kinetic energy - possess by any object in motion, a scalar value
    \item Translational KE - Exists when an object's center of mass is moving, defined as \[\frac{1}{2}*m*v^2\]
    \item Rotational KE - Exists when an object rotates, defined as \[\frac{1}{2}*I*\omega^2\]
    \item Gravitational Potential Energy - Energy stored in a gravitational field. Near a planet it is defined as \[m*g*h\]
    Far from a planet it is defined as \[-G\frac{M_1*M_2}{d}\]
    \item Elastic Potential Energy - Energy stored in a spring, defined as \[\frac{1}{2}*k*x^2\]
    \item Internal Energy - Relates to the energy stored in a multi-object system
    \item Microscopic Internal Energy - Relates to the temperature of an object, caused by the vibrational energy of its particles
    \item Internal Energy of a two-object system - Another way of saying potential energy
    \item Mechanical Energy - Sum of potential and kinetic energies
    \item Work - done when a force acts on something that moves a distance parallel to that force
    \item Power - Energy used / work done per second
\end{itemize}

\subsection{Energy}
\begin{itemize}
    \item Note the difference between object and systems, a single object alone cannot have potential energy as it is the result of the interaction between multiple objects
    \item Similarly, an object cannot store elastic potential energy, only the spring-object system can
\end{itemize}

\subsection{Work}
\begin{itemize}
    \item Work is done on an object when a force is exerted on it and the object moves parallel to the force, work is calculated with: \[W=F*\Delta x_{||}\]
    \item When an object moves in the direction of force, work is positive
    \item When an object moves opposite in the direction of force, work is negative
    \item For example, a string pulling on a box does positive work while the friction acting against the box's movement does negative work
    \item If a force acts at an angle to movement, split the force into components, one parallel to movement and the other perpendicular to movement
    \item To find net work done on an object, add up the work done by all forces on an object
    \item A conservative force converts potential energy to other types of mechanical energy, thus it does not change the total mechanical energy of a system
    \item The amount of work done by a conservative force is path independent, it only depends on the starting and end point (like displacement)
    \item A non-conservative force changes the mechanical energy of a system
    \item Friction is an example of a non-conservative force, work done by friction becomes microscopic internal energy and becomes heat, which cannot be recovered to convert back into kinetic energy
\end{itemize}

\subsection{The Work-Energy Theorem}
\begin{itemize}
    \item Since the work done by non conservative forces changes the total mechanical energy of a system, it is the sum of the change in kinetic energy and change in potential energy, shown below \[W_{NC}=\Delta KE+\Delta PE\]
    \item Note that there are only 3 ways to approach a mechanics proble: kinematics/Newton's Laws, momentum, and energy
    \item When there is a change in force, there is a change in acceleration, thus the kinematics equations would not work
    \item When there is no collision, there is likely no change in momentum and thus that approach would not work
    \item When a force changes as it is being applied, the definition of work equation would not work as it uses a constant force
    \item The work-energy theorem is the best approach when a force is changing as it is applied
\end{itemize}

\subsection{Power}
\begin{itemize}
    \item Power is a measure of how much energy something can output or how much work it can do per unit time
    \item Power has units joules per second or watts and is defined as \[P=\frac{W}{t}\]
\end{itemize}