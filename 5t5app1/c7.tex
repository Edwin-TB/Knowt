\section{Electricity: Coulomb's Law and Circuits}

\subsection{Vocabulary}
\begin{itemize}
    \item Charge - A scalar quantity that can either be positive or negative and refers to the excess electrical charge an object possesses as a result of electron imbalance
    \item Coulomb - Unit used to measure charge
    \item Current - Flow of charge over time
    \item Ampere - Unit for current, defined as one coulomb per second
    \item Resistance - A measure of how difficult it is for charge to flow through a circuit, measured in ohms
    \item Resistivity - The resistance of one cubic meter of a material
    \item Voltage - Electrical potential energy per coulomb of charge
    \item Series - Describes resistors connected in a single path
    \item Parallel - Describe resistors connected such that current divides and converges before and after flowing through the resistors
\end{itemize}

\subsection{Electric Charge}
\begin{itemize}
    \item The two particles that carry charge are protons and electrons, which carry positive and negative charge, respectively
    \item The charge of one proton is \(1.6*10^{-19}C\), the charge of one electron is \(-1.6*10^{-19}C\)
    \item Most objects are neutrally charged, that is, they have the same number of protons as electrons
    \item Some objects had more protons than electrons and are positively charged, others have more electrons than protons and are negatively charged
    \item The simple rule for electric charge is: like charges repel, opposite charges attract
    \item Only two types of charge exist, if an answer on the exam suggests a third type, reject the answer
    \item Coulomb's law is used to determine how strongly two charges attract or repel, or, how much force on exerts on another, shown below: \[F=k\frac{Q_1*Q_2}{d^2}\] Where \(k\) is Coulomb's constant, defined as \(9.0*10^9\frac{N*m^2}{C^2}\), the \(Q\)s is the magnitude of each charge, and \(d\) is the distance between the two charges
    \item The law of conservation of charge states that the amount of charge in a system is always the same, similar to other conservation laws
    \item Charge can still be transferred between objects, but the total charge will not change
\end{itemize}

\subsection{Circuits}
\begin{itemize}
    \item A circuit is any path that allows current to flow
    \item Current is defined as the flow of positive charge, and flows from the positive to negative terminals in a circuit
    \item Electricity can only flow when one side of a circuit has a higher potential energy than another
    \item Voltage is a measure of this potential difference per coulomb
    \item The resistance of a wire is given by: \[R=\frac{\rho*L}{A}\] Where \(\rho\) is its resistivity, \(L\) is its length, and \(A\) is its cross-sectional area
    \item In series circuits, all resistors carry the same current but different voltages, which all add up to the total voltage across the circuit
    \item In parallel circuits, resistors carry the same voltage but different currents, which all add up to the total current flowing in the circuit
    \item Consider making a V-I-R chart to determine the voltage, current, and resistance of each resistor in a circuit
    \item The total resistance of resistors connected in series is the sum of the individual resistances, shown below: \[R=R_1+R_2+...\]
    \item The total resistance of resistors connected in parallel is the reciprocal of the sum of the reciprocals of the individual resistances, shown below: \[\frac{1}{R}=\frac{1}{R_1}+\frac{1}{R_2}+...\]
    \item Ohm's law relates the voltage, current, and resistance that flows through circuit elements, shown below: \[V=IR\]
    \item Ohm's law can only be used across a single row in a V-I-R chart
    \item Ohm's law and the rules for current and voltage for series and parallel circuits can be used to solve just about any simple circuit problem
\end{itemize}

\subsection{Kirchoff's Laws: Conservation of Charge and Energy}
\begin{itemize}
    \item Kirchoff's junction law states that the current entering a wire junction equals the current leaving the junction
    \item The junction law conserves charge as none is created or destroyed, what enters leaves
    \item Kirchoff's loop law states that the sum of voltage changes around a circuit loop is zero
    \item The loop law conserves energy as potential energy is not creates or destroyed
\end{itemize}

\subsection{Power in a Circuit}
\begin{itemize}
    \item Resistors convert electrical energy to some other form of energy, usually heat
    \item The power dissipated by a resistor is given by \[P=I*V\] Rearranging the above with ohm's law also gives power as: \[P=I^2*R=\frac{V^2}{R}\]
    \item Most circuit components besides voltage sources can be thought of as resistors
    \item The brightness of a light bulb depends on how much power it can dissipate, higher-watt bulbs are brighter
\end{itemize}

\subsection{Ammeters and Voltmeters}
\begin{itemize}
    \item Ammeters are devices used to measure current, voltmeters are devices used to measure voltage
    \item Remember that voltage is constant for all resistors in a parallel circuit, thus voltmeters are put in parallel to the resistor being measured
    \item Current is the same for all resistors in a series circuit, so ammeters are put in series with the resistor being measured
\end{itemize}