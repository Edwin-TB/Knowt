\section{Collisions: Impulse and Momentum}

\subsection{Vocabulary}
\begin{itemize}
    \item Momentum - An object's mass times its velocity, a vector in the direction of an object's movement
    \item Impulse - Defined as the product of a net force and the time during which the net force acts, also equal to the change in an object's momentum
    \item System - Made up of a group of objects that can be treated as a single thing. Important to define the system in question at the beginning 
    \item - Elastic Collision - A collision where kinetic energy is conserved
\end{itemize}

\subsection{The Impulse-Momentum Theorem}
\begin{itemize}
    \item The theorem is: \(\Delta\rho=F*\Delta t\)
    \item A force-time graph usually indicates you will be finding impulse, which is the area under the graph and the horizontal axis, best estimated with triangles and rectangles
    \item Impulse on its own does not say much, to find an object's final momentum you still need its initial momentum, to find its velocity you need its mass
    \item In any system where the only forces are acting between objects in the system, momentum is conserved
    \item When defining your initial and final velocities in a momentum problem, denote the final velocities and momentums with an apostrophe. Then, apply the conservation of momentum equation:
    \begin{itemize}
        \item Since momentum is conserved, the total change is 0, thus: \[0=\Delta\rho_A+\Delta\rho_B\]
        \item Then, split up the changes in momentum into the final and initial momentums: \[0=(\rho_A'-\rho_A)+(\rho_B'-\rho_B)\]
        \item You can then split this into the individual masses and velocities: \[0=(m_A*v_A'-m_A*v_A)+(m_B*v_B'-m_B*v_B)\]
        \item If the two objects stick together, then their final velocities are equal to each other, and can be treated as the same variable: \[v_A'=v_B'\]
    \end{itemize}
\end{itemize}

\subsection{When is the Momentum of a System not Conserved?}
\begin{itemize}
    \item The momentum of a system is not conserved if an object outside the system exerts a force on it
    \item Note that a change in the direction of momentum means conservation is not conserved. If a ball rebounds with the same velocity as it hit something, momentum was not conserved
    \item Momentum is only conserved when considering a single collision, if multiple collisions occur, momentum may or may not be conserved depending on what objects are within your system
\end{itemize}

\subsection{Elastic/Inelastic Collisions}
\begin{itemize}
    \item In an elastic collision, the total kinetic energy of both objects is the same before and after (A good way to check is by comparing total kinetic energy before and after)
    \item Always try to start a problem with momentum, only move onto kinetic energy if you have to
    \item Remember that just because two object bounce off of each other, it does not mean the collision is elastic
    \item Also remember that kinetic energy is a scalar, so it does not have direction and kinetic energies in different directions do not cancel out
\end{itemize}

\subsection{2D collisions}
\begin{itemize}
    \item When dealing with collisions in two dimensions, analyze the momentum in each direction separately
    \item If there is movement in one dimension after a collision but none in that dimension before the collision, then the total momentum in that dimension must equal 0 for it to be conserved
\end{itemize}

\subsection{Motion of the Center of Mass}
\begin{itemize}
    \item The center of mass of a system obeys Newton's second law
    \item If an astronaut pulls on an asteroid, then the two will collide at the center of mass between the two, so it will not accelerate 
    \item If a rocket is set to land 30m from its launch point splits up into two pieces in the air and one of its pieces lands 35 meters from the launch point, the other must land 25m from the launch point in order to maintain the center of mass' original path the same
    \item The center of mass of a system is given by \[x_{cm}=\frac{m_1*X_1+m_2*X_2+...}{M}\] Where \(M\) is the total mass of the system
\end{itemize}