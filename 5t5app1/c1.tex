\section{How Things Move}

\subsection{Vocabulary}
\begin{itemize}
    \item Position - Where an object is, relative to another (usually the origin or some other zero-point)
    \item Speed - How fast an object is moving
    \item Acceleration - The rate of change of an object's speed
    \item Displacement - The change in an object's position from start to finish, excluding any other motion made
    \item Position-Time Graph - A graph where time is on the x-axis and position is on the y-axis. Slope is speed
    \item Velocity-Time Graph - A graph where time is on the x-axis and velocity is on the y-axis. Slope is acceleration
    \item Free fall - Described motion where the only force acting on an object is gravity and its x velocity is 0
    \item Projectile Motion - Described motion where the only force acting on an object is gravity and its x velocity is non-zero but constant
\end{itemize}

The five most important motion variables are:
\begin{itemize}
    \item \(v_0\) - Initial velocity
    \item \(v_f\) - Final velocity
    \item \(\Delta x\) - Displacement
    \item \(a\) - Acceleration
    \item \(t\) - Time
\end{itemize}
If you know three of the five, you can use your equations to find the last two

\subsection{Graphical Analysis of Motion}
\begin{itemize}
    \item A common mistake is people confuse position-time and velocity-time graphs, DO NOT DO THIS
    \item A position-time graph gives an object's position on the vertical (y) axis
    \item The slope of a position-time graph is the object's speed
    \begin{itemize}
        \item Steeper slope means the object is moving faster
        \item If the slope is positive, the object is moving forward
        \item If the slope is negative, the object is moving backward
    \end{itemize}
    \item You may not know the exact slope of a position-time graph, just use your best guess
    \begin{itemize}
        \item Questions that ask for this want to see that you get the basic idea, not the exact answer
    \end{itemize}
    \item On a velocity-time graph, the object's speed is given on the vertical (y) axis
    \item The direction of motion on a velocity time graph is given by the sign of the vertical coordinate at any point in time
    \item The acceleration of an object is given by the slope of its velocity-time graph
    \item It is important to note that acceleration does not tell you if something is speeding up or slowing down
    \begin{itemize}
        \item If something is moving backwards and has a negative acceleration, it is speeding up in the negative direction
        \item Be careful not to make assumptions, only draw conclusion from information you know definitively
    \end{itemize}
    \item The area between a velocity-time graph and the horizontal axis gives the displacement of an object, which only tells you how far an object ended up from its starting position, NOT where it ends up
\end{itemize}

\subsection{Algebraic Analysis of Motion}
Algebraic analysis is different from geometric analysis in that you are given a description of a problem. The most important things to extract from this description are:
\begin{itemize}
    \item A positive direction
    \item What you know
    \item What you do not know
    \item What you WANT to know
    \item A start and end time
\end{itemize}

A good strategy is to make a table of known and unknown values with the same five variables from the previous section to better organize your problem solving. Caltulate the missing variables with the kinematic equations:
\begin{itemize}
    \item \(v_f=v_0+a*t\)
    \item \(\Delta x = v_0*t+\frac{1}{2}*a*t^2\)
    \item \(v_f^2=v_0^2+2*a*\Delta x\)
\end{itemize}

\subsection{Free Fall and Projectile Motion}
\begin{itemize}
    \item Remember that objects in free fall always have acceleration due to gravity, usually rounded to -10\(m/s^2\) in the vertical direction, and no horizontal velocity or acceleration
    \item Objects in projectile motion have two velocities, a horizontal velocity that remains constant and a vertical velocity that changes due to gravity. This means you should have two tables of known and unknown variables for objects in projectile motion, one for each velocity
    \item The two velocities for projectile motion also use the same variable for time as they describe the same object, just in different directions. They must use the same variable for time
    \item The final velocity of an object in free fall or projectile motion is not 0, that is its velocity once it is at rest
    \item The true final velocity is the object's velocity the instant before it hits the ground
\end{itemize}
