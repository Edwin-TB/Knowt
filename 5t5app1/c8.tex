\section{Waves and Simple Harmonic Motion}

\subsection{Vocabulary}
\begin{itemize}
    \item Period - The time one full cycle of simple harmonic motion takes to complete
    \item Frequency - The number of cycles passing a position in one second, measured in hertz, meaning "per second"
    \item Amplitude - The distance from the midpoint of a wave to its crest
    \item Wavelength - Measured from peak to peak or any successive identical points in a wave
    \item Spring Constant - Measure in newtons per meter, this relates to the stiffness of a spring
    \item Restoring Force - Any force that pushed an object to its equilibrium position
    \item Node - Stationary points on a standing wave
    \item Antinode - Positions in a standing wave with the largest amplitudes
\end{itemize}

\subsection{Harmonic Motion}
\begin{itemize}
    \item Simple harmonic motion refers to back and forth oscillations whose position-time graphs looks like a sine wave
    \item Common examples of SHM are a mass on a spring or a pendulum
    \item The period of a mass on a spring in SHM is given by \[2\pi\frac{\sqrt{m}}{\sqrt{k}}\]
    Where \(m\) is the mass of the mass and \(k\) is the spring constant
    \item Frequency and period are inverse of one another, that is, they are each others reciprocals
    \item The amount of restoring force exerted b a spring is given by \[F=kx\] Where \(x\) is the displacement of the spring equilibrium position
    \item The potential energy of a spring is given by \[PE=\frac{1}{2}*k*x^2\]
    \item To find the maximum kinetic energy from the potential energy of a spring, imagine the instant at which the spring is at its equilibrium point before it continues oscillating, here the potential energy is 0 because the spring's displacement is 0. Therefore, all the potential energy has been converted to kinetic energy. The below can then be used to find the maximum KE and thus maximum speed: \[\frac{1}{2}*k*x^2=\frac{1}{2}*m*v^2\]
    \item Pendulum's can be treated similar to springs, as they stil require an energy approach instead of a kinematics approach
    \item The period of a pendulum is given by: \[2\pi\frac{\sqrt{L}}{\sqrt{g}}\] Where \(L\) is the pendulum's length and \(g\) is the force of gravity
\end{itemize}

\subsection{Waves}
\begin{itemize}
    \item The only waves AP Physics 1 covers is mechanical waves, such as sound, string, and ocean waves
    \item A transverse wave is one in which the motion of material is perpendicular to the motion of the wave, such as an ocean wave
    \item A longitudinal wave is one in which mateial moves parallel to the motion of the wave, such as a sound wave or a wave in a spring
\end{itemize}

\subsection{Superposition and Interference}
\begin{itemize}
    \item When two waves collide, they interfere with each other
    \item Constructive interference occurs when waves collide such that their crests overlap, causing the combined crest to have a larger amplitude, the two crests then split up and continue moving in their original direction
    \item Destructive interference occurs when the crest of one wave overlaps with the trough of another, causing both of their amplitudes to drop during their interference
\end{itemize}

\subsection{Standing Waves}
\begin{itemize}
    \item A standing wave is one that appears to stand in one place
    \item Some positions in the string vibrate with a large amplitude, these are antinodes
    \item Some positions in the string of a standing wave stay in one place, these are called nodes
    \item The wavelength of a standing wave is twice the node to node distance
    \item Placing your finger on the center point of a string with a standing wave forces it to have a node there, and half the wavelength, thus twice the frequency
    \item You can do this with any whole number of nodes, the process creates harmonics or whole number multiples of the string's original frequency
    \item For a string fixed on both ends or a pipe open at both ends, the smallest frequency for a standing wave is given by: \[f_1=\frac{v}{2L}\] Where \(v\) is the speed of the waves in the string or pipe, generally the speed of sound in air, which is \(~340m/s\)
    \item The pitch of a note depends on its frequency, its volume depends on its amplitude
\end{itemize}

\subsection{Closed-End Pipe}
\begin{itemize}
    \item When a pipe is closed at one and but open at the other, the standing wave has a node at the closed end and an antinode at the open end, creating a wave whose smallest frequency is given by: \[f_1=\frac{v}{4L}\]
\end{itemize}

\subsection{Beats and the Doppler Effect}
\begin{itemize}
    \item Beats are rhythmic interference creates when two notes of different but close frequencies are played
    \item The interference caused by beats creates a "wa-wa" effect as the frequencies of the notes go in and out of sync
    \item The doppler effect is the change in a wave's frequency that you can observe whenever the source is moving towards or away from the observer
    \item When the sound source is moving closer to you, the waves become squished together as they are produced closer to you, this causes you to perceive a higher frequency
    \item The opposite happens when the sound source is moving away, the sound waves become spread apart and you perceive a lower frequency
    \item 
\end{itemize}