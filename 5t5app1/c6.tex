\section{Gravitation}

\subsection{Vocabulary}
\begin{itemize}
    \item Gravitational Field - Gives how much 1kg of mass weighs at a certain location relative to a planet. At Earth's surface, the field is ~10N/kg
    \item Gravitational Force - The force felt by an object as a result of a gravitational field
    \item Newton's Gravitation Constant - A constant that applies to all objects, defined as \[G=6*10^{-11}\frac{N*m^2}{kg^2}\]
    \item Free-Fall Acceleration - Also called acceleration due to gravity, this is the acceleration an object feels as the result of a planet's gravity
\end{itemize}

\subsection{Determining the Gravitational Field}
\begin{itemize}
    \item The gravitational force is the weakest of the fundamental forces
    \item The gravitational field is a vector quantity that points to the center of the planet
    \item The magnitude of a planet's gravitational field depends on the planet's mass and the distance you are from the planet's center, given by: \[g=G\frac{M}{d^2}\]
    \item If given a planet's measurements in terms of Earth's, you can replace the new distance / mass in the above formula to find \(g\)
\end{itemize}

\subsection{Determining Gravitational Force}
\begin{itemize}
    \item The weight of an object is the product of its mass and the gravitational field it experiences, \(m*g\)
\end{itemize}

\subsection{Force of Two Planets on One Another - Order of Magnitude Estimates}
\begin{itemize}
    \item The gravitational force of one object on another is given by: \[F=\frac{G*m_1*m_2}{d^2}\]
    \item You will likely not be asked to calculate the exact force of gravity between two objects, but rather estimate it by plugging in powers of 10 into the above equation and adding/subtracting exponents for the correct order of magnitude
\end{itemize}

\subsection{Gravitational Potential Energy}
\begin{itemize}
    \item Near the Earth's surface, the gravitational potential energy of an object is given by \[PE=m*g*h\]
    \item In general, the gravitational potential energy between two objects is given by \[PE=G\frac{M_1*M_2}{d}\]
\end{itemize}

\subsection{Gravitational and Inertial Mass}
\begin{itemize}
    \item Gravitational mass is how an objects responds to a gravitational field
    \item Inertial mass is how an object accelerates in response to a net force
    \item In every experiment ever conducted, an object's inertial and gravitational masses have been equal
    \item For the AP exam, anything that does not agree with the above should be rejected as ridiculous
\end{itemize}

\subsection{Fundamental Forces: Gravity versus Electricity}
\begin{itemize}
    \item Gravity and electricity are the only two fundamental force studied in AP Physics 1
    \item ll other forces are a result of these two (friction results from the electrical repulsion between two forces)
    \item At a microscopic scale, the electrical forces between two objects take over and the gravitational forces between them are negligible
    \item The opposite happens are very large scales
\end{itemize}